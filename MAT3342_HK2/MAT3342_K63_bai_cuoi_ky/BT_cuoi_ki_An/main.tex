 \documentclass[11pt, reqno]{amsart}
\usepackage[utf8]{vietnam}
\usepackage{verbatim}
\usepackage{amsmath,amsthm}
\usepackage{amssymb}
\usepackage{latexsym}
\usepackage{exscale}
\usepackage[mathscr]{eucal}
\usepackage{makeidx}
%\usepackage{lscape}
\usepackage{multicol}
\usepackage{afterpage}
\usepackage{graphicx}
\usepackage{color}
 \usepackage{colortbl}
\usepackage[colorlinks,hyperindex,unicode]{hyperref}
%\usepackage{cite}
\usepackage{biblatex}
\usepackage{parallel} \usepackage{framed}
\usepackage{lipsum}
\usepackage{tikz}
\usepackage[left=2.00cm, right=1.00cm, top=2.00cm, bottom=2.00cm]{geometry}

\title{bt xemina 2}
\author{anthule2000 }
\date{June 2021}

 

\begin{document}

Câu 4: Giải điều kiện cần và đủ cho ma trận K $\in$ $\mathbb{R}^{n,n}$ sao cho:

\begin{equation}
    \underset{m\geq0}{\mathlarger{\sum}}
    {\dfrac{1}{(2m)!}(-K)^mt^{2m} = 0   
    
\end{equation}
  
và
\begin{equation}
     \underset{m\geq0}{\mathlarger{\sum}}
        {\dfrac{1}{(2m+1)!}(-K)^mt^{2m+1} \geq 0 
        
\end{equation} 

ở trong 2 trường hợp:
  
\begin{tikzpicture}
    \node[rectangle,rounded corners,draw=black,text width
    =8cm,text ragged, fill=yellow]  (a) at (0,0)
    {
        i) K là vô hướng:
       \begin{itemize}
       \item Nếu $-K\geq0$ thì (1),(2) đều thỏa mãn.
       \item Nếu $-K<0$ thì 
           $-K= (-1)x^2$ 
           
           với x $\geq0$
           
           \begin{allign}
              f(t) = \underset{m\geq0}{\mathlarger{\sum}}
           {\dfrac{1}{(2m)!}(-K)^mt^{2m} \\
           = 
           \underset{m\geq0}{\mathlarger{\sum}}
           {\dfrac{1}{(2m)!}(-1)^m(xt)^{2m}
           =cos(xt)
           
           \end{allign}
            vì t $\in [0, +\infty)$  nên f(t) sẽ <0 với t nào đó. 
        \end{itemize}
        Vậy điều kiện cần và đủ của (1) và (2) là $-K\geq 0$
    }
\end{tikzpicture}
\begin{tikzpicture}
     \node[rectangle,rounded corners,draw=black,text width
    =9cm,text ragged, fill=yellow]  (b) at (3,0)
    {
     ii) K là ma trận vuông, đối xứng:
        \begin{itemize}
            \item Nếu $-K\geq0$ thì (1),(2) đều thỏa mãn.
            \item  Nếu $-K$ không $\geq0$ thì 
                 $$ -K = S^T A S $$ 
                 với S là ma trận trực giao, A là ma trận đường chéo
                 $A = diag(a_1, \dots, a_n)$.
                 
                 $$ (-K)^m = S^T diag(a_1^m, \dots, a_n^m) S $$
                 \begin{allign}
              f(t) = \underset{m\geq0}{\mathlarger{\sum}}
           {\dfrac{1}{(2m)!}(-K)^mt^{2m} \\
           = 
           \underset{m\geq0}{\mathlarger{\sum}}
           {\dfrac{1}{(2m)!}S^T diag(a_1^m, \dots, a_n^m) S
         
           \end{allign}
                 
        \end{itemize}
     
     }
\end{tikzpicture}

Câu 5: Giải điều kiện cần và đủ cho ma trận M,D,K,B sao cho hệ bậc 2:
 $$ M\ddot{x}(t) + D\dot{x}(t) + Kx(t) = Bu(t)  \hspace{2cm}(5) $$ với mọi $t\geq 0$ 
 với M $\in$ $\mathbb{R}^{n,n}$, khả nghịch.
 
 Giải: Vì M khả nghịch nên không mất tổng quát, cho M = $I_n$.
 
 b) Là $C_2$-điều khiển được, chuyển từ $(x_0, \dot{x}_0)$  $\in$ $\mathbb{R}^n \times \mathbb{R}^n$ sang $(x_1, \dot{x}_1)$ $\in$ $\mathbb{R}^n \times \mathbb{R}^n$ 
 
 Đặt $ z(t) = \dot{x}(t)$

 \def\xz{
    \begin{bmatrix}
        x(t) \\ z(t)
    \end{bmatrix}}
 \def\dotxz{
\begin{bmatrix}
    \dot{x}(t) \\ \dot{z}(t)
\end{bmatrix}}

\def \A {
\begin{bmatrix}
    0&I_n \\
      -K& -D 

      \end{bmatrix}}
}
\def \nB {
\begin{bmatrix}
   0\\B
\end{bmatrix}
}
\begin{equation}
    \dotxz = \A \xz + \nB u(t)
\end{equation}
Vậy hệ (5) là $C_2$-điều khiển được khi và chỉ khi  $\left[\A, \nB\right] $ trong hệ (3) là C-điều khiển được.



\end{document}







