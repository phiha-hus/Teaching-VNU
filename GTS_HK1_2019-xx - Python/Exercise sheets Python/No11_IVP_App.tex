\documentclass[11pt]{article}
\usepackage{amsmath}
%\usepackage{extsizes}
\usepackage{amsmath,amssymb}
%\usepackage{omegavn,ocmrvn}
%\usepackage[utf8x]{inputenc}
\usepackage[utf8]{vietnam}

\usepackage{listings}
\lstset{language=Python}          % Set your language (you can change the language for each code-block optionally)


\usepackage{longtable}
\usepackage{answers}
\usepackage{graphicx}
\usepackage{array}
\usepackage{pifont}
\usepackage{picinpar}
\usepackage{enumerate}
\usepackage[top=3.0cm, bottom=3.5cm, left=3.5cm, right=2.5cm] {geometry}
\usepackage{hyperref}


\newtheorem{bt}{Câu}
\newcommand{\RR}{\mathbb R}
\Newassociation{sol}{Solution}{ans}
\newtheorem{ex}{Câu}
\renewcommand{\solutionstyle}[1]{\textbf{ #1}.}


\begin{document}
% \noindent
\begin{tabular*}
{\linewidth}{c>{\centering\hspace{0pt}} p{.7\textwidth}}
Trường ĐHKHTN, ĐHQGHN & {\bf Học Kỳ 1 (2019-2020)}
\tabularnewline
K62 TTƯD & {\bf Bài Tập Giải Tích Số. No 11 \\ IVP - 5 phương pháp ẩn/hiện cơ bản}
\tabularnewline
\rule{1in}{1pt}  \small  & \rule{2in}{1pt} %(Due date:)
\tabularnewline

%  \tabularnewline
%  &(Đề thi có 1 trang)
\end{tabular*}
%
% \Opensolutionfile{ans}[ans1]

\begin{bt}
Water flows from an inverted conical tank with circular orifice at the rate
%
\[
x'(t) =  - 0.6 \pi r^2 \sqrt{2g} \dfrac{\sqrt{x}}{Ax} \ ,
\]
%
where r is the radius of the orifice, x is the height of the liquid level from the vertex of the cone,
and A(x) is the area of the cross section of the tank x units above the orifice. Suppose $r = 0.1$ ft,
$g = 32.1 ft/s^2$, and the tank has an initial water level of 8 ft and initial volume of $512(\pi/3) \ ft^3$. \\
a. Compute the water level after 10 min with h = 20 s. \\
b. Determine, to within 1 min, when the tank will be empty.
\end{bt}

\begin{bt}
The irreversible chemical reaction in which two molecules of solid potassium dichromate (K2Cr2O7),
two molecules of water (H2O), and three atoms of solid sulfur (S) combine to yield three molecules of
the gas sulfur dioxide (SO2), four molecules of solid potassium hydroxide (KOH), and two molecules
of solid chromic oxide (Cr2O3) can be represented symbolically by the stoichiometric equation:
%
\begin{equation}
2K_2Cr_2O_7 + 2H_2O + 3S \rightarrow 4KOH + 2Cr_2O_3 + 3SO_2 \ .
\end{equation}
%
If $n_1$ molecules of $K_2Cr_2O_7$, $n_2$ molecules of $H_2O$, and $n_3$ molecules of 
$S$ are originally available, the following differential equation describes the amount x(t) of KOH after time t:
%
\[
x'(t) = k \left( n_1 - \dfrac{x}{2} \right)^2 \left( n_2 - \dfrac{x}{2} \right)^2  \left( n_3 - \dfrac{3x}{4} \right)^3 
\]
%
where k is the velocity constant of the reaction. If $k = 6.22 \cdot 10^{-19}$, $n1 = n2 = 2 \cdot 10^3$, and
$n3 = 3 \cdot 10^3$, how many units of potassium hydroxide will have been formed after 0.2 s?
\end{bt}

\begin{bt} Viết các hàm trong Python để tính tích phân dựa trên 5 phương pháp ẩn và hiện cơ sở, ví dụ 
%
\begin{lstlisting}[frame=single] 
def Euler_exp(f,t0,tf,y_0,h):
return y , t
\end{lstlisting}
%	 
trong đó $f = f(t,y)$ là hàm \quad ; $t \in [t0,tf]$ là khoảng thời gian, $h$ là độ rộng mỗi bước và được cho trước. Chú ý các nút được sử dụng là cách đều, tức là $t_0<t_1<\dots<t_f$, $t_i = a+i*h$, $h=\cfrac{tf-t0}{n}$. 


\end{bt}

\centerline{———————————Hết——————————-}

\end{document}

\vspace{1cm}
\noindent{\bf Chú ý:} {\it Cán bộ coi thi không giải thích gì thêm}\\
\Closesolutionfile{ans}
\newpage
\begin{center}
{\LARGE{\bf ĐÁP ÁN}}
\end{center}

\begin{sol}
	\begin{figure}[h!]
		\centering
		\includegraphics[width=0.8\linewidth]{Solution1/Sol4_1.png}
		%\caption{}
		\label{fig:Sol4}
	\end{figure}
	Exercise 7: Convergence order is 3.	
\end{sol}

   
\end{document}



