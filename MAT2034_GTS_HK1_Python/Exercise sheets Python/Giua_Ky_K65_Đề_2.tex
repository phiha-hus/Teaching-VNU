\documentclass[10pt]{article}
\usepackage{amsmath}
%\usepackage{extsizes}
\usepackage{amsmath,amssymb}
%\usepackage{omegavn,ocmrvn}
%\usepackage[utf8x]{inputenc}
\usepackage[utf8]{vietnam}

\usepackage{longtable}
\usepackage{answers}
\usepackage{graphicx}
\usepackage{array}
\usepackage{pifont}
\usepackage{picinpar}
\usepackage{enumerate}
\usepackage[top=3.0cm, bottom=3.5cm, left=3.5cm, right=2.5cm] {geometry}
\usepackage{hyperref}


\usepackage{listings}
\lstset{language=Python}          % Set your language (you can change the language for each code-block optionally)


\newtheorem{bt}{Câu}
\newcommand{\RR}{\mathbb R}
\Newassociation{sol}{Solution}{ans}
\newtheorem{ex}{Câu}
\renewcommand{\solutionstyle}[1]{\textbf{ #1}.}
\newcommand{\m}[1]{
	\begin{bmatrix}
		#1
	\end{bmatrix}
}

\begin{document}

\begin{tabular*}
	{\linewidth}{c>{\centering\hspace{0pt}} p{.7\textwidth}}
	Trường ĐHKHTN, ĐHQGHN & {\bf Học Kỳ 2 (2021-2022)}
	\tabularnewline
	K64 TTƯD - Thầy Hà Phi & {\bf Bài Tập Giải Tích Số \\ \today}
	% Exercises on pages 239, 240 Cheney/Kincaid are really nice
	\tabularnewline
	\rule{1in}{1pt}  \small  & \rule{2in}{1pt} %(Due date:)
	\tabularnewline
	%  \tabularnewline
	%  &(Đề thi có 1 trang)
\end{tabular*}





\begin{center}
	{\bf Kiểm tra giữa kỳ - ĐỀ 2 - 4 bài - Thời gian 90 phút \\ Được sử dụng Laptop (chế độ máy bay) trong 40 phút cuối \\ Không được phép sử dụng tài liệu/dụng cụ hỗ trợ nào khác ngoài Laptop}
\end{center}

\begin{bt}(2 điểm) \\ % Exercise 22, Kiusalass p.170 
Cho hệ phương trình tuyến tính $Ax = b$ với các ma trận hệ số là 
\[
\verb| A = np.array([[89,59,77], [59,107,59], [77,59,89]]); b = np.array([18, 2, 85])  |
\]
a) Hãy kiểm tra xem hệ trên có điều kiện tốt hay xấu theo chuẩn $\|\cdot\|_{\infty}$. \\
b) Giải hệ trên bằng phương pháp Cholesky (\textbf{phải viết rõ từng bước}).
\end{bt}

\begin{bt}(2 điểm) \textbf{(Lập trình)}\\
Sử dụng các hàm built-in trong Python nếu muốn, hãy giải hệ trong Bài tập 1 bằng cả 5 phương pháp, ghi đáp số ra 1 bảng dạng sau. \textbf{Nếu phương pháp nào không sử dụng được thì cần giải thích tại sao.}
\begin{center}
\begin{tabular}{|c|c|c|c|c|c|}
	\hline
	Tên phương pháp & PLU & Cholesky & QR & SVD & numpy.linalg.solve \\
	\hline
	Nghiệm giải số &  &  &  &  & \\
	\hline
	Sai số tuyệt đối (so với la.solve) &  &  &  &  & \\
	\hline
\end{tabular}
\end{center}
\end{bt}

\begin{bt}(4 điểm) \\
Một cuộc điều tra dân số của Hoa Kỳ được thực hiện 10 năm một lần. Bảng sau liệt kê dân số, tính bằng hàng nghìn người, từ năm 1970 đến năm 2000, và số liệu cũng được trình bày trong bảng sau.
	%
	\begin{center}
			\begin{tabular}[6]{|l|l|l|l|l|l} \hline
			t (thập niên) & 1970 & 1980 & 1990 & 2000   \\ \hline 
			$f$ & 203,302 & 226,542 & 249,633 & 281,422 \\ \hline
		\end{tabular}	
	\end{center}
	%
a) Giả sử rằng dân số là một hàm đa thức. Hãy sử dụng công thức nội suy Lagrange để dự đoán giá trị của dân số tại thời điểm năm 1985. \textbf{Chú ý: không cần phải rút gọn công thức. SV có thể đổi lại mốc thời gian cho $t$ nhỏ đi và dễ tính hơn}. \\
b) Cũng giả sử như trong câu a), hãy đi tìm công thức đa thức nội suy Newton sử dụng bảng tỷ sai phân, và dự đoán giá trị của dân số tại thời điểm năm 1985. \\
c) Biết rằng dân số tăng theo hàm cấp mũ $f(x) = e^{ax^2+bx+c}$, hãy sử dụng phương pháp hồi quy kết hợp với phân tích QR để tìm công thức hàm phù hợp với bảng dữ liệu ở trên. \\
d) \textbf{Lập trình} để giải quyết câu c) và dự đoán giá trị của dân số tại các thập niên 2010s, 2020s.   
\end{bt}

\begin{bt}(2 điểm) \ \textbf{(Lập trình)} \\
Cho $[a,b] = [-10,10]$. Hãy nội suy hàm số $f(x) = \dfrac{1}{1 + x^2}$ sử dụng đa thức nội suy bậc $50$ trên \textbf{lưới đều} và trên lưới với các mốc nội suy Chebyshev-Gauss (lấy $n = 50$) như sau. 
%
\[
x_i = \dfrac{a+b}{2} - \dfrac{b-a}{2} \ \cos\dfrac{(2i+1) \ \pi }{2(n+1)}, \quad i = 0, . . . , n \ .
\]
%
Vẽ các sai số của 2 phép nội suy đó trên cùng 1 đồ thị. Em có kết luận gì từ đồ thị đó?
\end{bt}

\vspace{1cm}
\noindent{\bf Chú ý:} {\it Cán bộ coi thi không giải thích gì thêm}\\

\end{document}
