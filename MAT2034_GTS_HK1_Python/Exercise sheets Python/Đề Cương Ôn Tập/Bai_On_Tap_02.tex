\documentclass[11pt]{article}
%\usepackage{extsizes}
\usepackage{amsmath,amssymb}
%\usepackage{omegavn,ocmrvn}
%\usepackage[utf8x]{inputenc}
\usepackage[utf8]{vietnam}

\usepackage{framed}
\usepackage[most]{tcolorbox}
\usepackage{xcolor}
\colorlet{shadecolor}{orange!15}


\usepackage{longtable}
\usepackage{answers}
\usepackage{graphicx}
\usepackage{array}
\usepackage{pifont}
\usepackage{picinpar}
\usepackage{enumerate}
\usepackage[top=2.0cm, bottom=2.5cm, left=2.5cm, right=1.5cm] {geometry}
\usepackage{hyperref}


\newtheorem{bt}{Câu}
\newcommand{\RR}{\mathbb R}
\Newassociation{sol}{Solution}{ans}
\newtheorem{ex}{Câu}
\renewcommand{\solutionstyle}[1]{\textbf{ #1}.}


\begin{document}

\begin{tabular*}
	{\linewidth}{c>{\centering\hspace{0pt}} p{.7\textwidth}}
	Trường ĐHKHTN, ĐHQGHN & {\bf Học Kỳ 2 (2021-2022)}
	\tabularnewline
	K64 TTƯD - Thầy Hà Phi & {\bf Bài Tập Giải Tích Số \\ \today}
	% Exercises on pages 239, 240 Cheney/Kincaid are really nice
	\tabularnewline
	\rule{1in}{1pt}  \small  & \rule{2in}{1pt} %(Due date:)
	\tabularnewline
	%  \tabularnewline
	%  &(Đề thi có 1 trang)
\end{tabular*}




\begin{center}
	\textbf{Giới hạn các dạng bài tập ôn thi được bao gồm trong các chương sau.}
	\begin{enumerate}
		\item[i)] Giải số hệ phương trình tuyến tính bằng các phương pháp trực tiếp. 
		\item[ii)] Nội suy, bài toán bình phương tối thiểu.
		\item[iii)] Xấp xỉ đạo hàm, tích phân.
		\item[iv)] Giải gần đúng phương trình vi phân. 
	\end{enumerate}  
\end{center}

\begin{bt}
	Sử dụng phương pháp bình phương tối thiểu, hãy tìm hàm có dạng $f(x) = A \sqrt[3]{x}+ \dfrac{B}{x^2}$ (các hệ số chính xác đến 6 chữ số thập phân) để xấp xỉ tốt nhất bảng số liệu sau. 
	\begin{center}
		\begin{tabular}[5]{l|l|l|l|l|l|l|l}
			x    & 22  &  23  &  24  &  25 & 26 & 27 & 28\\ \hline
			f(x) & 1.2 & 1.5  & 1.9  & 2.1 & 2.6 & 2.8 & 3.7
		\end{tabular}	
	\end{center}
	% 
\end{bt}

\begin{bt}\label{bt2} Cho bảng số liệu sau. 
	\begin{center}
		\begin{tabular}[5]{l|l|l|l|l|l|l}
			x    & 1.0 & 1.3 & 1.6 & 1.9 & 2.2 & 2.5 \\ \hline
			y(x) & 1.032364 & 1.712278 & 2.575378 & 3.620370 & 4.846627 & 6.253805
		\end{tabular}	
	\end{center}
	% 
	Hãy sử dụng công thức nội suy Newton để (tính toán chính xác đến 6 chữ số thập phân) thực hiện những yêu cầu sau. \\
	a) Tính gần đúng đa thức nội suy $y(x)$ tại $x=1.2$ sử dụng các công thức nội suy Lagrange và Newton (được đa thức nội suy Newton tiến vì $x$ sắp xếp theo chiều tăng dần). \\
	b) Tính gần đúng đa thức nội suy $y(x)$ tại $x=1.2$ sử dụng bảng tỉ sai phân. \\
	c) Em hãy thử sắp xếp lại bảng dưới dạng chiều giảm dần của $x$, và tìm công thức đa thức nội suy Newton (được gọi là đa thức nội suy Newton lùi).	 	
\end{bt}

\begin{bt}\textbf{So sánh 2 đa thức nội suy Newton tiến và lùi} \\
	Cho hàm $\verb|f = lambda x: e^x|$. Cho bảng dữ liệu với 
	$\verb| x = np.linspace(-3,3,71)|$ và $y = f(x)$. \\
	a) Hãy thử kiểm tra sai số tương đối/tuyệt đối khi xấp xỉ $f(-3+ \sqrt{2} * 10^{-6})$, $f(3 - \sqrt{2} * 10^{-6})$ khi sử dụng hai đa thức nội suy Newton (tiến/lùi) trong 2 câu a) và c) của Bài tập \ref{bt2}. Từ đó rút ra kết luận khi nào nên sử dụng đa thức nào? \\
	b) Vậy nếu cần tính $f(1 + \sqrt{2} * 10^{-6})$ thì nên sử dụng đa thức nội suy kiểu gì? \\
	c) Viết hàm cải tiến nội suy Newton bằng cách sắp xếp lại các vector dữ liệu $x$, $y$ tùy thuộc vào giá trị của $z$. Sau đó mới tiến hành sử dụng phương pháp nội suy Newton. Giả sử các hàm $\verb|div_diff|$ và $\verb*|Newt_poly_eval|$ được cho trước.
\end{bt}

\begin{bt} 
	Xét tích phân: 
	\[ 
	I = \int_{1}^{2} \sqrt[3]{8x+3} \ dx \ .
	\]
	%
	a) Với n điểm chia ($n = 1, 2, 4$) hãy chia đoạn $[1,2]$ thành $n$ đoạn nhỏ bằng nhau. Với mỗi $n$, hãy viết công thức sử dụng công thức Simpson composite để ước lượng xấp xỉ $S_n \approx I$. \\
	b) Viết 1 đoạn code sử dụng công thức Simpson composite để tính gần đúng tích phân trên với sai số $1e-6$.
\end{bt}


\begin{bt}
\end{bt}

\begin{bt}
\end{bt}

\centerline{———————————Hết——————————-}

\end{document}

\vspace{1cm}
\noindent{\bf Chú ý:} {\it Cán bộ coi thi không giải thích gì thêm}\\
\Closesolutionfile{ans}
\newpage
\begin{center}
{\LARGE{\bf ĐÁP ÁN}}
\end{center}

\begin{sol}
	\begin{figure}[h!]
		\centering
		\includegraphics[width=0.8\linewidth]{Solution1/Sol4_1.png}
		%\caption{}
		\label{fig:Sol4}
	\end{figure}
	Exercise 7: Convergence order is 3.	
\end{sol}

   
\end{document}



