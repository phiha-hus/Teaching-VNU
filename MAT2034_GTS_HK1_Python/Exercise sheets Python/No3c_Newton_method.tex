\documentclass[11pt]{article}
\usepackage{amsmath}
%\usepackage{extsizes}
\usepackage{amsmath,amssymb}
%\usepackage{omegavn,ocmrvn}
%\usepackage[utf8x]{inputenc}
\usepackage[utf8]{vietnam}

\usepackage{longtable}
\usepackage{answers}
\usepackage{graphicx}
\usepackage{array}
\usepackage{pifont}
\usepackage{picinpar}
\usepackage{enumerate}
\usepackage[top=3.0cm, bottom=3.5cm, left=3.5cm, right=2.5cm] {geometry}
\usepackage{hyperref}


\newtheorem{bt}{Câu}
\newcommand{\RR}{\mathbb R}
\Newassociation{sol}{Solution}{ans}
\newtheorem{ex}{Câu}
\renewcommand{\solutionstyle}[1]{\textbf{ #1}.}


\begin{document}
% \noindent

\begin{tabular*}
	{\linewidth}{c>{\centering\hspace{0pt}} p{.7\textwidth}}
	Trường ĐHKHTN, ĐHQGHN & {\bf Học Kỳ 1 (2021-2022)}
	\tabularnewline
	K64 TTƯD - Thầy Hà Phi & {\bf Bài Tập Giải Tích Số. No 3c \\ Phương pháp Newton  \\ \today}
	% Exercises on pages 239, 240 Cheney/Kincaid are really nice
	\tabularnewline
	\rule{1in}{1pt}  \small  & \rule{2in}{1pt} %(Due date:)
	\tabularnewline
	
	%  \tabularnewline
	%  &(Đề thi có 1 trang)
\end{tabular*}
%
% \Opensolutionfile{ans}[ans1]

\begin{bt} % Exercise 11, Atkinson/Han p.108
Viết hàm trong Python cho phương pháp Newton (có bao hàm bước tiền xử lý để tìm điều kiện $x_0$ thích hợp) sau đó áp dụng để tìm nghiệm chính xác đến $1e-5$ cho các bài toán sau.\\
a) $e^x + 2^{-x} + 2 cos x - 6 = 0$ với $1 \leq x \leq 2$. \\
b) $\ln(x - 1) + \cos(x - 1) = 0$ với $1.3 \leq x \leq 2$. \\
c) Giải số nghiệm gần 100 nhất của phương trình $x=tan(x)$ với sai số $1e-9$.
\end{bt}

\begin{bt} % Exercises 3 & 4, Atkinson/Han p.88
a) Trong hầu hết máy tính cũ, $\sqrt{a}$ được tính dựa trên việc sử dụng phương pháp Newton để giải phương trình $x^2=a$. Hãy lập công thức lặp Newton dựa trên lý thuyết.  \\
b) Dựa vào lý thuyết được học trên lớp, hãy thiết lập các công thức truy hồi cho sai số tuyệt đối $\varepsilon_{abs}:=|x_n-\sqrt{a}|$ và tương đối $\varepsilon_{rel}:=|\cfrac{x_n-\sqrt{a}|}{\sqrt{a}}$.\\
c) Thực hiện nhiệm vụ câu a) để tìm $\sqrt[m]{a}$ với $a>0$, $m$ là 1 số nguyên dương. Viết script Python với input là $a$, $m$, output là $\sqrt[m]{a}$. Áp dụng để tính $\sqrt[8]{2}$ với 
sai số $1e-9$.
\end{bt}

\begin{bt} % Exercises 10,11,12, Atkinson/Han p.89
a) Sử dụng phương pháp Newton, hãy viết Python function để tìm nghiệm phương trình $p(x)=0$ trên 1 khoảng $[a,b]$. Input là vectơ các hệ số của $p(x)$ (theo thứ tự bậc cao đến thấp), $a$, $b$. In ra đa thức ban đầu và tất cả các nghiệm của nó. \\
b) Thử nghiệm số hàm vừa viết để giải phương trình 
%
\[ x^4 - 5.4 x^3 + 10.56 x^2 - 8.954 x + 2.7951 = 0, \]
%
trên khoảng $[-10,10]$. 
\end{bt}

\begin{bt} % Exercise 9, Atkinson/Han p.107
Hãy tìm 1 thuật toán và viết code để tìm $a$ lớn nhất với sai số $1e-6$ sao cho\\
a) $a \sqrt{x} \leq \sin(x)$ với mọi $x>0$. \\
b) $e^{ax} \leq \cfrac{1}{1+x^2}$ với mọi $x>0$.
\end{bt}

\centerline{———————————Hết——————————-}

\end{document}

\vspace{1cm}
\noindent{\bf Chú ý:} {\it Cán bộ coi thi không giải thích gì thêm}\\
\Closesolutionfile{ans}
\newpage
\begin{center}
{\LARGE{\bf ĐÁP ÁN}}
\end{center}

\begin{sol}
	\begin{figure}[h!]
		\centering
		\includegraphics[width=0.8\linewidth]{Solution1/Sol4_1.png}
		%\caption{}
		\label{fig:Sol4}
	\end{figure}
	Exercise 7: Convergence order is 3.	
\end{sol}

   
\end{document}



