\documentclass[11pt]{article}
\usepackage{amsmath}
%\usepackage{extsizes}
\usepackage{amsmath,amssymb}
%\usepackage{omegavn,ocmrvn}
%\usepackage[utf8x]{inputenc}
\usepackage[utf8]{vietnam}

\usepackage{longtable}
\usepackage{answers}

\usepackage{graphicx}
\usepackage{float}

\usepackage{listings}
\lstset{language=Python}          % Set your language (you can change the language for each code-block optionally)

\usepackage{array}
\usepackage{pifont}
\usepackage{picinpar}
\usepackage{enumerate}
\usepackage[top=3.0cm, bottom=3.5cm, left=3.5cm, right=2.5cm] {geometry}
\usepackage{hyperref}


\newtheorem{bt}{Câu}
\newcommand{\RR}{\mathbb R}
\Newassociation{sol}{Solution}{ans}
\newtheorem{ex}{Câu}
\renewcommand{\solutionstyle}[1]{\textbf{ #1}.}
\newcommand{\m}[1]{\begin{bmatrix}
		#1
\end{bmatrix}}

\begin{document}
% \noindent
\begin{tabular*}
{\linewidth}{c>{\centering\hspace{0pt}} p{.7\textwidth}}
Trường ĐHKHTN, ĐHQGHN & {\bf Học Kỳ 1 (2021-2022)}
\tabularnewline
K64 TTƯD - Thầy Hà Phi & {\bf Bài Tập Giải Tích Số. No 2b \\ Giải hệ pt tuyến tính Ax=b  \\ \today}
% Exercises on pages 239, 240 Cheney/Kincaid are really nice
\tabularnewline
\rule{1in}{1pt}  \small  & \rule{2in}{1pt} %(Due date:)
\tabularnewline

%  \tabularnewline
%  &(Đề thi có 1 trang)
\end{tabular*}
%
% \Opensolutionfile{ans}[ans1]
\vskip .2cm

Tìm hiểu toolbox linalg trong Python \url{https://docs.scipy.org/doc/numpy-1.15.1/reference/routines.linalg.html}.

\begin{bt}
Bốn vật nặng có khối lượng khác nhau $m_i$ được nối với nhau bằng những sợi dây có khối lượng không đáng kể. Ba trong số các khối nằm trên một mặt phẳng nghiêng, hệ số ma sát giữa các khối và mặt phẳng là $\mu_i$. Phương trình chuyển động của các khối có thể được biểu diễn là
%
\begin{align*}
T_1 + m_1a &= m_1g (\sin \theta - \mu_1 \cos \theta) \\
-T_1 + T_2 + m_2a &= m_2g (\sin \theta - \mu_2 \cos \theta) \\
-T_2 + T_3 + m_3a &= m_3g (\sin \theta - \mu_3 \cos \theta) \\
-T_3 + m_4a &= -m_4 g .
\end{align*}
%
Trong đó $T_i$ biểu thị lực kéo trong các sợi dây và $a$ là gia tốc của
hệ thống. Xác định $a$ và $T_i$ nếu $\theta = 45^o$, $g = 9,82 m / s^2$ và
$m = \m{10 & 4 & 5 & 6}^T$ kg, $\mu = \m{0.25 & 0.3 & 0.2}^T$.
%	
\begin{figure}[h!]
	\centering
	\includegraphics[width=0.4\linewidth]{mass_rope}
	\caption[]{Exercise 25, page 58, Kiusalass}
	\label{fig:massrope}
\end{figure}

\end{bt}

\begin{bt}
Công thức chuyển dời của hệ lò xo khối lượng được chỉ ra trong Hình (a) dẫn đến các phương trình cân bằng sau của các khối lượng
%
\[
\m{	k_1 + k_2 + k_3 + k_5 & -k_3 & -k_5 \\
	-k_3 & k_3 + k_4 & -k_4 \\
	-k_5 & -k_4 & k_4 + k_5
}
\m{	x_1 \\
	x_2 \\
	x_3
}
=
\m{ W_1 \\ W_2 \\ W_3}
\]
%
trong đó $k_i$ là độ cứng của lò xo, $W_i$ đại diện cho trọng lượng của các khối lượng và $x_i$ là độ dịch chuyển của các khối lượng từ cấu hình không định dạng của hệ thống. Viết chương trình giải các phương trình này cho k và W là các đầu vào, còn x là đầu ra. 
Sử dụng chương trình để tìm các chuyển vị nếu
\begin{align*}
k_1 &= k_3 = k_4 = k, \ k_2 = k_5 = 2k, \\
W_1 &= W_3 = 2W, \ W_2 = W \ .
\end{align*}

\begin{figure}[h!]
	\centering
	\includegraphics[width=0.3\linewidth]{mass_spring}
	\caption{Exercise 12, page 79, Kiusalass}
	\label{fig:massspring}
\end{figure}
\end{bt}

\begin{bt}
Công thức chuyển dời của giàn phẳng tương tự như công thức của hệ lò xo khối lượng. Sự khác biệt là (1) độ cứng của các bộ phận là $k_i = (EA / L)_i$, trong đó E là mô đun đàn hồi, A đại diện cho diện tích mặt cắt ngang, và L là chiều dài của bộ phận; và (2) có hai thành phần chuyển vị tại mỗi khớp. Đối với giàn không xác định tĩnh được hiển thị, công thức chuyển vị cho ra phương trình đối xứng $Ku = p$, trong đó
\begin{align*}
K &=
\m{	27.58 & 7.004 &-7.004 &0 &0 \\
	7.004 &29.57& -5.253& 0 &-24.32 \\
	-7.004 &-5.253 &29.57& 0& 0 \\
	0& 0& 0& 27.58& -7.004 \\
	0& -24.32& 0 &-7.004& 29.57} \quad MN/m \ (\mbox{millinewton/metre}) \\
p &= \m{0 &0& 0& 0& -45}^T \quad kN \ (\mbox{kilonewton}) \ .
\end{align*}
Xác định các vị trí $u_i$ của các khớp.

\begin{figure}[h!]
	\centering
	\includegraphics[width=0.4\linewidth]{plane_truss}
	\caption{Exercise 14, page 79, Kiusalass}
	\label{fig:planetruss}
\end{figure}
\end{bt}

\newpage 

\begin{bt}
Bốn thùng trộn được nối với nhau bằng đường ống. Chất lỏng trong hệ thống được bơm thông qua các đường ống với tỷ lệ hiển thị trong hình. Chất lỏng vào hệ thống
chứa hóa chất có nồng độ c theo chỉ định. Xác định nồng độ của hóa chất trong bốn bình, giả sử ở trạng thái dừng.
\begin{figure}[h!]
	\centering
	\includegraphics[width=0.5\linewidth]{mixing_tank}
	\caption{Exercise 15, page 79, Kiusalass}
	\label{fig:mixingtank}
\end{figure}
\end{bt}

\begin{bt} (BT 4.40/p163/Amos Gilat) \\
a) Hãy xây dựng hệ phương trình tuyến tính của các biến số $a$, $b$, $c$, và $d$ bằng việc cân bằng phản ứng hóa học dưới đây
%
\[
P_2I_4 + aP_4+ b H_2 0 \rightleftarrows c PH_4I + d H_3 P 0_4
\]
%
b) Giải hệ phương trình nói trên bằng phương pháp LU và PLU. \\
c) Có thể áp dụng phương pháp Cholesky để giải hệ nói trên hay không? Vì sao?
\end{bt}

\begin{bt}
Nhiều bài tập hay khác còn có thể tìm được trong Chương 4 của cuốn Amos Gilat.
\end{bt}

\centerline{———————————Hết——————————-}

\end{document}

\vspace{1cm}
\noindent{\bf Chú ý:} {\it Cán bộ coi thi không giải thích gì thêm}\\
\Closesolutionfile{ans}
\newpage
\begin{center}
{\LARGE{\bf ĐÁP ÁN}}
\end{center}

\begin{sol}
	\begin{figure}[h!]
		\centering
		\includegraphics[width=0.8\linewidth]{Solution1/Sol4_1.png}
		%\caption{}
		\label{fig:Sol4}
	\end{figure}
	Exercise 7: Convergence order is 3.	
\end{sol}

   
\end{document}



