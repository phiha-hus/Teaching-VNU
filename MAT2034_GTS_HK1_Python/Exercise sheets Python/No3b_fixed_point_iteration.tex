\documentclass[11pt]{article}
\usepackage{amsmath}
%\usepackage{extsizes}
\usepackage{amsmath,amssymb}
%\usepackage{omegavn,ocmrvn}
%\usepackage[utf8x]{inputenc}
\usepackage[utf8]{vietnam}

\usepackage{longtable}
\usepackage{answers}
\usepackage{graphicx}
\usepackage{array}
\usepackage{pifont}
\usepackage{picinpar}
\usepackage{enumerate}
\usepackage[top=3.0cm, bottom=3.5cm, left=3.5cm, right=2.5cm] {geometry}
\usepackage{hyperref}


\newtheorem{bt}{Câu}
\newcommand{\RR}{\mathbb R}
\Newassociation{sol}{Solution}{ans}
\newtheorem{ex}{Câu}
\renewcommand{\solutionstyle}[1]{\textbf{ #1}.}


\begin{document}
% \noindent
\begin{tabular*}
	{\linewidth}{c>{\centering\hspace{0pt}} p{.7\textwidth}}
	Trường ĐHKHTN, ĐHQGHN & {\bf Học Kỳ 1 (2021-2022)}
	\tabularnewline
	K64 TTƯD - Thầy Hà Phi & {\bf Bài Tập Giải Tích Số. No 3b \\ Phương pháp lặp đơn \\ \today}
	% Exercises on pages 239, 240 Cheney/Kincaid are really nice
	\tabularnewline
	\rule{1in}{1pt}  \small  & \rule{2in}{1pt} %(Due date:)
	\tabularnewline
	
	%  \tabularnewline
	%  &(Đề thi có 1 trang)
\end{tabular*}
%
% \Opensolutionfile{ans}[ans1]

\begin{bt} % Exercises 3 & 4, Atkinson/Han p.106
a) Phương trình sau có bao nhiêu nghiệm $x=e^{-x}$? \\
b) Chứng minh hàm số $g(x) = e^{-x}$ là 1 tự ánh trên đoạn $[0.1,1]$. \\
c) Phép lặp đơn $x_{n+1}=e^{-x_n}$ có hội tụ với giá trị $x_0$ phù hợp hay không? \\
d) Câu hỏi tương tự với phép lặp $x_{n+1} = 1 + \arctan(x_n)$.
\end{bt}

\begin{bt} % Exercise 5, Atkinson/Han p.107
Chứng minh rằng với các hằng số c, d thỏa mãn $|d|<1$, phương trình $x=c+d \cos(x)$ có nghiệm duy nhất. Kiểm tra tính hội tụ của phép lặp $x_{n+1}=c+d \cos(x_n)$ và hãy đưa ra đánh giá cho tốc độ hội tụ.
\end{bt}

\begin{bt} % Exercise 8, Atkinson/Han p.107
Các phép lặp sau có hội tụ đến $x^*$ hay không? Nếu hội tụ, hãy xác định tốc độ hội tụ, cho $x_0$ đủ gần $x^*$.
%
\begin{align*}
a) \  x_{n+1} &= \cfrac{15 x_n^2-24x_n+13}{4x_n} \ , \ x^*=1, \\
b) \  x_{n+1} &= \cfrac{3}{4} x_n + \cfrac{1}{x_n^3} \ , \ x^*=\sqrt{2}.
\end{align*}
%
Tìm số bước lặp cần thiết để nhận được xấp xỉ với sai số tuyệt đối không quá $1e-6$, với $x_0 = x^* + 0.1$. 
\end{bt}

\begin{bt} % Exercise 9, Atkinson/Han p.107
Giả sử bài toán tìm nghiệm $f(x)=0$ có nghiệm $x^*$ thỏa mãn $f'(x^*)\not=0$. Ta có thể chuyển nó về bài toán tìm điểm bất động $x$ của hàm số $g(x)=x+cf(x)$ với hằng số $c$. Phải chọn $c$ thế nào để đảm bảo sự hội tụ nhanh của phép lặp đơn $x_{n+1}=g(x_n)$, giả sử rằng $x_0$ đủ gần $x^*$? Kiểm nghiệm kết quả tìm được cho bài toán $x^3-5=0$.
\end{bt}

\begin{bt} % Exercise 11, Atkinson/Han p.108
Phép lặp đơn $x_{n+1}=2-(1+c)x_n+cx_n^3$ sẽ hội tụ đến $x^*=1$ với một số giá trị của c, giả sử $x_0$ đủ gần $x^*$. \\
a) Tìm tất cả mọi $c$ để phép lặp đơn này hội tụ. Tìm mọi $c$ để phép lặp đơn này hội tụ bậc hai. \\
b) Với một $c$ như vậy, hãy tính số bước lặp cần thiết để đạt được 10 chữ số chắc, cho điều kiện ban đầu $x_0=x^* + 0.1$.  
\end{bt}

\begin{bt} 
Phương trình $x^3+4x^2-10=0$ có nghiệm duy nhất trong đoạn $[1, 2]$. Có rất nhiều các khác nhau để chuyển về bài toán tìm điểm bất động. Hãy xét sự hội tụ của các phép lặp đơn sau, với điều kiện đầu $x_0=1.5$. Tìm bậc hội tụ của các phương pháp đó (nếu có) và tính sai số với $n=1,...,10$, từ đó so sánh với phương pháp phân đôi.\\
%
\begin{tabular}{lll}
a) $x=g_1(x)=x+x^3+4x^2-10$	&  & b) $x=g_2(x) = \sqrt{10/x - 4x}$ \\ 
c) $x=g_3(x)=\cfrac{1}{2}\sqrt{10-x^3}$	&  &  d) $x=g_2(x) = \sqrt{\cfrac{10}{x+4}}$ \\ 
e) $x=g_3(x)=x-\cfrac{x^3+4x^2-10}{3x^2+8x}$	&  &  \\ 
\end{tabular} 
%	
\end{bt}

\begin{bt} % Exercise 18, Atkinson/Han p.109
Cho tham số thực $a>0$. Tìm bậc hội tụ của phương pháp lặp sau  $x_{n+1}=\cfrac{x_n(x_n^2+3a)}{3x_n^2+a}$ \ trong trường hợp nó hội tụ đến điểm bất động $x^* = \sqrt{a}$.
\end{bt}

\begin{bt}
Cho các phương trình sau
%
\[ a) \ 3(2x-1)= \cos(x) \hskip 6cm b) \ x^4-2x-3=0  \]
%
Hãy xây dựng cho mỗi phương trình một phương pháp lặp đơn hội tụ, biết rằng phương trình a) (t.ứ. b)) có nghiệm duy nhất trong $(0,1)$ (t.ứ. $(0,2)$). Viết các công thức đánh giá sai số tiên nghiệm, hậu nghiệm sao cho sai số nhỏ hơn $1e-6$.
\end{bt}

\centerline{———————————Hết——————————-}

\end{document}

\vspace{1cm}
\noindent{\bf Chú ý:} {\it Cán bộ coi thi không giải thích gì thêm}\\
\Closesolutionfile{ans}
\newpage
\begin{center}
{\LARGE{\bf ĐÁP ÁN}}
\end{center}

\begin{sol}
	\begin{figure}[h!]
		\centering
		\includegraphics[width=0.8\linewidth]{Solution1/Sol4_1.png}
		%\caption{}
		\label{fig:Sol4}
	\end{figure}
	Exercise 7: Convergence order is 3.	
\end{sol}

   
\end{document}



