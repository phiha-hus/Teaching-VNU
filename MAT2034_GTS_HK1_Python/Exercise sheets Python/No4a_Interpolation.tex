\documentclass[11pt]{article}
\usepackage{amsmath}
%\usepackage{extsizes}
\usepackage{amsmath,amssymb}
%\usepackage{omegavn,ocmrvn}
%\usepackage[utf8x]{inputenc}
\usepackage[utf8]{vietnam}

\usepackage{longtable}
\usepackage{answers}
\usepackage{graphicx}
\usepackage{array}
\usepackage{pifont}
\usepackage{picinpar}
\usepackage{enumerate}
\usepackage[top=3.0cm, bottom=3.5cm, left=3.5cm, right=2.5cm] {geometry}
\usepackage{hyperref}


\newtheorem{bt}{Câu}
\newcommand{\RR}{\mathbb R}
\Newassociation{sol}{Solution}{ans}
\newtheorem{ex}{Câu}
\renewcommand{\solutionstyle}[1]{\textbf{ #1}.}


\begin{document}
	% \noindent

\begin{tabular*}
	{\linewidth}{c>{\centering\hspace{0pt}} p{.7\textwidth}}
	Trường ĐHKHTN, ĐHQGHN & {\bf Học Kỳ 1 (2022-2023)}
	\tabularnewline
	K65 MT\&KHTT - Thầy Hà Phi & {\bf Bài Tập Giải Tích Số \\ \today}
	% Exercises on pages 239, 240 Cheney/Kincaid are really nice
	\tabularnewline
	\rule{1in}{1pt}  \small  & \rule{2in}{1pt} %(Due date:)
	\tabularnewline
	%  \tabularnewline
	%  &(Đề thi có 1 trang)
\end{tabular*}




\begin{center}
	{\bf Bài Tập Giải Tích Số. No 4a \\ Các phép nội suy Lagrange và Newton. Bảng tỷ sai phân.}
\end{center}
%
% \Opensolutionfile{ans}[ans1]

\begin{figure}[h!]
	\centering
	\includegraphics[scale = 0.5]{"Figures/Exe_sheet_No6_1"}
\end{figure}

\begin{figure}[h!]
	\centering
	\includegraphics[scale = 0.5]{"Figures/Exe_sheet_No6_2"}
\end{figure}

\begin{figure}[ht]
	\centering
	\includegraphics[scale = 0.5]{"Figures/Exe_sheet_No6_3"}
	\caption{Bài tập sưu tầm từ webpage của PGS Vũ Hoàng Linh}
\end{figure}

\end{document}

\begin{bt}
	a) Hãy tìm đa thức nội suy bậc 3 của hàm số $f(x) = x^4$ tại 4 mốc nội suy $\pm 1$, $0$, $2$.  
	b) Đưa ra ước lượng sai số tuyệt đối và so sánh với công thức được học trong bài.
\end{bt}

\vspace{1cm}
\noindent{\bf Chú ý:} {\it Cán bộ coi thi không giải thích gì thêm}\\
\Closesolutionfile{ans}
\newpage
\begin{center}
{\LARGE{\bf ĐÁP ÁN}}
\end{center}

\begin{sol}
	\begin{figure}[h!]
		\centering
		\includegraphics[width=0.8\linewidth]{Solution1/Sol4_1.png}
		%\caption{}
		\label{fig:Sol4}
	\end{figure}
	Exercise 7: Convergence order is 3.	
\end{sol}

   
\end{document}



