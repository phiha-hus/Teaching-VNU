\documentclass[11pt]{article}
\usepackage{amsmath}
%\usepackage{extsizes}
\usepackage{amsmath,amssymb}
%\usepackage{omegavn,ocmrvn}
%\usepackage[utf8x]{inputenc}
\usepackage[utf8]{vietnam}

\usepackage{longtable}
\usepackage{answers}

\usepackage{graphicx}
\usepackage{float}

\usepackage{listings}
\lstset{language=Python}          % Set your language (you can change the language for each code-block optionally)

\usepackage{array}
\usepackage{pifont}
\usepackage{picinpar}
\usepackage{enumerate}
\usepackage[top=3.0cm, bottom=3.5cm, left=3.5cm, right=2.5cm] {geometry}
\usepackage{hyperref}


\newtheorem{bt}{Câu}
\newcommand{\RR}{\mathbb R}
\Newassociation{sol}{Solution}{ans}
\newtheorem{ex}{Câu}
\renewcommand{\solutionstyle}[1]{\textbf{ #1}.}
\newcommand{\m}[1]{\begin{bmatrix}
		#1
\end{bmatrix}}

\begin{document}
% \noindent
\begin{tabular*}
{\linewidth}{c>{\centering\hspace{0pt}} p{.7\textwidth}}
Trường ĐHKHTN, ĐHQGHN & {\bf Học Kỳ 1 (2021-2022)}
\tabularnewline
K64 TTƯD - Thầy Hà Phi & {\bf Bài Tập Giải Tích Số. No 2a \\ Giải hệ pt tuyến tính Ax=b \\ \today}
% Exercises on pages 239, 240 Cheney/Kincaid are really nice
\tabularnewline
\rule{1in}{1pt}  \small  & \rule{2in}{1pt} %(Due date:)
\tabularnewline

%  \tabularnewline
%  &(Đề thi có 1 trang)
\end{tabular*}
%
% \Opensolutionfile{ans}[ans1]
\vskip .2cm

Tìm hiểu toolbox linalg trong Python \url{https://docs.scipy.org/doc/numpy-1.15.1/reference/routines.linalg.html}.

\begin{bt}
a) Với tham số $t$, sử dụng toolbox linalg hãy đi tìm chuẩn 1, 2, $\infty$ của ma trân $A= \begin{bmatrix} 1 & -t \\ 2t & 1+t^2 \end{bmatrix}$ với $t=10,100,200,...,1000$. Tìm số điều kiện của các ma trận A đó.\\
b) Tìm chuẩn và số điều kiện của A trong trường hợp tổng quát (t bất kỳ) bằng việc tính toán lý thuyết. 

\end{bt}

\begin{bt}
Ma trận Hermit được định nghĩa bởi $H_n = \Big[ \dfrac{1}{i+j+1} \Big]_{i,j=0}^n$. Hãy đi tìm số điều kiện của $H_5$, $H_{12}$ theo các chuẩn 1, 2, $\infty$. 
\end{bt}

\begin{bt}  
a) \textbf{Lý thuyết:} Sử dụng phương pháp  khử Gauss trong lý thuyết để tìm phân tích $LU$ và giải hệ phương trình sau
%
\begin{align*}
 2 x_1 + 4 x_2 + 3x_3   &= 3, \\
 3 x_1 + x_2 - 2 x_3    &= 3, \\
 4 x_1 + 11 x_2 + 7 x_3 &= 4. 
\end{align*}
%
b) \textbf{Thực hành:} Viết hàm trong Python để tìm phân tích $A=LU$ và giải hệ phương trình $Ax=b$ ở trên. So sánh kết quả của các em với cách giải sử dụng toolbox linalg trong Python. 
\end{bt}

\begin{bt} 
a) \textbf{Lý thuyết:} Sử dụng phương pháp khử Gauss kết hợp với Pivoting trong lý thuyết để tìm phân tích $PLU$ và giải hệ phương trình sau
%
\begin{align*}
	3 x_1 + 17 x_2   + 10 x_3   &= 30, \\
	2 x_1 + 4 x_2 - 2 x_3    &= 4, \\
	6 x_1 + 18 x_2 -12 x_3 &= 12. 
\end{align*}
%
b) \textbf{Thực hành:} Viết hàm trong Python để tìm phân tích $PLU$ của ma trận $A$ và giải hệ phương trình $Ax=b$ ở trên. 
So sánh kết quả của các em với cách giải sử dụng toolbox linalg trong Python. 	
\end{bt}

\begin{bt}
a) Tìm hiểu các ma trận Pascal, Leslie và Van der Monde trong module scipy.linalg, ứng với các ma trận cỡ $3\times 3$ hoặc $4 \times 4$. \\
b) Trong các ma trận cùng cỡ $3\times 3$ hoặc $4 \times 4$ ở trên, ma trận nào có số điều kiện lớn nhất (vì sao)? Chú ý kết quả phụ thuộc vào cách chọn ma trận của các em (ngẫu nhiên). \\
c) Tìm các phân tích LU và PLU của chúng bằng tính toán lý thuyết. 
\end{bt}

\centering{------------------------------Kết thúc phần 1--------------------------------------}

\newpage 

\begin{bt} Tìm các phân tích PLU của các ma trận $A$ dưới đây và giải hệ phương trình vi phân tương ứng, với $b$ là random vector, làm tròn 4 chữ số thập phân khi tính toán bằng tay. \\
i) $A = \m{4 & 5 & -2 \\ 2 & -5 & 2 \\ 6 & 2 & 4}$ \qquad 
ii) $A = \m{2 & 4 & 8 \\ 10 & 12 & 0 \\ 0 & 2 & 8 }$, \\
iii) $A = \m{0 & 1 & 0 & 0  \\ 2 & 0 & 1 & 0  \\ 0 & 2 & 0 & 1 \\ 0 & 0 & 2 & 0}$ \qquad iv) $A = \m{2 & 2 & 0 & 0  \\ 2 & 2 & 1 & 0  \\ 1 & 1 & 0 & 2 \\ 0 & 1 & 0 & 0}$.	
\end{bt}

\begin{bt}
Trong trường hợp ma trận A là đối xứng, xác định dương thì phương pháp Cholesky thường được sử dụng. Hãy đọc phương pháp này trang 116-120 (Giáo trình ĐHBK) hoặc Section 2.4 (Giáo trình Kiusalass) và tìm hiểu hàm $numpy.linalg.cholesky$ trong Python. Áp dụng để giải hệ phương trình sau đối với vế phải $b$ lần lượt bằng $[2 \ 3 \ 0]^T$ và $[2 \ 5 \ -2]^T$.
	%
	\begin{align*}
	4 x_1 - 2 x_2 + 4 x_3   &= b_1, \\
	-2 x_1 + 5 x_2 - 4 x_3  &= b_2, \\
	4 x_1  -4 x_2 + 6 x_3   &= b_3. 
	\end{align*}
	%
\end{bt}

\begin{bt}
Cho hệ phương trình sau, trong đó $\Delta b$ là các sai số tuyệt đối do quá trình đo đạc, còn $\Delta x$ là sai số tuyệt đối của nghiệm.
%
\[
\m{1 & 0 \\ 100 & 1} (x+\Delta x) = b+\Delta b.
\]
%
a) Tính số điều kiện của ma trận $A$ theo chuẩn $\infty$. \\
b) Xét theo chuẩn nói trên, nếu sai số tương đối cho phép của nghiệm $x$ là $1\%$, thì sai số tương đối tối đa của vế phải là bao nhiêu $\%$?
\end{bt}

\begin{bt}
Bạn Hải Anh sau khi học lớp thầy Phi đưa ra kết luận ``Nếu số điều kiện của 1 ma trận vuông A là xấu (tức là rất lớn), thì định thức của ma trận A phải rất gần 0 (tức là có giá trị tuyệt đối rất nhỏ)." 
Hãy khẳng định hoặc phủ định khẳng định trên của Hải Anh, xét trong 3 trường hợp chuẩn $1$, $\infty$ và $2$. \\
Hint: For the 2-norm, one can use SVD, noticing that the 2-norm is preserved under orthogonal transformations.
\end{bt}

\centering{------------------------------ Kết thúc phần 2 --------------------------------------}



% \vskip .5cm

% \centerline{———————————Hết——————————}

\end{document}

\vspace{1cm}
\noindent{\bf Chú ý:} {\it Cán bộ coi thi không giải thích gì thêm}\\
\Closesolutionfile{ans}
\newpage
\begin{center}
{\LARGE{\bf ĐÁP ÁN}}
\end{center}

\begin{sol}
	\begin{figure}[h!]
		\centering
		\includegraphics[width=0.8\linewidth]{Solution1/Sol4_1.png}
		%\caption{}
		\label{fig:Sol4}
	\end{figure}
	Exercise 7: Convergence order is 3.	
\end{sol}

   
\end{document}



