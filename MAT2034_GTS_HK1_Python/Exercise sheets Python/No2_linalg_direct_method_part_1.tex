\documentclass[11pt]{article}
\usepackage{amsmath}
%\usepackage{extsizes}
\usepackage{amsmath,amssymb}
%\usepackage{omegavn,ocmrvn}
%\usepackage[utf8x]{inputenc}
\usepackage[utf8]{vietnam}

\usepackage{longtable}
\usepackage{answers}

\usepackage{graphicx}
\usepackage{float}

\usepackage{listings}
\lstset{language=Python}          % Set your language (you can change the language for each code-block optionally)

\usepackage{array}
\usepackage{pifont}
\usepackage{picinpar}
\usepackage{enumerate}
\usepackage[top=3.0cm, bottom=3.5cm, left=3.5cm, right=2.5cm] {geometry}
\usepackage{hyperref}


\newtheorem{bt}{Câu}
\newcommand{\RR}{\mathbb R}
\Newassociation{sol}{Solution}{ans}
\newtheorem{ex}{Câu}
\renewcommand{\solutionstyle}[1]{\textbf{ #1}.}
\newcommand{\m}[1]{\begin{bmatrix}
		#1
\end{bmatrix}}

\begin{document}
% \noindent
\begin{tabular*}
{\linewidth}{c>{\centering\hspace{0pt}} p{.7\textwidth}}
Trường ĐHKHTN, ĐHQGHN & {\bf Học Kỳ 1 (2019-2020)}
\tabularnewline
K63 TTƯD - Thầy Hà Phi & {\bf Bài Tập Giải Tích Số. No 6 \\ Giải hệ pt tuyến tính Ax=b}
% Exercises on pages 239, 240 Cheney/Kincaid are really nice
\tabularnewline
\rule{1in}{1pt}  \small  & \rule{2in}{1pt} %(Due date:)
\tabularnewline

%  \tabularnewline
%  &(Đề thi có 1 trang)
\end{tabular*}
%
% \Opensolutionfile{ans}[ans1]
\vskip .2cm

Tìm hiểu toolbox linalg trong Python \url{https://docs.scipy.org/doc/numpy-1.15.1/reference/routines.linalg.html}.

\begin{bt}
Với tham số $t$, sử dụng toolbox linalg hãy đi tìm chuẩn 1, 2, $\infty$ của ma trân $A= \begin{bmatrix} 1 & t \\ t & 1+t^2 \end{bmatrix}$ với $t=10,100,200,...,1000$. Tìm số điều kiện của các ma trận A đó.
\end{bt}

\begin{bt}
Ma trận Hermit được định nghĩa bởi $H_n = \Big[ \dfrac{1}{i+j-1} \Big]_{i,j=1}^n$. Hãy đi tìm số điều kiện của $H_5$, $H_{12}$ theo các chuẩn 1, 2, $\infty$. 
\end{bt}

\begin{bt}  
a) \textbf{Lý thuyết:} Sử dụng phương pháp  khử Gauss trong lý thuyết để tìm phân tích $LU$ và giải hệ phương trình sau
%
\begin{align*}
 2 x_1 + 4 x_2 + 3x_3   &= 3, \\
 3 x_1 + x_2 - 2 x_3    &= 3, \\
 4 x_1 + 11 x_2 + 7 x_3 &= 4. 
\end{align*}
%
b) \textbf{Thực hành:} Viết hàm trong Python để tìm phân tích $A=LU$ và giải hệ phương trình $Ax=b$ ở trên. So sánh kết quả của các em với cách giải sử dụng toolbox linalg trong Python. 
\end{bt}

\begin{bt} 
a) \textbf{Lý thuyết:} Sử dụng phương pháp khử Gauss kết hợp với Pivoting trong lý thuyết để tìm phân tích $PLU$ và giải hệ phương trình sau
%
\begin{align*}
	-4 x_1 + 1 x_2   -1 x_3   &= -9, \\
	-1 x_1 + 4 x_2 - 1 x_3    &= -12, \\
	1 x_1 + 1 x_2 + 4 x_3 &= 11. 
\end{align*}
%
b) \textbf{Thực hành:} Viết hàm trong Python để tìm phân tích $PLU$ của ma trận $A$ và giải hệ phương trình $Ax=b$ ở trên. 
So sánh kết quả của các em với cách giải sử dụng toolbox linalg trong Python. 	
\end{bt}

\begin{bt}
	Tìm hiểu các ma trận Pascal, Leslie và Van der Monde trong module scipy.linalg. Tìm các phân tích LU và PLU của chúng, ứng với các ma trận cỡ $3\times 3$ hoặc $4 \times 4$.
\end{bt}

\centering{------------------------------Kết thúc phần 1--------------------------------------}

\newpage 

\begin{bt} Tìm các phân tích PLU của các ma trận $A$ dưới đây và giải hệ phương trình vi phân tương ứng, với $b$ là random vector, làm tròn 4 chữ số thập phân khi tính toán bằng tay. \\
i) $A = \m{4 & -1 & -2 \\ -1 & 4 & -1 \\ -1 & -2 & 4}$ \qquad ii) $A = \m{-3 & 1 & -1 \\ -1 & 4 & 1 \\ 2 & -1 & 6}$, \\
iii) $A = \m{0 & 1 & 0 & 0  \\ 2 & 0 & 1 & 0  \\ 0 & 2 & 0 & 1 \\ 0 & 0 & 2 & 0}$ \qquad iv) $A = \m{2 & 2 & 0 & 0  \\ 2 & 2 & 1 & 0  \\ 1 & 1 & 0 & 2 \\ 0 & 1 & 0 & 0}$.	
\end{bt}

\begin{bt}
Trong trường hợp ma trận A là đối xứng, xác định dương thì phương pháp Cholesky thường được sử dụng. Hãy đọc phương pháp này trang 116-120 (Giáo trình ĐHBK) hoặc Section 2.4 (Giáo trình Kiusalass) và tìm hiểu hàm $numpy.linalg.cholesky$ trong Python. Áp dụng để giải hệ phương trình sau đối với vế phải $b$ lần lượt bằng $[2 \ 3 \ 0]^T$ và $[2 \ 5 \ -2]^T$.
	%
	\begin{align*}
	4 x_1 - 2 x_2 + 4 x_3   &= b_1, \\
	-2 x_1 + 5 x_2 - 4 x_3  &= b_2, \\
	4 x_1  -4 x_2 + 6 x_3   &= b_3. 
	\end{align*}
	%
\end{bt}

\centering{------------------------------ Kết thúc phần 2 --------------------------------------}



% \vskip .5cm

% \centerline{———————————Hết——————————}

\end{document}

\vspace{1cm}
\noindent{\bf Chú ý:} {\it Cán bộ coi thi không giải thích gì thêm}\\
\Closesolutionfile{ans}
\newpage
\begin{center}
{\LARGE{\bf ĐÁP ÁN}}
\end{center}

\begin{sol}
	\begin{figure}[h!]
		\centering
		\includegraphics[width=0.8\linewidth]{Solution1/Sol4_1.png}
		%\caption{}
		\label{fig:Sol4}
	\end{figure}
	Exercise 7: Convergence order is 3.	
\end{sol}

   
\end{document}



