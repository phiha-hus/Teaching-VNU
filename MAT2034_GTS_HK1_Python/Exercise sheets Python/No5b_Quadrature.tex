\documentclass[11pt]{article}
\usepackage{amsmath}
%\usepackage{extsizes}
\usepackage{amsmath,amssymb}
%\usepackage{omegavn,ocmrvn}
%\usepackage[utf8x]{inputenc}
\usepackage[utf8]{vietnam}

\usepackage{listings}
\lstset{language=Python}          % Set your language (you can change the language for each code-block optionally)


\usepackage{longtable}
\usepackage{answers}
\usepackage{graphicx}
\usepackage{array}
\usepackage{pifont}
\usepackage{picinpar}
\usepackage{enumerate}
\usepackage[top=3.0cm, bottom=3.5cm, left=3.5cm, right=2.5cm] {geometry}
\usepackage{hyperref}


\newtheorem{bt}{Câu}
\newcommand{\RR}{\mathbb R}
\Newassociation{sol}{Solution}{ans}
\newtheorem{ex}{Câu}
\renewcommand{\solutionstyle}[1]{\textbf{ #1}.}


\begin{document}
% \noindent

\begin{tabular*}
	{\linewidth}{c>{\centering\hspace{0pt}} p{.7\textwidth}}
	Trường ĐHKHTN, ĐHQGHN & {\bf Học Kỳ 1 (2022-2023)}
	\tabularnewline
	K65 MT\&KHTT - Thầy Hà Phi & {\bf Bài Tập Giải Tích Số \\ \today}
	% Exercises on pages 239, 240 Cheney/Kincaid are really nice
	\tabularnewline
	\rule{1in}{1pt}  \small  & \rule{2in}{1pt} %(Due date:)
	\tabularnewline
	%  \tabularnewline
	%  &(Đề thi có 1 trang)
\end{tabular*}




\begin{center}
	{\bf Bài Tập Giải Tích Số. No 5b: Tính gần đúng tích phân \\ Các phương pháp Newton-Cotes \& Gauss}
\end{center}

\begin{center}
	BÀI TẬP LÝ THUYẾT \\
%	\vskip .5cm 
\end{center}

\begin{bt}\label{bt1}
a) Hãy tính các tích phân sau sử dụng cả 2 phương pháp \emph{Hình thang/Trung điểm} đến 4 chữ số thập phân và tinh sai số tương đối, biết kết quả chính xác là $0.7468$. 
\[ \int_{0}^{1} e^{-x^2} dx \ . \]
%	
b) Nếu cần tính tích phân nói trên với sai số nhỏ hơn $1e-9$ bằng các phương pháp \emph{Hình thang/Trung điểm \ composite} thì cần sử dụng tối thiểu khoảng bao nhiêu điểm nút?
\end{bt}

\begin{bt}
Tích phân logarit là một dạng tích phân phụ thuộc tham số đặc biệt có dạng 
%
\[ \rm{li}(x) = \int_{2}^{x} \cfrac{dt}{\ln t} dt  \ . \]
%
Với $x$ đủ lớn, số lượng các số nguyên tố nhỏ hơn hoặc bằng $x$ là xấp xỉ gần bằng $\rm{li}(x)$. Ví dụ, có 46 số nguyên tố nhỏ hơn hoặc bằng 200, và li(200) thì xấp xỉ 50. \\
a) Hỏi có bao nhiêu số nguyên tố nhỏ hơn hoặc bằng 300 nếu sử dụng công thức trên. \\
b) Sử dụng pp Simpson 1/3 composite để tính toán tích phân trên với độ chính xác $1e-6$ thì cần dùng bao nhiêu điểm nút? Bước $h$ của phương pháp $=?$ \\
c) Hãy lập trình phương pháp Simpson 1/3 và 3/8 để tìm $\rm{li}(n)$ với $n=200,\ 300, ...,\ 1000$. So sánh kết quả thực tế và xấp xỉ theo bảng sau.
%
\begin{center}
\begin{tabular}{|c|c|c|c|c|c|}
	\hline
	n &  200 & 300 & \dots  & \dots & 1000  \\
	\hline
	Simpson 1/3 &  &  &  &  &  \\
	\hline
	Simpson 3/8 &  &  &  &  &  \\
	\hline
    Kết quả thực tế &  &  &  &  &  \\
    \hline
    Sai số &  &  &  &  &  \\
    \hline		
\end{tabular}
\end{center}
%
% Gợi ý: Như vậy tìm $h$ để sai số toàn phần bé hơn 1e-3, sau đó dùng quy tắc cầu phương cho các điểm nút cách đều h.
\end{bt}


\begin{bt} % Exercise 5, 6 Burden Faires, p. 226
	Hãy xác định các hằng số $a$, $b$, $c$, và $d$ sao cho quy tắc cầu phương sau có cấp chính xác là 3.
	%
	\[ \int_{-1}^{1} f(x) dx = a f(-1) + bf (1) + cf'(-1) + df'(1) \ .
	\]
	%
\end{bt}

\begin{bt}
	Hãy xác định các hằng số $a$, $b$, $c$, $d$, $e$ sao cho quy tắc cầu phương sau có cấp chính xác là 4.
	%
	\[  \int_{-1}^{1} f(x) dx = a f(-1) + bf (1) + cf(0) + d f'(-1) + e f'(1) \ .  \]
	%
\end{bt}

\begin{bt}
	Hãy tìm 4 hằng số $A$, $B$, $C$, $D$ sao cho quy tắc cầu phương sau có cấp chính xác lớn nhất có thể. 
	%
	\[  Af(-h) + B f(0) + C f(h) = hDf'(h) + \int_{-h}^{h} f(x) dx \ .
	\]
	%

\begin{center}
	==============BÀI TẬP LẬP TRÌNH============== \\
	%	\vskip .5cm 
\end{center}
\end{bt}

\begin{bt} a) Viết hàm trong Python để tính tích phân dựa trên các quy tắc cầu phương Simpson 1/3 và 3/8 dạng composite, ví dụ như
	%
	\begin{lstlisting}[frame=single] 
		def Simpson_3_8(f,a,b,tol):
		return I, h
	\end{lstlisting}
	%	 
trong đó $f$ là hàm; $a$ và $b$ là 2 đầu mút, $tol$ là sai số cần đạt được của việc tính xấp xỉ tích phân, $I$ là giá trị gần đúng của tích phân, $h$ là độ rộng mỗi bước. \\
b) Test hàm đã cho và so sánh sai số của hàm đó với hàm \verb|simpson| trong module \verb|scipy.integrate|. Chú ý tìm hiểu hàm \verb|simpson| cẩn thận.
\end{bt}

\begin{bt}\label{bt4} 
a) Đề xuất phương án và thực hiện việc tính xấp xỉ các tích phân suy rộng sau với sai số $tol = 1e-9$, sử dụng các hàm đã viết trong bài tập trước. \\
i) $A = \int_{1}^{\infty} \dfrac{1}{x^2} dx$ \ (=1) \hskip 2cm  ii) $B = \int_{0}^{\infty} \dfrac{t}{e^{2t}} dx$ \ (=1/4) . \\
b) So sánh sai số của kết quả tìm được với sai số của hàm \verb|quad| trong module \verb|scipy.integrate|.
\end{bt}

\begin{bt}
Tìm hiểu các hàm \verb|newton_cotes|, \verb|quadrature|, \verb|fixed_quad| trong module \verb|scipy.integrate| để trả lời câu hỏi \textbf{\emph{"Hàm đấy dùng để làm gì, hoạt động như thế nào và dựa trên thuật toán nào."}} Ứng dụng các quy tắc đó để tính tích phân trong các Bài Tập \ref{bt1} và \ref{bt4}. \\

%
%\hline
\end{bt}

\begin{center}
	==============BÀI TẬP THÊM (Tự luyện tập)============== \\
	%	\vskip .5cm 
\end{center}

\begin{bt}
Xét tích phân $\int_{0}^{1} sin(\pi x^2/2) dx$ và giả sử rằng chúng ta muốn tính gần đúng tích phân với sai số bé hơn $1e-4$. \\
a. Nếu chúng ta dùng quy tắc hình thang với các nút cách đều thì bước $h$ cần dùng là bao nhiêu?\\
b. Câu hỏi tương tự với quy tắc Simpson?	
\end{bt}

\begin{bt}\label{bt5}
	Độ dài của một đường cong được biểu diễn bởi hàm số $y=f(x)$ trên một đoạn $[a,b]$ được tính bởi tích phân $I(f) = \int_{a}^{b} \sqrt{1+[f'(x)]} dx$.\\ 
	Sử dụng các hàm vừa viết trong bài tập trước để tính độ dài của các đường cong sau. \\ 
	(a) $f(x)=sin(\pi x)$, $0\leq x \leq 1$, \qquad (b) $f(x)=e^x$, $0\leq x \leq 1$, \qquad (c) $f(x)=e^{x^2}$, $0\leq x \leq 1$.
\end{bt}

\begin{bt}
Hãy thực hiện lại phương pháp hệ số bất định để tìm các hệ số của phương pháp Newton-Cotes ứng với các trường hợp $n=1$ (hình thang), $n=2$ (Simpson 1/3) và $n=3$ (Simpson 3/8). 
\end{bt}


\end{document}

\vspace{1cm}
\noindent{\bf Chú ý:} {\it Cán bộ coi thi không giải thích gì thêm}\\
\Closesolutionfile{ans}
\newpage
\begin{center}
{\LARGE{\bf ĐÁP ÁN}}
\end{center}

\begin{sol}
	\begin{figure}[h!]
		\centering
		\includegraphics[width=0.8\linewidth]{Solution1/Sol4_1.png}
		%\caption{}
		\label{fig:Sol4}
	\end{figure}
	Exercise 7: Convergence order is 3.	
\end{sol}

   
\end{document}



