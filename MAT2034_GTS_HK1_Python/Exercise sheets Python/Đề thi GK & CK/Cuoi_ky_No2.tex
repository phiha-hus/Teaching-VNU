\documentclass[11pt]{article}
%\usepackage{extsizes}
\usepackage{amsmath,amssymb}
%\usepackage{omegavn,ocmrvn}
%\usepackage[utf8x]{inputenc}
\usepackage[utf8]{vietnam}

\usepackage{framed}
\usepackage[most]{tcolorbox}
\usepackage{xcolor}
\colorlet{shadecolor}{orange!15}


\usepackage{longtable}
\usepackage{answers}
\usepackage{graphicx}
\usepackage{array}
\usepackage{pifont}
\usepackage{picinpar}
\usepackage{enumerate}
\usepackage[top=3.0cm, bottom=3.5cm, left=3.5cm, right=2.5cm] {geometry}
\usepackage{hyperref}


\newtheorem{bt}{Câu}
\newcommand{\RR}{\mathbb R}
\Newassociation{sol}{Solution}{ans}
\newtheorem{ex}{Câu}
\renewcommand{\solutionstyle}[1]{\textbf{ #1}.}


\begin{document}

\begin{tabular*}
	{\linewidth}{c>{\centering\hspace{0pt}} p{.7\textwidth}}
	Trường ĐHKHTN, ĐHQGHN & {\bf Học Kỳ 2 (2021-2022)}
	\tabularnewline
	K64 TTƯD - Thầy Hà Phi & {\bf Bài Tập Giải Tích Số \\ \today}
	% Exercises on pages 239, 240 Cheney/Kincaid are really nice
	\tabularnewline
	\rule{1in}{1pt}  \small  & \rule{2in}{1pt} %(Due date:)
	\tabularnewline
	%  \tabularnewline
	%  &(Đề thi có 1 trang)
\end{tabular*}




\begin{center}
	\textbf{ĐỀ THI CUỐI KỲ: GIẢI TÍCH SỐ - MAT 2404 \\ Thời gian: 13h - 14h30 ngày 20/01/2022 \\ Mỗi Câu Trong Đề Thi Này Tương Ứng Với 2 Ý Trong Google Form } 
\end{center}

\newcommand{\m}[1]{
	\begin{bmatrix}
		#1
	\end{bmatrix}
}

\begin{bt}\textbf{Đề lập trình thì riêng bài này không cần điền đáp số vào google form.} \\
Xét hệ phương trình tuyến tính
	%
	\begin{equation}
		\m{-4 & -1 & 0 & 0 \\ -1 & -5 & -1 & 0 \\ 0 & 1 & 6 & 1 \\ -1 & 0 & -1 & -4} x = \m{-1 \\ -7 \\ 16 \\ -14}
	\end{equation}
	%
	a) Sử dụng phương pháp Gauss-Seidel. Hãy chuyển các công thức trên về dạng $x^{(i+1)} = K x^{(i)} + k$ trong đó $K$, $k$ là ma trận và vector phù hợp. Phương pháp lặp có hội tụ hay không, vì sao? \\ 
	b) Với $x^{(0)}=\m{0 & 0 & 0 & 0}^T$. Hãy ước lượng sai số hậu nghiệm của $x^{(3)}$. Sử dụng ước lượng tiên nghiệm, hãy tìm $n$ để sai số tuyệt đối của $x^{(n)}$ nhỏ hơn $tol=1e-9$.
\end{bt}

\begin{bt} \textbf{Google form chỉ cần điền kết quả chính xác đến 4 chữ số thập phân.}\\
	Sử dụng phương pháp bình phương tối thiểu, hãy tìm hàm có dạng $f(x) = A \sqrt[4]{x}+ \dfrac{B}{x}$ để xấp xỉ tốt nhất bảng số liệu sau. 
	\begin{center}
		\begin{tabular}[5]{l|l|l|l|l}
			x    & 18  &  19  &  20  &  25 \\ \hline
			f(x) & 1.2 & 1.5  & 1.9  & 2.1
		\end{tabular}	
	\end{center}
	% 
	a) Câu 2A trong google form điền kết quả A = ? \\
	b) Câu 2B trong google form điền kết quả B = ? \\
\end{bt}

\begin{bt} \textbf{Google form chỉ cần điền kết quả chính xác đến 4 chữ số thập phân.} \\
	Cho bảng số liệu sau. 
	\begin{center}
		\begin{tabular}[5]{l|l|l|l|l}
			x    & 1.0 & 1.7 & 2.2 & 2.5 \\ \hline
			y(x) & 3.7 & 4.3 & 5.8 & 6.7
		\end{tabular}	
	\end{center}
	% 
	a) Hãy sử dụng công thức nội suy Lagrange để tính gần đúng đa thức nội suy $y(x)$ tại $x=1.2$. \\
	b) Sử dụng công thức nội suy Newton để tính gần đúng đa thức nội suy $y'(x)$ tại $x=1.2$.	
\end{bt}

\begin{bt} \textbf{Google form chỉ cần điền kết quả chính xác đến 4 chữ số thập phân.} \\
	Xét bài toán Cauchy  
	\begin{align}
		y'(x) &= y^2 \ (x+1), \quad 1 \leq x, \\
		y(1)  &= 2 \ . 
	\end{align}
	%
	a) Sử dụng phương pháp trung điểm hiện, tính xấp xỉ $y(x)$ tại $x=1.2$ với bước $h = 0.1$. \\
	b) Sử dụng phương pháp Euler ẩn, tính xấp xỉ $y(x)$ tại $x=1.2$ với bước $h = 0.1$. \\	
\end{bt}

\begin{bt} \textbf{Phần thi viết bắt buộc.} \\ 
	Hãy xác định các hằng số $a$, $b$, $c$, $d$, $e$ sao cho quy tắc cầu phương sau có cấp chính xác là 4.
	%
	\[  \int_{-1}^{9} f(x) dx = a f(-1) + bf (9) + cf(5) + d f'(-1) + e f'(9) \ .  \]
	%
\end{bt}

\centerline{———————————Hết——————————-}

\end{document}


\vspace{1cm}
\noindent{\bf Chú ý:} {\it Cán bộ coi thi không giải thích gì thêm}\\
\Closesolutionfile{ans}
\newpage
\begin{center}
{\LARGE{\bf ĐÁP ÁN}}
\end{center}

\begin{sol}
	\begin{figure}[h!]
		\centering
		\includegraphics[width=0.8\linewidth]{Solution1/Sol4_1.png}
		%\caption{}
		\label{fig:Sol4}
	\end{figure}
	Exercise 7: Convergence order is 3.	
\end{sol}

   
\end{document}



