\documentclass[12pt,a4paper]{exam}
\usepackage[top=1.5cm, bottom=2cm, left=2cm, right=2cm] {geometry}
\usepackage[utf8]{vietnam}
\usepackage{amsmath,amssymb,amsfonts,mathrsfs}
\usepackage{times,mathptmx}
\usepackage{longtable,tabularx,multirow}
\usepackage{graphicx}
\usepackage{enumitem,indentfirst}
\title{Template for VNU-HUS final test}
\author{Ha Phi}
\date{\today}

\newtheorem{bt}{Câu}
\newcommand{\R}{\mathbb R}
%\Newassociation{sol}{Solution}{ans}
\newtheorem{ex}{Câu}
%\renewcommand{\solutionstyle}[1]{\textbf{ #1}.}

\begin{document}
	
	\pagestyle{empty}
	%\fontsize{13}{18}\selectfont
	
	\begin{longtable}{p{4cm}p{0.5cm}p{4cm}p{4cm}}
		\multicolumn{1}{c}{\bf ĐẠI HỌC QUỐC GIA HÀ NỘI } & &  \multicolumn{2}{l}{\bf ĐỀ THI KẾT THÚC HỌC PHẦN} \\
		
		\multicolumn{1}{c}{\bf TRƯỜNG ĐẠI HỌC KHOA HỌC TỰ NHIÊN} & & \multicolumn{2}{l}
		{{\bf HỌC KỲ I, NĂM HỌC 2022-2023}} \\
	\end{longtable}
	
	\begin{longtable}{p{5cm}p{0.5cm}p{8cm}p{4cm}}
		\multicolumn{1}{l}{Tên HP: {\bf GIẢI TÍCH SỐ}} & &  \multicolumn{1}{l}{Mã HP: {\bf MATH 2034} \hskip 3cm Số tín chỉ: {\bf 03}} &  \\
		
		\multicolumn{1}{l}{\bf Mã đề: 02} & & \multicolumn{1}{l}{
			{\it Dành cho sinh viên lớp học phần: \textbf{MAT2034 2/3 CLCMTTT} } 
		} & \\
		
		\multicolumn{1}{c}{(Đề thi gồm có {\bf 01} trang)} & & \multicolumn{2}{l}{Thời gian làm bài: {\bf 120 phút} (không kể thời gian phát đề)}\\
	\end{longtable}
	
	\vspace*{0.3cm}
	%\noindent

	\begin{bt}\textbf{(2.5 điểm)} \ Xét bài toán Cauchy  
		\begin{align}
			y'(x) &= (x+1)/y, \quad \forall \ 1 \leq x, \\
			y(1)  &= 0.5 \ . 
		\end{align}
		%
		a) Sử dụng phương pháp Euler ẩn, tính xấp xỉ $y(x)$ tại $x=1.2$ với bước $h = 0.1$. \\
		b) Sử dụng phương pháp trung điểm hiện, tính xấp xỉ $y(x)$ tại $x=1.2$ với bước $h = 0.1$. \\		
	\end{bt}

	\begin{bt}\textbf{(2.5 điểm)} \\ 
	a) Lập bảng các công thức tính đạo hàm, tích phân như dưới đây. 
	\begin{center}
	\begin{tabular}{|c|c|}
		\hline
		\rule[-1ex]{0pt}{2.5ex} Yêu cầu sử dụng & Công thức tính \\
		\hline
		\rule[-1ex]{0pt}{3ex} Sai phân tiến/lùi tính $f'(x)$  &  \\
		\hline
		\rule[-1ex]{0pt}{3ex} Sai phân trung tâm tính $f'(x)$ &  \\
		\hline
		\rule[-1ex]{0pt}{3ex} Sai phân trung tâm tính $f"(x)$ &  \\ 
		\hline
		\rule[-1ex]{0pt}{4ex} Cầu phương Gauss tính tích phân 
		$\int_{-1}^{1} f(x)dx$ &  \\
		\hline
	\end{tabular}
	\end{center}
	b) Viết 1 đoạn code để tính tích phân $\int_{a}^{b} f(x)dx$ sử dụng công thức Simpson 1/3 composite với sai số $tol = 1e-6$. \textbf{Chú ý: Không được sử dụng built-in function để tính tích phân có sẵn trong Python.}
	\end{bt}
	
	\begin{bt} \textbf{(3 điểm)} \ Vận tốc đo được của một chiếc xe đua được thể hiện trong bảng số liệu sau. 
		\begin{center}
			\begin{tabular}[5]{l|l|l|l|l}
				t \mbox{(giây)}   & 10 & 12 & 16 & 18 \\ \hline
				v(t) \mbox{(vận tốc m/s)} & 74 & 86 & 116 & 134
			\end{tabular}	
		\end{center}
		% 
		a) Nếu vận tốc của chiếc xe đua được giả sử là 1 đa thức bậc hai của biến thời gian $t$ thì bảng dữ liệu đo đạc trên có chính xác không? Vì sao? \\
		b) Dựa vào bảng dữ liệu đó, vẫn giả sử vận tốc của chiếc xe đua được giả sử là 1 đa thức bậc hai, hãy tính gần đúng vận tốc của chiếc xe đua tại thời điểm $t = 14$ giây.\\
		c) Nếu vận tốc tính trong câu a) là chính xác 100\%, hãy tính xấp xỉ gia tốc của chiếc xe đua tại thời điểm $t = 14$ giây bằng phương pháp sai phân hữu hạn theo cách tốt nhất mà em có thể làm được. \\
		d) Biết vận tốc $v(t)$ là một hàm đa thức bậc 2 của $t$. Hãy tính xấp xỉ gia tốc tại thời điểm $t = 17$ giây bằng cách sử dụng phương pháp bình phương tối thiểu.
	\end{bt}
	
	\begin{bt}\textbf{(2 điểm)} \\ 
	a) Trong Câu 1 quy tắc cầu phương Gauss được áp dụng cho đoạn lấy tích phân $[-1,1]$, hãy xây dựng quy tắc cầu phương tương tự để tính tích phân trên đoạn $[0,n]$. \\
	b) Hãy xác định các hằng số $a$, $b$, $c$, và $d$ sao cho quy tắc cầu phương sau có cấp chính xác là 3.
	%
	\[ \int_{-1}^{1} f(x) dx = a f(-1) + bf (1) + cf'(-1) + df'(1) \ .
	\]
	% 
	\end{bt}
	

	\centerline{\rule[0.1cm]{1cm}{0.5pt} HẾT \rule[0.1cm]{1cm}{0.5pt}}
	
	\textit{\textbf{\underline{Lưu ý:}}}
	
	\begin{itemize}
		\setlength{\itemindent}{1.5cm}
		\item[$-$] {\it Thí sinh \textbf{không} được sử dụng tài liệu khi làm bài}.
		\item[$-$] {\it Cán bộ coi thi không giải thích gì thêm}.
	\end{itemize}
	
\end{document}

	\vfill
	
	\newpage
	
	\begin{longtable}{|p{8cm}|p{8cm}|}
		\hline 
		\multicolumn{2}{|c|}{\bf Không in phần này khi sao in đề thi} \\ \hline
		\multicolumn{1}{|c|}{Ngày ... tháng ... năm 20...} & \multicolumn{1}{c|}{Ngày ... tháng ... năm 20...} \\
		\multicolumn{1}{|c|}{Trưởng bộ môn duyệt} & \multicolumn{1}{c|}{Giảng viên ra đề } \\
		\multicolumn{1}{|c|}{(kí và ghi rõ họ tên)} & \multicolumn{1}{c|}{(kí và ghi rõ họ tên)} \\ 
		\vspace{2.5cm} & \\
		\hline
	\end{longtable}
	
	\newpage
	
	\begin{longtable}{p{4cm}p{0.5cm}p{4cm}p{4cm}}
		\multicolumn{1}{c}{\bf TRƯỜNG ĐHKHTN, ĐHQGHN} & &  \multicolumn{2}{c}{\bf ĐÁP ÁN ĐỀ THI KẾT THÚC HỌC PHẦN}\\
		
		\multicolumn{1}{c}{\bf KHOA TOÁN} & & \multicolumn{2}{l}{Tên HP: {\bf ...}} \\
		
		& &  \multicolumn{1}{l}{Mã HP: {\bf MAT...}} &  \multicolumn{1}{l}{Số tín chỉ: {\bf ...}}\\
		
		\multicolumn{1}{c}{\bf Đáp án chính thức} & & \multicolumn{1}{l}{Học kỳ: {\bf I/II}} & \multicolumn{1}{l}{Năm học: {\bf 20...--20...}}\\
		
		\multicolumn{1}{c}{\bf Đề số 1} & & \multicolumn{2}{l}{Ngày thi: \textbf{.../.../20...}} \\
		
		\multicolumn{1}{c}{(Đáp án gồm có {\bf ...} trang)} & & \multicolumn{2}{l}{Thời gian làm bài: {\bf 90 phút} (không kể thời gian phát đề)}\\
	\end{longtable}
	
	
	\begin{longtable}{|p{14.5cm}|p{1.5cm}|}
		\hline 
		\centerline{\bf Nội dung} & \centerline{\bf Điểm}\\ \hline
		\textbf{Câu 1} & \centerline{\bf 2.0 điểm} \\ \hline
		a. ... & \centerline{0.5}\\ \hline
		... & \centerline{0.5}\\ \hline
		b. ... & \centerline{0.5}\\ \hline
		... & \centerline{0.5}\\ \hline
		\textbf{Câu 2} & \centerline{\bf 2.0 điểm} \\ \hline
		a. ... & \centerline{0.5}\\ \hline
		... & \centerline{0.5}\\ \hline
		b. ... & \centerline{0.5}\\ \hline
		... & \centerline{0.5}\\ \hline
		\textbf{Câu 3} & \centerline{\bf 2.0 điểm} \\ \hline
		a. ... & \centerline{0.5}\\ \hline
		... & \centerline{0.5}\\ \hline
		b. ... & \centerline{0.5}\\ \hline
		... & \centerline{0.5}\\ \hline
		\textbf{Câu 4} & \centerline{\bf 2.0 điểm} \\ \hline
		a. ... & \centerline{0.5}\\ \hline
		... & \centerline{0.5}\\ \hline
		b. ... & \centerline{0.5}\\ \hline
		... & \centerline{0.5}\\ \hline
		\textbf{Câu 5} & \centerline{\bf 2.0 điểm} \\ \hline
		a. ... & \centerline{0.5}\\ \hline
		... & \centerline{0.5}\\ \hline
		b. ... & \centerline{0.5}\\ \hline
		... & \centerline{0.5}\\ \hline
	\end{longtable}
	
	\centerline{\rule[0.1cm]{1cm}{0.5pt} HẾT \rule[0.1cm]{1cm}{0.5pt}}

	\vfill
	
	\newpage
	
	\begin{longtable}{|p{8cm}|p{8cm}|}
		\hline 
		\multicolumn{2}{|c|}{\bf Không in phần này khi sao in đề thi} \\ \hline
		\multicolumn{1}{|c|}{Ngày ... tháng ... năm 20...} & \multicolumn{1}{c|}{Ngày ... tháng ... năm 20...} \\
		\multicolumn{1}{|c|}{Trưởng bộ môn duyệt} & \multicolumn{1}{c|}{Giảng viên ra đề } \\
		\multicolumn{1}{|c|}{(kí và ghi rõ họ tên)} & \multicolumn{1}{c|}{(kí và ghi rõ họ tên)} \\ 
		\vspace{2.5cm} & \\
		\hline
	\end{longtable}
	
 