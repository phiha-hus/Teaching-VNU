\documentclass[11pt]{article}
\usepackage{amsmath}
%\usepackage{extsizes}
\usepackage{amsmath,amssymb}
%\usepackage{omegavn,ocmrvn}
%\usepackage[utf8x]{inputenc}
\usepackage[utf8]{vietnam}

\usepackage{listings}
\lstset{language=Python}          % Set your language (you can change the language for each code-block optionally)

\usepackage{longtable}
\usepackage{answers}
\usepackage{graphicx}
\usepackage{array}
\usepackage{pifont}
\usepackage{picinpar}
\usepackage{enumerate}
\usepackage[top=3.0cm, bottom=3.5cm, left=3.5cm, right=2.5cm] {geometry}

\usepackage{hyperref}


\newtheorem{bt}{Câu}
\newcommand{\RR}{\mathbb R}
\Newassociation{sol}{Solution}{ans}
\newtheorem{ex}{Câu}
\renewcommand{\solutionstyle}[1]{\textbf{ #1}.}


\begin{document}
% \noindent
\begin{tabular*}
{\linewidth}{c>{\centering\hspace{0pt}} p{.7\textwidth}}
Trường ĐHKHTN, ĐHQGHN & {\bf Học Kỳ 1 (2020-2021)}
\tabularnewline
K63 TTƯD - Thầy Hà Phi & {\bf Bài Tập Giải Tích Số. No 3 \\ Các Phương pháp Phân Đôi và Lặp Đơn}
\tabularnewline
\rule{1in}{1pt}  \small  & \rule{2in}{1pt} %(Due date:)
\tabularnewline

%  \tabularnewline
%  &(Đề thi có 1 trang)
\end{tabular*}
%
% \Opensolutionfile{ans}[ans1]

\centerline{——————————— BÀI TẬP PHƯƠNG PHÁP PHÂN ĐÔI ——————————-}

\begin{bt}
	Hãy viết hàm phân đôi (bisection trong Python có dạng) sau:
	%
	\begin{lstlisting}[frame=single] 
	def bisection(f,a,b,nmax,tol):
	return c, err, n
	\end{lstlisting}
	%
	trong đó f là hàm số; a,b: là điểm đầu và điểm cuối của đoạn ta tìm nghiệm; nmax: số lượng tối đa các phép lặp; tol: sai số nhỏ nhất cho phép; c: nghiệm xấp xỉ;
	err: đánh giá cận trên của sai số tuyệt đối của nghiệm xấp xỉ; n: số bước lặp thực hiện để tìm c.\\
	\textbf{Trong các câu sau hãy sử dụng chính hàm bisection vừa viết trong câu 1 để thực hành trong Python.}
\end{bt}

\begin{bt} 
	Sử dụng phương pháp phân đôi để tìm nghiệm của các phương trình sau, với sai số $tol=1e-6$.\\ 
	(a) Các nghiệm thực của phương trình $x^3 - x^2 - x - 1 = 0$. \\
	(b) Nghiệm của phương trình $x = 1 + 0.3 cos(x)$. \\
	(c) Nghiệm dương nhỏ nhất của $cos(x) = 1/2 + sin (x)$. \\
	(d) Nghiệm của $x = e^{-x}$. \\
	(e) Nghiệm dương nhỏ nhất của phương trình $e^{-x} = sin(x)$. 
\end{bt}

\begin{bt}
	Vẽ đồ thị của hai hàm số ở 2 vế của phương trình $x = tan(x)$, và quan sát giao điểm của 2 đồ thị đó.\\
	(a) Dựa vào đồ thị, hãy chọn 2 điểm đầu mút $a$, $b$ cho phương pháp phân đôi để tìm nghiệm dương nhỏ nhất của phương trình $x = tan (x)$, với độ chính xác 
	$\epsilon=1e-4$. \\
	(b) Tìm nghiệm gần $100$ nhất của phương trình $x = tan(x)$ sử dụng phương pháp phân đôi.
\end{bt}

\begin{bt}
	a) Viết script trong Python sử dụng phương pháp phân đôi để tìm tất cả các nghiệm của phương trình sau với độ chính xác $\epsilon=1e-6$.
	\[ f(x) := 32x^6 - 48x^4 +18x^2 - 1 = 0.  \] 
	b) Tính số bước lặp theo công thức trong lý thuyết và so sánh với kết quả lập trình trong Python.
\end{bt}

\centerline{——————————— BÀI TẬP PHƯƠNG PHÁP LẶP ĐƠN ——————————-}

\begin{bt}
	Hãy viết hàm lặp đơn trong Python dạng sau:
	%
	\begin{lstlisting}[frame=single] 
	def lapdon(f,x0,nmax,tol):
	return c, err, n
	\end{lstlisting}
	%
	trong đó f là hàm số; x0 là giá trị đầu tiên của phép lặp đơn; nmax: số lượng tối đa các phép lặp; tol: sai số nhỏ nhất cho phép; c: nghiệm xấp xỉ;
	err: đánh giá cận trên của sai số tuyệt đối của nghiệm xấp xỉ; n: số bước lặp thực hiện để tìm c.\\
	\textbf{Trong các câu sau hãy sử dụng chính hàm vừa viết trong câu 5 để thực hành trong Python, khi mà ta có tính chất ánh xạ co của hàm f.}
\end{bt}

\begin{bt} % Exercises 3 & 4, Atkinson/Han p.106
a) Phương trình sau có bao nhiêu nghiệm $x=e^{-x}$? Phép lặp đơn $x_{n+1}=e^{-x_n}$ có hội tụ với giá trị $x_0$ phù hợp hay không? Viết Python script để tính $6$ giá trị đầu tiên của $x_n$.\\
b) Câu hỏi tương tự với phép lặp $x_{n+1} = 1 + \arctan(x_n)$.
\end{bt}

\begin{bt} % Exercise 5, Atkinson/Han p.107
Chứng minh rằng với các hằng số c, d thỏa mãn $|d|<1$, phương trình $x=c+d \cos(x)$ có nghiệm duy nhất. Kiểm tra tính hội tụ của phép lặp $x_{n+1}=c+d \cos(x_n)$ và hãy đưa ra đánh giá cho tốc độ hội tụ.
\end{bt}

\begin{bt} % Exercise 8, Atkinson/Han p.107
Các phép lặp sau có hội tụ đến $\alpha$ hay không? Nếu hội tụ, hãy xác định tốc độ hội tụ, cho $x_0$ đủ gần $\alpha$.
%
\[
a) x_{n+1} = \cfrac{15 x_n^2-24x_n+13}{4x_n} \ , \ \alpha=1, \hskip 2cm \hfill  b) x_{n+1} = \cfrac{3}{4} x_n + \cfrac{1}{x_n^3} \ , \ \alpha=\sqrt{2}.
\]
%
a) Sử dụng công thức ước lượng tiên nghiệm, hãy tìm số bước lặp cần thiết để nhận được xấp xỉ với sai số tuyệt đối không quá $1e-5$, với $x_0 = \alpha + 1e-1$. \\
b) Cùng câu hỏi, nhưng cho sai số tương đối
\end{bt}

\begin{bt} % Exercise 11, Atkinson/Han p.108
Phép lặp đơn $x_{n+1}=2-(1+c)x_n + c x_n^3$ sẽ hội tụ đến $\alpha=1$ với một số giá trị của c, giả sử $x_0$ đủ gần $\alpha$. \\
a) Tìm tất cả mọi $c$ để phép lặp đơn này hội tụ. Tìm mọi $c$ để phép lặp đơn này hội tụ bậc hai, tức là $|x_{n+1}-\alpha| = \mathcal{O} (\ |x_n-\alpha|^2)$. \\
b) Với một $c$ như vậy, hãy tính số bước lặp cần thiết để đạt được 5 chữ số chắc, cho điều kiện ban đầu $x_0=\alpha + 1e-3$.  
\end{bt}

\begin{bt} 
Phương trình $x^3+4x^2-10=0$ có nghiệm duy nhất trong đoạn $[1, 2]$. Có rất nhiều các khác nhau để chuyển về bài toán tìm điểm bất động. Hãy xét sự hội tụ của các phép lặp đơn sau, với điều kiện đầu $x_0=1.5$. Tìm bậc hội tụ của các phương pháp đó (nếu có) và tính sai số với $n=1,...,10$, từ đó so sánh với phương pháp phân đôi.\\
%
\begin{tabular}{lll}
a) $x=g_1(x)=x+x^3+4x^2-10$	&  & b) $x=g_2(x) = \sqrt{10/x - 4x}$ \\ 
c) $x=g_3(x)=\cfrac{1}{2}\sqrt{10-x^3}$	&  &  d) $x=g_2(x) = \sqrt{\cfrac{10}{x+4}}$ \\ 
e) $x=g_3(x)=x-\cfrac{x^3+4x^2-10}{3x^2+8x}$	&  &  \\ 
\end{tabular} 
%	
\end{bt}

\begin{bt}
Cho các phương trình sau
%
\[ a) \ 3(2x-1)= \cos(x) \qquad \qquad b) \ x^4-2x-3=0  \]
%
Hãy xây dựng cho mỗi phương trình một phương pháp lặp đơn hội tụ, biết rằng phương trình a) (t.ứ. b)) có nghiệm duy nhất trong $(0,1)$ (t.ứ. $(0,2)$). Viết các công thức đánh giá sai số tiên nghiệm, hậu nghiệm sao cho sai số nhỏ hơn $1e-4$.
\end{bt}

\centerline{———————————Hết——————————-}

\end{document}

\vspace{1cm}
\noindent{\bf Chú ý:} {\it Cán bộ coi thi không giải thích gì thêm}\\
\Closesolutionfile{ans}
\newpage
\begin{center}
{\LARGE{\bf ĐÁP ÁN}}
\end{center}

\begin{sol}
	\begin{figure}[h!]
		\centering
		\includegraphics[width=0.8\linewidth]{Solution1/Sol4_1.png}
		%\caption{}
		\label{fig:Sol4}
	\end{figure}
	Exercise 7: Convergence order is 3.	
\end{sol}

   
\end{document}



