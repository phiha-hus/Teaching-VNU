\documentclass[11pt]{article}
\usepackage{amsmath}
%\usepackage{extsizes}
\usepackage{amsmath,amssymb}
%\usepackage{omegavn,ocmrvn}
%\usepackage[utf8x]{inputenc}
\usepackage[utf8]{vietnam}

\usepackage{listings}
\lstset{language=Python}          % Set your language (you can change the language for each code-block optionally)


\usepackage{longtable}
\usepackage{answers}
\usepackage{graphicx}
\usepackage{array}
\usepackage{pifont}
\usepackage{picinpar}
\usepackage{enumerate}
\usepackage[top=3.0cm, bottom=3.5cm, left=3.5cm, right=2.5cm] {geometry}
\usepackage{hyperref}


\newtheorem{bt}{Câu}
\newcommand{\RR}{\mathbb R}
\Newassociation{sol}{Solution}{ans}
\newtheorem{ex}{Câu}
\renewcommand{\solutionstyle}[1]{\textbf{ #1}.}


\begin{document}
% \noindent

\begin{tabular*}
	{\linewidth}{c>{\centering\hspace{0pt}} p{.7\textwidth}}
	Trường ĐHKHTN, ĐHQGHN & {\bf Học Kỳ 2 (2021-2022)}
	\tabularnewline
	K64 TTƯD - Thầy Hà Phi & {\bf Bài Tập Giải Tích Số \\ \today}
	% Exercises on pages 239, 240 Cheney/Kincaid are really nice
	\tabularnewline
	\rule{1in}{1pt}  \small  & \rule{2in}{1pt} %(Due date:)
	\tabularnewline
	%  \tabularnewline
	%  &(Đề thi có 1 trang)
\end{tabular*}




\begin{center}
	{\bf Bài Tập Giải Tích Số. No 4c \\ Phương pháp Bình Phương Tối Thiểu (Least Square)}
\end{center}

\begin{center}
BÀI TẬP LÝ THUYẾT \\
\vskip .5cm 
Trong các bài tập từ 1-4, hãy sử dụng phương pháp bình phương tối thiểu theo cả 2 cách: lập phương trình chính tắc, và sử dụng phương pháp QR.
\end{center}

\begin{bt} % Cheney/Kincaid 07, page 503, Ex. 13
Hãy tìm hàm $f(x)=ax+b$ để xấp xỉ tốt nhất bảng số liệu sau theo phương pháp bình phương tối thiểu
%
\begin{center}
\begin{tabular}[7]{l|l|l|l|l|l|l}
	x & 1 & 2 & 3  & 4 \\ \hline 
	y & 0 & 1 & 1  & 2
\end{tabular}	
\end{center}
\end{bt}

\begin{bt} % Che/Kincaid 07, page 503, Ex. 16
Độ nhớt của một chất lưu là thông số đại diện cho ma sát trong của dòng chảy. Độ nhớt được biểu diễn qua một hàm bậc hai của nhiệt độ T, tức là $V = a + bT + cT^2$. Hãy tìm hàm xấp xỉ tốt nhất bảng số liệu sau theo phương pháp bình phương tối thiểu.
%
\begin{center}
\begin{tabular}[7]{l|l|l|l|l|l|l|l}
T & 1    & 2    & 3    & 4    & 5    & 6    & 7 \\ \hline
V & 2.31 & 2.01 & 1.80 & 1.66 & 1.55 & 1.47 & 1.41.
\end{tabular}	
\end{center}
%
\end{bt}

\begin{bt} a) Cường độ phóng xạ của một nguồn phóng xạ được cho bởi công thức $y=ae^{bx}$. Hãy xác định các tham số để xấp xỉ tốt nhất bảng số liệu sau theo phương pháp bình phương tối thiểu.
%
\begin{center}
	\begin{tabular}[5]{l|l|l|l|l|l}
		x & 0.2    & 0.3    & 0.4    & 0.5    & 0.6  \\ \hline
		y & 3.16   & 2.38   & 1.75   & 1.34   & 1.00
	\end{tabular}	
\end{center}
%
b) Câu hỏi tương tự phần a) nếu như hàm số được xét có dạng $y=(ax+b)^{-1}$.
\end{bt}

\begin{bt} Cho bảng số liệu sau.
%
\begin{center}
	\begin{tabular}[10]{l|l|l|l|l|l|l|l|l|l|l}
	x & 4.0 & 4.2 & 4.5 & 4.7 & 5.1 & 5.5 & 5.9 & 6.3 & 6.8 & 7.1 \\ \hline
    y & 102.56 & 113.18 &130.11 & 142.05 & 167.53 & 195.14 & 224.87 & 256.73 & 299.50 & 326.72
 	\end{tabular}	
\end{center}
%
a. Hãy tìm đa thức bậc 1 (tuyến tính) tốt nhất theo nghĩa bình phương tối thiểu. Tìm sai số. \\
b. Hãy tìm đa thức bậc 2 tốt nhất theo nghĩa bình phương tối thiểu. Tìm sai số. \\
c. Hãy tìm đa thức bậc 3 tốt nhất theo nghĩa bình phương tối thiểu. Tìm sai số. \\
d. Hãy tìm hàm xấp xỉ dạng $be^{ax}$ tốt nhất theo nghĩa bình phương tối thiểu. Tìm sai số. \\
e. Hãy tìm hàm xấp xỉ dạng $bx^{a}$ tốt nhất theo nghĩa bình phương tối thiểu. Tìm sai số. 
\end{bt}

\newpage 

\begin{center}
	BÀI TẬP LẬP TRÌNH
\end{center}

\begin{bt}
Viết 2 hàm trong Python như sau. Hàm 1 để giải bài toán $\min \|Ax-b\|$ có dạng
%
\begin{lstlisting}[frame=single] 
def least_square(A,b):
return x
\end{lstlisting}
%
để tìm bình phương tối thiểu. Cho trước A, b, đi tìm x bằng 2 cách: phương trình chính tắc và QR. Chú ý có thể gọi ngay $numpy.linalg.qr$ nếu cần thiết. \\
Hàm 2 để giải bài toán curve fitting có dạng
%
\begin{lstlisting}[frame=single] 
def curve_fitting(x,y,n):
return p
\end{lstlisting}
%
trong đó x, y là các vector lưu dữ liệu, n là bậc của đường cong đa thức, tức là $y \approx p_0 x^n+p_1 x^{n-1}+...+p_n$. Vector output $p=[p_0 \ p_1 \ ... \ p_n]$.
\end{bt}

\begin{bt}
Giải 4 bài tập trên bằng hai hàm các em vừa viết. So sánh kết quả vừa tìm được với kết quả của việc dùng 2 built-in function trong Python. \\ \vskip .2cm

\noindent $numpy.linalg.lstsq$ : giải bài toán tìm bình phương tối thiểu,  \\
\url{https://docs.scipy.org/doc/numpy/reference/generated/numpy.polyfit.html} \\ \vskip .2cm

\noindent $numpy.polyfit$ : tìm các hệ số của đa thức xấp xỉ tốt nhất cho dữ liệu dạng bảng. \\
\url{https://docs.scipy.org/doc/numpy/reference/generated/numpy.linalg.lstsq.html#numpy.linalg.lstsq}
\end{bt}

\begin{bt} Để tìm hàm nghiệm dạng $g(x)=a + bx^{-1} + cx^{-2}$ của bài toán tìm bình phương tối thiểu của một bảng số liệu dạng
	%
	\begin{center}
		\begin{tabular}[5]{l|l|l|l|l}
			x & $x_0$ & $x_1$ & $\dots$ & $x_n$  \\ \hline
			y & $y_0$ & $y_1$ & $\dots$ & $y_n$
		\end{tabular}	
	\end{center}
	%
	một sinh viên quyết định biến đổi bài toán thành tìm nghiệm dạng $x^2 g(x) = ax^2 + bx + c$	của bài toán mới tương ứng với bảng số liệu
	%
	\begin{center}
		\begin{tabular}[5]{l|l|l|l|l}
			x & $x_0$ & $x_1$ & $\dots$ & $x_n$  \\[.2pt] \hline
			y & $x_{0}^2y_0$ & $x_{1}^2 y _1$ & $\dots$ & $x_{n}^2 y _n$
		\end{tabular}	
	\end{center}
	%
	Hỏi kết quả hàm $g(x)$ trong hai bài toán này có trùng nhau không? Vì sao? Gợi ý: thử lập phương trình chính tắc cho TH n=2 xem.
\end{bt}

\centering{------------------------------Hết------------------------------}

\newpage 

\centering{------------------------------PHẦN NÀY BỎ TRONG CHƯƠNG TRÌNH MỚI ------------------------}

\begin{center}	
	\textbf{PHẦN II: PHƯƠNG PHÁP BÌNH PHƯƠNG TỐI THIỂU TRÊN $L_2[a,b]$}
\end{center}


\begin{bt}
	Tìm đa thức nghiệm dạng $ax+b$ cho bài toán bình phương tối tiểu của hàm số $f(x)$ trog các trường hợp sau đây \\
	a. $f(x) = x^2 + 3x + 2$, $[-1,1]$; \qquad b. $f(x) = x^3$, $[-1, 1]$; \qquad c. $f(x) = \cfrac{1}{x}$\ , $[-1, 1]$; \\
	d. $f(x) = e^x$, $[0, 2]$; \qquad e. $f(x) = 1/2 cos x + 1/3 sin 2x$, $[0, 1]$; \qquad f. $f(x) = x \ln x$, $[1, 3]$.
\end{bt}

\begin{bt}
Xét hàm $f(x) = e^{2x}$ trên đoạn $[0, \pi]$. Chúng ta muốn xấp xỉ nó bằng đa thức lượng giác có dạng $p(x) = a + b \cos(x) + c \sin(x)$. Hãy lập hệ phương trình và tìm $a$, $b$, $c$ dựa vào phương pháp bình phương tối thiểu.
\end{bt}

\begin{bt} Đối với hệ phương trình tuyến tính dạng $Ax=b$, bài toán (và cả ma trận $A$) gọi là có \textbf{điều kiện xấu} nếu như số điều kiện $cond(A)$ là lớn. Sử dụng hàm $cond$ trong Matlab/Octave, hãy kiểm tra điều kiện của hệ phương trình chuẩn tắc tương ứng với các trường hợp sau:\\
a) $\min \|f-P\|_{L_2}$ với $f(x)=e^{2x}$, $[a,b]=[0,\pi]$, $p(x)=a_0 + a_1 x + a_2 x^2 $.	 \\
b) $\min \|f-P\|_{L_2}$ với $f(x)=e^{2x}$, $[a,b]=[0,\pi]$, $p(x)=a_0 + a_1 \cos(x) + a_2 \sin(x) $. \\
c) $\min \|f-P\|_{L_2}$ với $f(x)=e^{2x}$, $[a,b]=[0,\pi]$, $p(x)=a_0 L_0(x) + a_1 L_1(x) + a_2 L_2(x)$, trong đó $L_i(x)$ là đa thứ Legendre bậc $i$.\\
So sánh sai số (chính là $\min \|f-P\|_{L_2}$) trong 3 trường hợp trên và đưa ra kết luận.
\end{bt}


\centerline{———————————Hết——————————-}

\end{document}

\vspace{1cm}
\noindent{\bf Chú ý:} {\it Cán bộ coi thi không giải thích gì thêm}\\
\Closesolutionfile{ans}
\newpage
\begin{center}
{\LARGE{\bf ĐÁP ÁN}}
\end{center}

\begin{sol}
	\begin{figure}[h!]
		\centering
		\includegraphics[width=0.8\linewidth]{Solution1/Sol4_1.png}
		%\caption{}
		\label{fig:Sol4}
	\end{figure}
	Exercise 7: Convergence order is 3.	
\end{sol}

   
\end{document}



