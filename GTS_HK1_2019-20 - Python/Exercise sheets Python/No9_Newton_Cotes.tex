\documentclass[11pt]{article}
\usepackage{amsmath}
%\usepackage{extsizes}
\usepackage{amsmath,amssymb}
%\usepackage{omegavn,ocmrvn}
%\usepackage[utf8x]{inputenc}
\usepackage[utf8]{vietnam}

\usepackage{listings}
\lstset{language=Python}          % Set your language (you can change the language for each code-block optionally)


\usepackage{longtable}
\usepackage{answers}
\usepackage{graphicx}
\usepackage{array}
\usepackage{pifont}
\usepackage{picinpar}
\usepackage{enumerate}
\usepackage[top=3.0cm, bottom=3.5cm, left=3.5cm, right=2.5cm] {geometry}
\usepackage{hyperref}


\newtheorem{bt}{Câu}
\newcommand{\RR}{\mathbb R}
\Newassociation{sol}{Solution}{ans}
\newtheorem{ex}{Câu}
\renewcommand{\solutionstyle}[1]{\textbf{ #1}.}


\begin{document}
% \noindent
\begin{tabular*}
{\linewidth}{c>{\centering\hspace{0pt}} p{.7\textwidth}}
Trường ĐHKHTN, ĐHQGHN & {\bf Học Kỳ 1 (2019-2020)}
\tabularnewline
K62 TTƯD & {\bf Bài Tập Giải Tích Số. No 9 \\ Tính gần đúng tích phân \\ Các quy tắc cơ bản \\ Lớp CT Newton-Cotes}
\tabularnewline
\rule{1in}{1pt}  \small  & \rule{2in}{1pt} %(Due date:)
\tabularnewline

%  \tabularnewline
%  &(Đề thi có 1 trang)
\end{tabular*}
%
% \Opensolutionfile{ans}[ans1]

\begin{center}	
	\textbf{PHẦN I: Các quy tắc cầu phương cơ bản: Trung điểm; Hình Thang; Simpson}
\end{center}

\begin{center}
	BÀI TẬP LÝ THUYẾT \\
%	\vskip .5cm 
\end{center}

\begin{bt}
a) Hãy tính các tích phân sau sử dụng cả 3 phương pháp \emph{Hình thang/Simpson/Trung điểm} với các điểm nút cách đều và so sánh kết quả. 
\[ \int_{0}^{1} e^{-x^2} dx \ . \]
%	
b) Nếu tích phân $\int_{0}^{1} e^{-x^2} dx$ cần tính với sai số nhỏ hơn $1e-7$, sử dụng các nút cách đều trong phương pháp hình thang, hỏi chúng ta cần bao nhiêu điểm nút?
\end{bt}

\begin{bt}
Tích phân logarit là một dạng tích phân phụ thuộc tham số đặc biệt có dạng 
%
\[ \rm{li}(x) = \int_{2}^{x} \cfrac{dt}{\ln t} dt  \ . \]
%
Với $x$ đủ lớn, số lượng các số nguyên tố nhỏ hơn hoặc bằng $x$ là xấp xỉ gần bằng $\rm{li}(x)$. Ví dụ, có 46 số nguyên tố nhỏ hơn hoặc bằng 200, và li(200) thì xấp xỉ 50. Hãy tìm 
li(200) đến 3 chữ số chắc, sử dụng ba phương pháp cầu phương đã nói ở trên. \\
Gợi ý: Như vậy tìm $h$ để sai số toàn phần bé hơn 1e-3, sau đó dùng quy tắc cầu phương cho các điểm nút cách đều h.
\end{bt}

\begin{bt}
	Xét tích phân $\int_{0}^{1} sin(\pi x^2/2) dx$ và giả sử rằng chúng ta muốn tính gần đúng tích phân với sai số bé hơn $1e-4$. \\
	a. Nếu chúng ta dùng quy tắc hình thang với các nút cách đều thì độ rộng $h$ cần dùng là bao nhiêu?\\
	b. Câu hỏi tương tự với quy tắc Simpson?	
\end{bt}

\begin{center}
	BÀI TẬP THỰC HÀNH \\
	%	\vskip .5cm 
\end{center}

\begin{bt} Viết các hàm trong Python để tính tích phân dựa trên các quy tắc cầu phương Trung điểm; Hình thang; Simpson dạng composite, ví dụ như
%
\begin{lstlisting}[frame=single] 
def Simpson(f,a,b,tol):
return I, h
\end{lstlisting}
%	 
trong đó $f$ là hàm; $a$ và $b$ là 2 đầu mút, $tol$ là sai số cần đạt được của việc tính xấp xỉ tích phân, $I$ là giá trị gần đúng của tích phân, $h$ là độ rộng mỗi bước. Chú ý các nút được sử dụng là cách đều, tức là $x_0=a<x_1<\dots<x_n=b$, $x_i = a+i*h$, $h=\cfrac{b-a}{n}$. 
\end{bt}

\begin{bt}\label{bt5}
	Nhắc lại rằng độ dài của một đường cong được biểu diễn bởi hàm số $y=f(x)$ trên một đoạn $[a,b]$ được tính bởi tích phân $I(f) = \int_{a}^{b} \sqrt{1+[f'(x)]} dx$.\\ 
	Sử dụng các hàm vừa viết trong bài tập trước để tính độ dài của các đường cong sau. \\ 
	(a) $f(x)=sin(\pi x)$, $0\leq x \leq 1$, \\ 
	(b) $f(x)=e^x$, $0\leq x \leq 1$, \\ 
	(c) $f(x)=e^{x^2}$, $0\leq x \leq 1$.\\
	Dựa trên kết quả lý thuyết, hãy đánh giá các sai số toàn phần với mỗi phương pháp và mỗi đường cong.  	
\end{bt}

\begin{center}	
	\textbf{PHẦN II: Các quy tắc cầu phương Newton-Cotes}
\end{center}

\begin{bt}
Để tính tích phân $\int_{0}^{1} f(x)dx$, ngoài 3 phương pháp nêu trên người ta còn tìm các công thức Newton-Cotes khác bằng phương pháp hệ số bất định (đã học trên lớp sáng nay). Hãy sử dụng phương pháp đó để đi tìm công thức cầu phương sử dụng 4 điểm chia $0$, $\dfrac{1}{3}$, $\dfrac{2}{3}$, $1$. Công thức thu được được gọi là \emph{Quy tắc Simpson 3/8}, khác với \emph{Quy tắc Simpson 1/3} đã học trên lớp. Hãy tìm quy tắc Simpson 3/8 đó.
\end{bt}


\begin{bt}
Đối với trường hợp 5 điểm chia, ta có quy tắc Boole được sinh ra từ lớp công thức Newton-Cotes sử dụng 5 điểm chia $0$, $\dfrac{1}{4}$, $\dfrac{1}{2}$, $\dfrac{3}{4}$, $1$. Hãy tìm công thức Boole đó.
\end{bt}

\begin{bt}
Sử dụng các quy tắc Simpson 3/8 và Boole để tính xấp xỉ độ dài các đường cong trong Bài tập \ref{bt5}. So sánh thời gian tính toán và sai số của các phương pháp này với các công thức cầu phương cơ bản.
\end{bt}
\centerline{———————————Hết——————————-}

\end{document}

\vspace{1cm}
\noindent{\bf Chú ý:} {\it Cán bộ coi thi không giải thích gì thêm}\\
\Closesolutionfile{ans}
\newpage
\begin{center}
{\LARGE{\bf ĐÁP ÁN}}
\end{center}

\begin{sol}
	\begin{figure}[h!]
		\centering
		\includegraphics[width=0.8\linewidth]{Solution1/Sol4_1.png}
		%\caption{}
		\label{fig:Sol4}
	\end{figure}
	Exercise 7: Convergence order is 3.	
\end{sol}

   
\end{document}



