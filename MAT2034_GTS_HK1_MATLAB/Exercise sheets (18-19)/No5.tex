\documentclass[11pt]{article}
\usepackage{amsmath}
%\usepackage{extsizes}
\usepackage{amsmath,amssymb}
%\usepackage{omegavn,ocmrvn}
%\usepackage[utf8x]{inputenc}
\usepackage[utf8]{vietnam}

\usepackage{longtable}
\usepackage{answers}
\usepackage{graphicx}
\usepackage{array}
\usepackage{pifont}
\usepackage{picinpar}
\usepackage{enumerate}
\usepackage[top=3.0cm, bottom=3.5cm, left=3.5cm, right=2.5cm] {geometry}
\usepackage{hyperref}


\newtheorem{bt}{Câu}
\newcommand{\RR}{\mathbb R}
\Newassociation{sol}{Solution}{ans}
\newtheorem{ex}{Câu}
\renewcommand{\solutionstyle}[1]{\textbf{ #1}.}


\begin{document}
% \noindent
\begin{tabular*}
{\linewidth}{c>{\centering\hspace{0pt}} p{.7\textwidth}}
Trường ĐHKHTN, ĐHQGHN & {\bf Học Kỳ 1 (2018-2019)}
\tabularnewline
K61 TTƯD & {\bf Bài Tập Giải Tích Số. No 5 \\ Phương pháp Newton}
\tabularnewline
\rule{1in}{1pt}  \small  & \rule{2in}{1pt} %(Due date:)
\tabularnewline

%  \tabularnewline
%  &(Đề thi có 1 trang)
\end{tabular*}
%
% \Opensolutionfile{ans}[ans1]

\begin{bt} % Exercise 11, Atkinson/Han p.108
Viết hàm trong Matlab cho phương pháp Newton dạng $[x,n] = newton(f,df,x0,nmax,eps)$, sau đó áp dụng để tìm nghiệm chính xác đến $1e-5$ cho các bài toán sau.\\
a.) $e^x + 2^{-x} + 2 cos x - 6 = 0$ với $1 \leq x \leq 2$. \\
b.) $\ln(x - 1) + \cos(x - 1) = 0$ với $1.3 \leq x \leq 2$.
\end{bt}

\begin{bt} % Exercises 3 & 4, Atkinson/Han p.88
a) Trong hầu hết máy tính cũ, $\sqrt{a}$ được tính dựa trên việc sử dụng phương pháp Newton để giải phương trình $x^2=a$. Hãy lập công thức lặp Newton dựa trên lý thuyết.  \\
b) Dựa vào lý thuyết được học trên lớp, hãy thiết lập các công thức truy hồi cho sai số tuyệt đối $\varepsilon_{abs}:=|x_n-\sqrt{a}|$ và tương đối $\varepsilon_{rel}:=|\cfrac{x_n-\sqrt{a}|}{\sqrt{a}}$.\\
c) Thực hiện nhiệm vụ câu a) để tìm $\sqrt[m]{a}$ với $a>0$, $m$ là 1 số nguyên dương. Viết script Matlab với input là $a$, $m$, output là $\sqrt[m]{a}$. Áp dụng để tính $\sqrt[8]{2}$ với 6 chữ số chắc.
\end{bt}

\begin{bt} % Exercise 5, Atkinson/Han p.88
a) Giải số nghiệm gần 100 nhất của phương trình $x=tan(x)$ đến 6 chữ số chắc bằng phương pháp phân đôi (gợi ý: sử dụng đồ thị để tìm khoảng $[a,b]$ phù hợp). \\
b) Hãy thử sử dụng phương pháp Newton để giải quyết vấn đề trong câu a). Điều kiện ban đầu $x_0$ phải gần nghiệm đến mức nào để phương pháp hội tụ? Từ đó hãy kết hợp phương pháp phân đôi và phương pháp Newton như lý thuyết được học trên lớp để tìm $x_0$ và tìm nghiệm đến 6 chữ số chắc bằng phương pháp Newton.
\end{bt}

\begin{bt} % Exercises 10,11,12, Atkinson/Han p.89
a) Sử dụng phương pháp Newton, hãy viết Matlab function để tìm nghiệm phương trình $p(x)=0$ trên 1 khoảng $[a,b]$. Input là bậc/vectơ các hệ số của $p(x)$, $a$, $b$. \\
Yêu cầu thực hiện cả 2 phương pháp: 1)Tính $p(x)$, $p'(x)$ trực tiếp. 2) Sử dụng phương pháp Horner để tính $p(x)$ và $p'(x)$. \\
b) Thử nghiệm số hàm vừa viết để giải phương trình 
%
\[ x^4 - 5.4 x^3 + 10.56 x^2 - 8.954 x + 2.7951 = 0, \]
%
trên khoảng $[1,1.2]$. 
\end{bt}

\begin{bt} % Exercise 9, Atkinson/Han p.107
Tìm $a$ nhỏ nhất đến 5 chữ số chắc sao cho\\
a) $a \sqrt{x} \leq \sin(x)$ với mọi $x>0$. \\
b) $e^{-ax} \leq \cfrac{1}{1+x^2}$ với mọi $x>0$.
\end{bt}

\centerline{———————————Hết——————————-}

\end{document}

\vspace{1cm}
\noindent{\bf Chú ý:} {\it Cán bộ coi thi không giải thích gì thêm}\\
\Closesolutionfile{ans}
\newpage
\begin{center}
{\LARGE{\bf ĐÁP ÁN}}
\end{center}

\begin{sol}
	\begin{figure}[h!]
		\centering
		\includegraphics[width=0.8\linewidth]{Solution1/Sol4_1.png}
		%\caption{}
		\label{fig:Sol4}
	\end{figure}
	Exercise 7: Convergence order is 3.	
\end{sol}

   
\end{document}



