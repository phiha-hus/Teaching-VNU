\documentclass[11pt]{article}
\usepackage{amsmath}
%\usepackage{extsizes}
\usepackage{amsmath,amssymb}
%\usepackage{omegavn,ocmrvn}
%\usepackage[utf8x]{inputenc}
\usepackage[utf8]{vietnam}

\usepackage{longtable}
\usepackage{answers}
\usepackage{graphicx}
\usepackage{array}
\usepackage{pifont}
\usepackage{picinpar}
\usepackage{enumerate}
\usepackage[top=3.0cm, bottom=3.5cm, left=3.5cm, right=2.5cm] {geometry}
\usepackage{hyperref}


\newtheorem{bt}{Câu}
\newcommand{\RR}{\mathbb R}
\Newassociation{sol}{Solution}{ans}
\newtheorem{ex}{Câu}
\renewcommand{\solutionstyle}[1]{\textbf{ #1}.}


\begin{document}
% \noindent
\begin{tabular*}
{\linewidth}{c>{\centering\hspace{0pt}} p{.7\textwidth}}
Trường ĐHKHTN, ĐHQGHN & {\bf Học Kỳ 1 (2018-2019)}
\tabularnewline
K61 TTƯD & {\bf Bài Tập Giải Tích Số. No 9 \\ Tính gần đúng tích phân \\ Các Quy tắc Cầu Phương: \\ Hình Thang, Simpson, Trung điểm}
\tabularnewline
\rule{1in}{1pt}  \small  & \rule{2in}{1pt} %(Due date:)
\tabularnewline

%  \tabularnewline
%  &(Đề thi có 1 trang)
\end{tabular*}
%
% \Opensolutionfile{ans}[ans1]

\begin{bt}
Giải thích ý nghĩa hình học và viết hàm trong Matlab để tính tích phân dựa trên \emph{\textbf{Công Thức Trung Điểm với các nút các đều}} sau
%
\[ \int_{a}^{b} f(x)dx = h \ \sum_{i=0}^{n-1} f(x_i + h/2), \]
%
trong đó $h=\cfrac{b-a}{n}$, $x_i = a+i*h$.
\end{bt}

\begin{bt}
a) Hãy tính các tích phân sau sử dụng cả 3 phương pháp \emph{Hình thang/Simpson/Trung điểm} với các điểm nút cách đều và so sánh kết quả. 
\[ \int_{0}^{1} e^{-x^2} dx \ . \]
%	
b) Nếu tích phân $\int_{0}^{1} e^{-x^2} dx$ cần tính với sai số nhỏ hơn $1e-7$, sử dụng các nút cách đều trong phương pháp hình thang, hỏi chúng ta cần bao nhiêu điểm nút?
\end{bt}

\begin{bt}
Tích phân logarit là một dạng tích phân phụ thuộc tham số đặc biệt có dạng 
%
\[ \rm{li}(x) = \int_{2}^{x} \cfrac{dt}{\ln t} dt  \ . \]
%
Với $x$ đủ lớn, số lượng các số nguyên tố nhỏ hơn hoặc bằng $x$ là xấp xỉ gần bằng $\rm{li}(x)$. Ví dụ, có 46 số nguyên tố nhỏ hơn hoặc bằng 200, và li(200) thì xấp xỉ 50. Hãy tìm 
li(200) đến 3 chữ số chắc, sử dụng ba phương pháp cầu phương đã nói ở trên.
\end{bt}

\begin{bt}
Nhắc lại rằng độ dài của một đường cong được biểu diễn bởi hàm số $y=f(x)$ trên một đoạn $[a,b]$ được tính bởi tích phân $I(f) = \int_{a}^{b} \sqrt{1+[f'(x)]} dx$.\\ 
Viết các hàm tích tích phân trong Matlab sử dụng các công thức hình thang và Simpson để tính độ dài của các đường cong sau. \\ 
(a) $f(x)=sin(\pi x)$, $0\leq x \leq 1$, \\ 
(b) $f(x)=e^x$, $0\leq x \leq 1$, \\ 
(c) $f(x)=e^{x^2}$, $0\leq x \leq 1$.	
\end{bt}

\begin{bt}
	Xét tích phân $\int_{0}^{1} sin(\pi x^2/2) dx$ và giả sử rằng chúng ta muốn tính gần đúng tích phân với sai số bé hơn $1e-4$. \\
	a. Nếu chúng ta dùng quy tắc hình thang với các nút cách đều thì độ rộng $h$ cần dùng là bao nhiêu?\\
	b. Câu hỏi tương tự với quy tắc Simpson?	
\end{bt}

\centerline{———————————Hết——————————-}

\end{document}

\vspace{1cm}
\noindent{\bf Chú ý:} {\it Cán bộ coi thi không giải thích gì thêm}\\
\Closesolutionfile{ans}
\newpage
\begin{center}
{\LARGE{\bf ĐÁP ÁN}}
\end{center}

\begin{sol}
	\begin{figure}[h!]
		\centering
		\includegraphics[width=0.8\linewidth]{Solution1/Sol4_1.png}
		%\caption{}
		\label{fig:Sol4}
	\end{figure}
	Exercise 7: Convergence order is 3.	
\end{sol}

   
\end{document}



