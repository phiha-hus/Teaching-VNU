\documentclass[11pt]{article}
\usepackage{amsmath}
%\usepackage{extsizes}
\usepackage{amsmath,amssymb}
%\usepackage{omegavn,ocmrvn}
%\usepackage[utf8x]{inputenc}
\usepackage[utf8]{vietnam}

\usepackage{longtable}
\usepackage{answers}
\usepackage{graphicx}
\usepackage{array}
\usepackage{pifont}
\usepackage{picinpar}
\usepackage{enumerate}
\usepackage[top=3.0cm, bottom=3.5cm, left=3.5cm, right=2.5cm] {geometry}
\usepackage{hyperref}


\newtheorem{bt}{Câu}
\newcommand{\RR}{\mathbb R}
\Newassociation{sol}{Solution}{ans}
\newtheorem{ex}{Câu}
\renewcommand{\solutionstyle}[1]{\textbf{ #1}.}


\begin{document}
% \noindent
\begin{tabular*}
{\linewidth}{c>{\centering\hspace{0pt}} p{.7\textwidth}}
Trường ĐHKHTN, ĐHQGHN & {\bf Học Kỳ 1 (2018-2019)}
\tabularnewline
K61 TTƯD & {\bf Bài Tập Giải Tích Số. No 7 \\ Thực hành Nội suy Newton bằng tỷ sai phân}
\tabularnewline
\rule{1in}{1pt}  \small  & \rule{2in}{1pt} %(Due date:)
\tabularnewline

%  \tabularnewline
%  &(Đề thi có 1 trang)
\end{tabular*}
%
% \Opensolutionfile{ans}[ans1]

\begin{bt} % Cheney/Kincaid 07
a) Phải chăng càng nhiều mốc nội suy thì đa thức $P_n(x)$ càng xấp xỉ tốt hàm $f(x)$? Hãy xét hai trường hợp sau.\\
i) Hàm $f(x)$ là \textbf{hàm Dirichlet}, tức là $f(x)=0$ tại các điểm vô tỉ, và $f(x)=1$ tại các điểm hữu tỉ. \\
ii) Hàm $f(x)$ là \textbf{hàm Runge} $f(x) = \cfrac{1}{1+x^2}$. Hãy dùng 21 mốc nội suy cách đều trên đoạn $[-5, 5]$ để tìm đa thức nội suy bậc 20 dạng $P(x)$ của hàm Runge. Tính toán và vẽ đồ thị $f(x)$ và $P(x)$ tại 41 điểm cách đều trên đoạn $[-5, 5]$. Các em quan sát được gì ở đây? Gợi ý vẽ: dùng hàm linspace.\\
b) Hãy sử dụng 21 mốc nội suy Chebyshev $x_i=5\cos((2i+1)\pi/42)$, $0\leq i\leq 20$ để tìm đa thức nội suy của hàm Runge. Vẽ sai số tại 41 điểm cách đều trên đoạn $[-5, 5]$ của 2 phép nội suy trong câu a-ii) và b). Các em quan sát được gì ở đây?
\end{bt}

\begin{bt} a) Viết hàm Matlab để tìm tỉ sai phân dạng $v = div\_diff(x,y)$ trong đó $x$ là vector chứa các mốc nội suy, $y$ là vector lưu giá trị của hàm số tại các mốc nội suy đó, $v$ là vector output, chứa các hệ số của đa thức nội suy Newton (cũng chính là đường chéo chính của bảng tỉ sai phân đã học trên lớp). \\
b) Test script đã viết cho các bài toán nội suy sau. Hãy đưa ra nhận xét về bậc của đa thức nội suy và số lượng mốc nội suy.  \\
\begin{tabular}[7]{l|l|l|l|l|l|l}
	x & -2 & -1 & 0  & 1  & 2  & 3 \\ \hline 
	y & 1  & 4  & 11 & 16 & 13 & -4
\end{tabular}	
\hfill 
\begin{tabular}[6]{l|l|l|l|l|l}
	x & 0 & 1 & 2  & -1  & 3 \\ \hline 
	y & -1  & -1  & -1 & -7 & 5
\end{tabular}	\\
\begin{tabular}[6]{l|l|l|l|l|l}
	x & 1 & 3 & -2  & 4  & 5 \\ \hline 
	y & 2  & 6  & -1 & -4 & 2
\end{tabular}
\hfill 
\begin{tabular}[6]{l|l|l|l|l|l}
	x & 0 & 1 & 3  & 2  & 5 \\ \hline 
	y & 2  & 1  & 5 & 6 & -183
\end{tabular}\\
\end{bt}

\begin{bt}
Đa thức nội suy Newton được liên hệ với tỉ sai phân bởi công thức
%
\begin{align*}
P_n(x) = f(x_0)+f[x_0,x_1](x-x_0) &+f[x_0,x_1,x_2](x-x_0)(x-x_1)+ \dots + \\ 
&+ f[x_0,x_1,\dots,x_n](x-x_0)\cdots(x-x_{n-1}).
\end{align*}
%
a) Viết script Matlab dạng $value = newton\_interp(x,y,w)$ để tính giá trị của hàm số tại điểm $w$, sử dụng nội suy Newton với các mốc nội suy $x$, giá trị $y$. Chú ý sử dụng thuật toán Horner kết hợp với nội suy Newton. \\	
b) Test code vừa viết để tìm $f(0.5)$ cho các ví dụ trong câu b) bài tập trên. 
\end{bt}

\begin{bt} Chứng minh các ước lượng sai số của các phép nội suy mốc cách đều sau.
\begin{enumerate}
\item[i)]  $\frac{1}{8}h^2M$ đối với phép nội suy tuyến tính, trong đó $h = x_1-x_0$ và $M = \max_{x_0 \leq x \leq x_1} |f"(x)|$. 
\item[ii)] $\frac{1}{9\sqrt{3}}h^3M$ đối với nội suy bậc 2, trong đó $h = x_1-x_0 = x_2-x_1$ và $M = \max_{x_0 \leq x \leq x_2} |f"(x)|$. 
\item[iii)] $\frac{3}{128} h^4M$ đối với nội suy bậc 2, trong đó $h = x_1-x_0 = x_2-x_1= x_3-x_2$ \linebreak và $M = \max_{x_0 \leq x \leq x_3} |f"(x)|$.
\end{enumerate}	
\end{bt}

\begin{bt} Hãy lập công thức nội suy Newton cho hàm $\cos x$ bằng đa thức bậc $n$, sử dụng $n+1$ điểm cách đều trên đoạn $[0, 1]$. Thiết lập công thức đánh giá sai số cho phép nội suy này theo n? Tìm $n$ sao cho sai số này nhỏ hơn $1e-7$?	
\end{bt}

\centerline{———————————Hết——————————-}

\end{document}

\vspace{1cm}
\noindent{\bf Chú ý:} {\it Cán bộ coi thi không giải thích gì thêm}\\
\Closesolutionfile{ans}
\newpage
\begin{center}
{\LARGE{\bf ĐÁP ÁN}}
\end{center}

\begin{sol}
	\begin{figure}[h!]
		\centering
		\includegraphics[width=0.8\linewidth]{Solution1/Sol4_1.png}
		%\caption{}
		\label{fig:Sol4}
	\end{figure}
	Exercise 7: Convergence order is 3.	
\end{sol}

   
\end{document}



