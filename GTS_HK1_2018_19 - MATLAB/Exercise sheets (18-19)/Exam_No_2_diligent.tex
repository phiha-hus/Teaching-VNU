\documentclass[11pt]{article}
\usepackage{amsmath}
%\usepackage{extsizes}
\usepackage{amsmath,amssymb}
%\usepackage{omegavn,ocmrvn}
%\usepackage[utf8x]{inputenc}
\usepackage[utf8]{vietnam}

\usepackage{longtable}
\usepackage{answers}
\usepackage{graphicx}
\usepackage{array}
\usepackage{pifont}
\usepackage{picinpar}
\usepackage{enumerate}
\usepackage[top=3.0cm, bottom=3.5cm, left=3.5cm, right=2.5cm] {geometry}
\usepackage{hyperref}


\newtheorem{bt}{Câu}
\newcommand{\RR}{\mathbb R}
\Newassociation{sol}{Solution}{ans}
\newtheorem{ex}{Câu}
\renewcommand{\solutionstyle}[1]{\textbf{ #1}.}


\begin{document}
% \noindent
\begin{tabular*}
{\linewidth}{c>{\centering\hspace{0pt}} p{.7\textwidth}}
Trường ĐHKHTN, ĐHQGHN & {\bf Học Kỳ 1 (2018-2019)}
\tabularnewline
K61 TTƯD & {\bf Bài kiểm tra thường xuyên \\ Đề III \& IV}
\tabularnewline
\rule{1in}{1pt}  \small  & \rule{2in}{1pt} %(Due date:)
\tabularnewline

%  \tabularnewline
%  &(Đề thi có 1 trang)
\end{tabular*}
%
% \Opensolutionfile{ans}[ans1]

\begin{center}	
	\textbf{Đề III}
\end{center}

\begin{bt}
a) Khảo sát số nghiệm và khoảng nghiệm của phương trình 
%
\[ f(x) := x^3 - 6x^2 + 11x - 6 = 0.  \] 
%
b) Viết hàm phân đôi trong Matlab và script sử dụng hàm đó để tìm tất cả các nghiệm của phương trình trên với độ chính xác $\epsilon=1e-6$.	
Trong quá trình tính toán, hãy sử dụng công thức Horner để tối ưu hóa nếu cần ước lượng giá trị của hàm $f$.\\
c) Tính số bước lặp theo công thức trong lý thuyết để đạt được độ chính xác $\epsilon=1e-8$.
\end{bt}

\begin{bt}
Cho phương trình sau
%
\[ 3(2x-1)= \cos(x) \ .\]
%
a) Tìm số nghiệm thực và khoảng nghiệm tương ứng của phương trình đó. \\
b) Hãy xây dựng cho phương trình này một phương pháp lặp đơn hội tụ, giải thích vì sao? \\
c) Viết Matlab function/script để giải phương trình trên bằng phương pháp lặp đơn trong câu b. Output cần có cả nghiệm, số bước lặp và đánh giá sai số tuyệt đối. \\
d) Cần bao nhiêu bước lặp để sai số tuyệt đối nhỏ hơn $1e-6$. \\
e) Viết công thức đánh giá hậu nghiệm cho sai số và áp dụng cho phép lặp đơn trong câu b.
\end{bt}

\begin{center}	
	\textbf{Đề IV}
\end{center}

\begin{bt}
Cho phương trình sau
%
\[ x^4-2x-3=0  \ . \]
%
a) Tìm số nghiệm thực và các khoảng nghiệm tương ứng của phương trình đó. \\
b) Chọn 1 khoảng nghiệm em mong muốn. Hãy xây dựng cho phương trình này một phương pháp lặp đơn hội tụ, giải thích vì sao? \\
c) Viết Matlab function/script để giải phương trình trên bằng phương pháp lặp đơn trong câu b. Output cần có cả nghiệm, số bước lặp và đánh giá sai số tuyệt đối. \\
d) Cần bao nhiêu bước lặp để sai số tuyệt đối nhỏ hơn $1e-6$. \\
e) Viết công thức đánh giá hậu nghiệm cho sai số và áp dụng cho phép lặp đơn trong câu b.
\end{bt}

\begin{bt}
a) Hãy viết script Matlab để tìm nghiệm của phương trình sau sử dụng phương pháp Newton $x^3 = x^2 + x + 1$. Giả sử ta đã sẵn có giá trị ban đầu $x_0$ đủ tốt. \\
b) Để phương pháp Newton hội tụ thì người ta cần đòi hỏi điều kiện gì của giá trị ban đầu $x_0$? \\
c) Nếu chưa biết $x_0$ thì ta cần làm gì để tìm $x_0$ phù hợp cho phương pháp Newton? 
\end{bt}

\centerline{———————————Hết——————————-}

\end{document}

\vspace{1cm}
\noindent{\bf Chú ý:} {\it Cán bộ coi thi không giải thích gì thêm}\\
\Closesolutionfile{ans}
\newpage
\begin{center}
{\LARGE{\bf ĐÁP ÁN}}
\end{center}

\begin{sol}
	\begin{figure}[h!]
		\centering
		\includegraphics[width=0.8\linewidth]{Solution1/Sol4_1.png}
		%\caption{}
		\label{fig:Sol4}
	\end{figure}
	Exercise 7: Convergence order is 3.	
\end{sol}

   
\end{document}



