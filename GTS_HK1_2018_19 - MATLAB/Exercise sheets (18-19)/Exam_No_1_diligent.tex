\documentclass[11pt]{article}
\usepackage{amsmath}
%\usepackage{extsizes}
\usepackage{amsmath,amssymb}
%\usepackage{omegavn,ocmrvn}
%\usepackage[utf8x]{inputenc}
\usepackage[utf8]{vietnam}

\usepackage{longtable}
\usepackage{answers}
\usepackage{graphicx}
\usepackage{array}
\usepackage{pifont}
\usepackage{picinpar}
\usepackage{enumerate}
\usepackage[top=3.0cm, bottom=3.5cm, left=3.5cm, right=2.5cm] {geometry}
\usepackage{hyperref}


\newtheorem{bt}{Câu}
\newcommand{\RR}{\mathbb R}
\Newassociation{sol}{Solution}{ans}
\newtheorem{ex}{Câu}
\renewcommand{\solutionstyle}[1]{\textbf{ #1}.}
\newcommand{\m}[1]{
	\begin{bmatrix}
		#1
	\end{bmatrix}
}

\begin{document}
% \noindent
\begin{tabular*}
{\linewidth}{c>{\centering\hspace{0pt}} p{.7\textwidth}}
Trường ĐHKHTN, ĐHQGHN & {\bf Học Kỳ 1 (2018-2019)}
\tabularnewline
K61 TTƯD & {\bf Bài kiểm tra thường xuyên \\ Đề I \& II}
\tabularnewline
\rule{1in}{1pt}  \small  & \rule{2in}{1pt} %(Due date:)
\tabularnewline

%  \tabularnewline
%  &(Đề thi có 1 trang)
\end{tabular*}
%
% \Opensolutionfile{ans}[ans1]

\begin{center}	
	\textbf{Đề I}
\end{center}

\begin{bt}
	a) Viết công thức Horner để tính toán giá trị của hàm số sau một cách hiệu quả
	%
	\[
	f(x)=x^4-5*x^3 + x^2 +8*x+9
	\]
	%
	b) Hãy sử dụng thuật toán Horner để thực hiện phép chia đa thức $f(x)$ cho $x-1$.\\	   	
	c) Sử dụng thuật toán Horner đầy đủ để tìm các hệ số trong khai triển Taylor của hàm số $f(x)$ tại $x=1$. 
\end{bt}

\begin{bt} 
a)	Chứng minh rằng phương trình $1+4x-10x^3=0$ có nghiệm duy nhất trong đoạn $[1/2, 1]$. \\
b)	Có rất nhiều các khác nhau để chuyển về bài toán tìm điểm bất động. Hãy tìm ít nhất 1 cách để phương pháp lặp đơn hội tụ. Giải thích vì sao? \\ 
c)  Bằng phương pháp lặp đơn vừa tìm được, hãy viết Matlab function lặp đơn và script trong Matlab sử dụng hàm đó để giải phương trình $1+4x-10x^3 = 0$.\\
d)  Tìm số bước lặp cần thiết sao cho sai số tuyệt đối của nghiệm bé hơn $1e-6$. \\
e)  Đưa ra công thức đánh giá ước lượng hậu nghiệm cho bài toán này. 
\end{bt}

\textbf{Chú ý:} Nếu không tìm được phương pháp lặp đơn nào hội tụ thì vẫn thực hiện các câu c)-e) như bình thường đối với phương pháp lặp đơn các em tự chọn. 
\begin{center}	
	\textbf{Đề II}
\end{center}

\begin{bt}
	a) Khảo sát số nghiệm và khoảng nghiệm của phương trình 
	%
	\[ f(x) := 32x^3 - 48x^2 +18x - 1 = 0.  \] 
	%
	b) Viết hàm phân đôi trong Matlab và script sử dụng hàm đó để tìm tất cả các nghiệm của phương trình trên với độ chính xác $\epsilon=1e-6$.	
	Trong quá trình tính toán, hãy sử dụng công thức Horner để tối ưu hóa nếu cần ước lượng giá trị của hàm $f$.\\
	c) Tính số bước lặp theo công thức trong lý thuyết để đạt được độ chính xác $\epsilon=1e-8$.
\end{bt}

\begin{bt} % Exercises 10,11,12, Atkinson/Han p.89
	a) Sử dụng phương pháp Newton, hãy viết Matlab function để tìm nghiệm phương trình $p(x):=a_3 x^3 + a_2 x^2 + a_1 x + a_0 = 0$ với điều kiện ban đầu $x_0$. Cũng có thể viết trực tiếp cho đa thức tổng quát bậc bất kỳ nếu muốn. Chú ý sử dụng phương pháp Horner để tính $p(x)$ và $p'(x)$. \\
	b) Viết script ứng dụng hàm Newton mới viết trong phần a) để giải phương trình 
	%
	\[ x^3 - (3+1e-13) x^2 + 3 x - 1 = 0, \]
	%
	trên khoảng $[0.8,1.2]$. Điều gì sẽ xảy ra nếu ta chọn điều kiện ban đầu $x_0=1$. \\ 	
    c) Phương pháp Newton hội tụ với điều kiện nào của $x_0$? Để đảm bảo điều kiện đó, thông thường người ta sẽ làm gì khi chưa biết $x_0$?\\	
	d) Hãy viết 1 script/function trong Matlab để thực hiện ý tưởng vừa nêu trong câu c) và thử nghiệm ngay vào ví dụ trong câu b).
\end{bt}


\centerline{———————————Hết——————————-}

%\end{document}

\vspace{1cm}
\noindent{\bf Chú ý:} {\it Cán bộ coi thi không giải thích gì thêm}\\
\Closesolutionfile{ans}
\newpage
\begin{center}
{\LARGE{\bf ĐÁP ÁN}}
\end{center}

\begin{sol}
Ta có thể xét các phép lặp đơn sau 
\begin{enumerate}
\item[i)] $\varphi(x) = 1+5x-10x^3$ (phân kỳ),
\item[ii)] $\varphi(x) = \cfrac{10x^3-1}{4}$ (phân kỳ),
\item[iii)] $\varphi(x) = \dfrac{1+4x}{10x^2}$ (hội tụ),
\item[iv)] $\varphi(x) = \sqrt{\dfrac{1+4x}{10x}}$ (hội tụ rất nhanh).
\end{enumerate}
%
The best solution would be iv), since the contraction constant on the interval $[0.7,0.8]$ is approximately $0.14$.
\end{sol}
   
\end{document}



