\documentclass[12pt]{article}

\usepackage{amsmath,amssymb,amsthm}
%\usepackage{omegavn,ocmrvn}
%\usepackage[utf8x]{inputenc}
\usepackage[utf8]{vietnam}

\usepackage{listings}
\lstset{language=Python}          % Set your language (you can change the language for each code-block optionally)


\usepackage{longtable}
\usepackage{answers}
\usepackage{graphicx}
\usepackage{array}
\usepackage{pifont}
\usepackage{picinpar}
\usepackage{enumerate}
\usepackage[top=3.0cm, bottom=3.5cm, left=3.5cm, right=2.5cm] {geometry}

\usepackage{extarrows}
\usepackage{hyperref}

\newtheorem{preremark}{Câu}
\newenvironment{bt}%
{\begin{preremark}\upshape}{\end{preremark}}


\newcommand{\RR}{\mathbb R}
\Newassociation{sol}{Solution}{ans}
\newtheorem{ex}{Câu}
\renewcommand{\solutionstyle}[1]{\textbf{ #1}.}

\def\hro{\mathbb}
\def\vphi{\varphi}
\def\tet{\theta}
\def\a{\alpha}
\def\b{\beta}
\def\rar{\rightarrow}
\def\R{\hro{R}}
\def\C{\hro{C}}
\def\Si{\Sigma}
\def\si{\sigma}
\def\ep{\varepsilon}
\def\rank{\mathrm{rank}}
\newcommand{\m}[1]{
	\begin{bmatrix}
		#1
	\end{bmatrix}
}


\begin{document}
% \noindent

\begin{tabular*}
	{\linewidth}{c>{\centering\hspace{0pt}} p{.5\textwidth}}
	ĐẠI HỌC QUỐC GIA HÀ NỘI	
	 & {ĐỀ THI KẾT THÚC HỌC PHẦN}  
	\tabularnewline
	TRƯỜNG ĐẠI HỌC KHOA HỌC TỰ NHIÊN & {HỌC KỲ II, NĂM HỌC 2022-2023}
	% Exercises on pages 239, 240 Cheney/Kincaid are really nice
	\tabularnewline
	\rule{3in}{1pt}  \small  & \rule{2in}{1pt} %(Due date:)
	\tabularnewline
	%  \tabularnewline
	%  &(Đề thi có 1 trang)
\end{tabular*}

\begin{center}
	{\bf Bài Thi Giữa Kỳ - Tính Toán Khoa Học. \\ Đề 1 - Thời gian 60 phút.
	 }
\end{center}

\begin{bt}\label{Câu 1}\textbf{(3 điểm)} % Bài 11 trang 15o sách Nguyễn Quang Hoàng
a) Sử dụng MATLAB (hoặc tính tay nếu muốn) để xác định các hệ số của đa thức bậc 3 có dạng 
%
\[
q(t) = a_0 + a_1 t + a_2 t^2 + a_3 t^3, \quad \forall 0\leq t \leq T 
\]
%
để chuyển động của tay máy một bậc tự do thỏa mãn các điều kiện sau
%
\[
T = 2, \ q(0) = 120^o, \ q(T) = 60^o, \ \dot{q}(0) = \dot{q}(T) = 0.
% T = 4, \ q(0) = -5^o, \ q(T) = 80^o, \ \dot{q}(0) = \dot{q}(T) = 0.
\]
%
b) Vẽ đồ thị của $q(t)$ vừa tìm được trên đoạn $[0,T]$ với bước $h = 0.01$. \\
Trên cùng hệ trục tọa độ đó, hãy vẽ đồ thị của hàm số 
%
\[
f(x) = x^3 - 3x^2 + 5xsin \left(\dfrac{\pi x - 5\pi}{4} \right) + 3 \ .
\]%
\end{bt}

\begin{bt}\textbf{(2 điểm)}
Tính các tích phân kép sau đây sử dụng hàm sẵn có trong MATLAB.\\
a) $A = \int_0^4 \int_0^{\pi} x^2 sin(y) dxdy$ 
\qquad b) $B = \int_0^2 \int_y^{3} x^2(x+y) dxdy$.
\end{bt}

\begin{bt}\label{Câu 1b}
Sử dụng phương pháp Heun, ode45 và ode15s để giải IVP sau.
\begin{equation}
(1+t) \ \dot{y} = t y + e^t (3t^2+2t+1), \quad y(0) = 1, \quad \forall 0\leq t\leq 2,
\end{equation}
với bước $h = 1e-2$. Vẽ các đồ thị sai số tuyệt đối so với nghiệm chính xác 
%
\[
y(t) = e^t (t^2+1) \ .
\]
%
\textbf{Các em có thể vẽ chung các sai số trên cùng 1 đồ thị nếu muốn.}
\end{bt}

\begin{center}
	--------------------------- Hết ---------------------------
\end{center}

\end{document}



