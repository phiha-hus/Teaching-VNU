\documentclass[11pt]{article}
\usepackage{amsmath}
%\usepackage{extsizes}
\usepackage{amsmath,amssymb}
%\usepackage{omegavn,ocmrvn}
%\usepackage[utf8x]{inputenc}
\usepackage[utf8]{vietnam}

\usepackage{listings}
\lstset{language=Python}          % Set your language (you can change the language for each code-block optionally)
\usepackage[most]{tcolorbox}

\usepackage{longtable}
\usepackage{answers}
\usepackage{graphicx}
\usepackage{array}
\usepackage{pifont}
\usepackage{picinpar}
\usepackage{enumerate}
\usepackage[top=3.0cm, bottom=3.5cm, left=3.5cm, right=2.5cm] {geometry}
\usepackage{hyperref}


\newtheorem{bt}{Câu}
\newcommand{\RR}{\mathbb R}
\Newassociation{sol}{Solution}{ans}
\newtheorem{ex}{Câu}
\renewcommand{\solutionstyle}[1]{\textbf{ #1}.}
\def\im{\mathrm{im}}

\begin{document}
% \noindent

\begin{tabular*}
	{\linewidth}{c>{\centering\hspace{0pt}} p{.7\textwidth}}
	Trường ĐHKHTN, ĐHQGHN & {\bf Học Kỳ 2 (2021-2022)}
	\tabularnewline
	K64 TTƯD - Thầy Hà Phi & {\bf Bài Tập Giải Tích Số \\ \today}
	% Exercises on pages 239, 240 Cheney/Kincaid are really nice
	\tabularnewline
	\rule{1in}{1pt}  \small  & \rule{2in}{1pt} %(Due date:)
	\tabularnewline
	%  \tabularnewline
	%  &(Đề thi có 1 trang)
\end{tabular*}




\begin{center}
	{\bf Bài Tập Lý Thuyết Điều Khiển Hệ Thống - No. 3 \\
	     Controllability of LTV systems}
\end{center}

\begin{bt}
Xét hệ thống tuyến tính có $D \equiv 0$ như sau
%
\begin{align}\label{eq1}
	\dot{x} &= A(t)x + B(t)u, \quad \forall t \geq t_0, \\
	y &= C(t)x \ . 
\end{align}
a) Trong một số ứng dụng thực tế, ví dụ như các hệ cơ học, nhiều khi người ta không quan tâm đến toàn bộ trạng thái $x(t)$ mà chỉ 1 phần của nó. Khi đó hệ được gọi là điều khiển được 1 phần (\textbf{partial controllable}). Hãy chứng minh kết quả sau.

\begin{tcolorbox}[colback=red!5!white,colframe=green!75!black]
Cho trước thời điểm $t_1 > t_0$. Khi đó tồn tại một hàm đầu vào $u$ để điều khiển đầu ra $y_t$ của hệ từ trạng thái ban đầu $y_0 = C(0)x_0$ tới trạng thái $y_1$ tại thời điểm $t_1$ khi và chỉ khi 
%
\begin{equation}\label{eq2}
	y_1 - C(t_1) \Phi(t_1,t_0)x_0 \in \im 
	C(t_1) \Phi(t_1,t_0) W_c(t_0,t_1),
\end{equation}
%
trong đó $\{\Phi(t,s)\}$ là nửa nhóm tiến hóa t.ứ với ptvp của $x$ còn $W_c(t_0,t_1)$ là ma trận điều khiển Kalman. 
\end{tcolorbox}
 
b) Chứng minh Định lý về tính điều khiển được cho hệ LTV thông qua kiểm tra điều kiện \textbf{ma trận Kalman đủ hạng dòng}, i.e., \\
%
\begin{equation}
 \rank \ K(t) = \m{M_0(t) & M_1(t) & \dots & M_{n-1}(t) } = n \ \mbox{ với $t>t_0$ nào đó}.
\end{equation}
%
% Hiệu ứng gương, phản chiếu chữ
% https://tex.stackexchange.com/questions/34713/mirror-effect-in-documents
\raisebox{\depth}{\scalebox{1}[-1]{
Hint: Nếu không làm được các em có thể xem Định lý 6.12 (trang 178 cuốn của Chen).
}}\\
c) (Open question) Em hãy thử mở rộng điều kiện hạng Kalman trong phần b) cho hệ điều khiển được 1 phần.
\end{bt}

\begin{bt}
Hãy xét tính điều khiển được của các hệ điều khiển sau	
\begin{equation}\label{1}
\dot{x} = \m{0 & 1 \\ 0 & t} x + \m{0 \\ 1} u, \quad y = \m{0 & 1} x \ .
\end{equation}
%
\begin{equation}\label{2}
	\dot{x} = \m{0 & 0 \\ 0 & -1} x + \m{1 \\ e^{-t}} u, \quad y = \m{0 & e^{-t}} x \ .
\end{equation}
%
\end{bt}
	
\begin{bt}
	Hãy xây dựng và chứng minh điều kiện cần và đủ cho tính chất điều khiển được (tương tự như định lý về Gramian điều khiển $W_c$) của hệ phương trình vi phân sau
	%
	\begin{equation}\label{Lyapunov diff. equ.}
		\dot{X}(t) = A(t)X(t) + X(t)B(t) + C(t)U(t)D(t), \ \forall t\geq t_0.
	\end{equation}
%
\begin{tcolorbox}[colback=yellow!5!white,colframe=blue!75!black]
Phương trình vi phân \eqref{Lyapunov diff. equ.} có tên gọi là phương trình vi phân Sylvester và có công thức nghiệm tường minh như sau.
%
\begin{equation}
	X(t) = \Phi_A(t,t_0) D \Phi_{B^T}(t,t_0)^T + \int_{t_0}^t \Phi_A(t,s) C(s) \Phi_{B^T}(t,s)^T ds,
\end{equation}
%
trong đó $\{\Phi_A(t,s)\}$ và $\{\Phi_{B^T}(t,s)\}$ là hai họ tiến hóa tương ứng với hai phương trình vi phân
$\dot{X} = A(t)X$ và $\dot{X} = B^T(t)X$. \\

Các em có thể tham khảo thêm Định lý 1.1.1-1.1.5 trong cuốn sách \\
\emph{Abou-Kandil, H., Freiling, G., Ionescu, V., Jank, G.: Matrix Riccati Equations in Control and Systems Theory. Birkhäuser, Basel (2003)}. 
\end{tcolorbox}
\end{bt}

\begin{bt} \textbf{Controllability of frozen system} \\ 
	Ta gọi hệ LTV \eqref{eq1} là bị đóng băng từ thời điểm $\si > t_0$ nào đó nếu lấy $A(t) \equiv A(\si)$ và $B(t) \equiv B(\si)$, với mọi $t \geq \si$.\\  	
	a) Giả sử rằng tại mọi thời điểm $\si > t_0$ hệ đóng băng 
	%
	\begin{align*}
		\dot{x} = A(\si) x(t) + B(\si) u(t), \ \forall t\geq \si,
	\end{align*}%
	là điều khiển được. Hỏi hệ LTV \eqref{eq1} có điều khiển được hay không? Vì sao? \\
	b) Cho trước thời điểm $\si \geq t_0$, và giả sử rằng hệ LTV \eqref{eq1} là điều khiển được từ thời điểm $\si$ đến thời điểm bất kỳ $t_1 > \si$. Hỏi hệ đóng băng tại $\si$ có điều khiển được hay không? Vì sao?
\end{bt}
	
\end{document}

\begin{bt}
	Mệnh đề sau có luôn luôn đúng không?
	%
	\[
	\rank \m{B & A B & \dots & A^{n-1}B} = \rank \m{AB & A^2 B & \dots & A^{n} B} \ .
	\]
	% 
	Nếu không, nó sẽ đúng với điều kiện nào?
\end{bt}

\begin{bt}
	Chuyển các hệ điều khiển LTI cấp $n$ sau về hệ điều khiển cấp 1 với biến điều khiển $u$, biến trạng thái $x$ và xét tính điều khiển được của hệ cấp 1 thu được. \\
	a) 
	%
	\begin{equation}
		x^{(n)} + \a_{n-1} x^{(n-1)} + \dots + \a_1 \dot{x} + \a_0 x = u(t) 
	\end{equation}
	%
	b*) 
	\begin{equation}
		x^{(n)} + \a_{n-1} x^{(n-1)} + \dots + \a_1 \dot{x} + \a_0 x = 
		u^{(n)} + \b_{n-1} u^{(n-1)} + \dots + \b_1 \dot{u} + \b_0 u \ . 
	\end{equation} 
\end{bt}

\begin{bt}
	Đối với hệ LTI, hãy chứng tỏ rằng $(A, B)$ là điều khiển được khi và chỉ khi $(-A, B)$ là điều khiển được. Điều này có đúng với các hệ thống LTV không?
\end{bt}   

\begin{bt}
	Hãy xét tính điều khiển được của các hệ điều khiển sau
	%
	\begin{equation}
		\dot{x} = \m{0 & 1 & 0 \\ 0 & 0 & 1 \\ 1 & -3 & -3} x + \m{0 \\ 0 \\ 1} u, \quad
		y = \m{1 & 2 & 1} x. 
	\end{equation}
\end{bt}
%
