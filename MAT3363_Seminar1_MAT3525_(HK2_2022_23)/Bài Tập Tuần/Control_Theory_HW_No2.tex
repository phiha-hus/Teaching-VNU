\documentclass[11pt]{article}
\usepackage{amsmath}
%\usepackage{extsizes}
\usepackage{amsmath,amssymb}
%\usepackage{omegavn,ocmrvn}
%\usepackage[utf8x]{inputenc}
\usepackage[utf8]{vietnam}

\usepackage{listings}
\lstset{language=Python}          % Set your language (you can change the language for each code-block optionally)


\usepackage{longtable}
\usepackage{answers}
\usepackage{graphicx}
\usepackage{array}
\usepackage{pifont}
\usepackage{picinpar}
\usepackage{enumerate}
\usepackage[top=3.0cm, bottom=3.5cm, left=3.5cm, right=2.5cm] {geometry}
\usepackage{hyperref}

\newtheorem{bt}{Câu}
\newcommand{\RR}{\mathbb R}
\Newassociation{sol}{Solution}{ans}
\newtheorem{ex}{Câu}
\renewcommand{\solutionstyle}[1]{\textbf{ #1}.}


\begin{document}
% \noindent

\begin{tabular*}
	{\linewidth}{c>{\centering\hspace{0pt}} p{.7\textwidth}}
	Trường ĐHKHTN, ĐHQGHN & {\bf Học Kỳ 2 (2021-2022)}
	\tabularnewline
	K64 TTƯD - Thầy Hà Phi & {\bf Bài Tập Giải Tích Số \\ \today}
	% Exercises on pages 239, 240 Cheney/Kincaid are really nice
	\tabularnewline
	\rule{1in}{1pt}  \small  & \rule{2in}{1pt} %(Due date:)
	\tabularnewline
	%  \tabularnewline
	%  &(Đề thi có 1 trang)
\end{tabular*}




\begin{center}
	{\bf Bài Tập Lý Thuyết Điều Khiển Hệ Thống - No. 2}
\end{center}

\begin{bt}
Cho trước thời điểm $t_1 > t_0 \geq 0$, một hệ thống tuyến tính 
\begin{align}
	\dot{x} &= A(t)x + B(t)u, \quad \forall t \geq t_0, \\
	y &= C(t)x + D(t)u
\end{align}
được gọi là \textbf{điều khiển được từ trạng thái $x_0 \in \R^n$ đến trạng thái $x_1 \in \R^n$} nếu tồn tại một hàm đầu vào $u(t)$ (khả tích, liên tục từng khúc) nào đó sao cho $x(t_1;t_0,x_0,u) = x_1$. Trong trường hợp $x_1 = 0$ ta nói \textbf{hệ là điều khiển được về $0$ (null-controllable)}. 
Nếu mọi trạng thái $x_1 \in \R^n$ đều điều khiển được từ $x_0$ thì ta nói hệ là điều khiển được toàn phần (từ $x_0$) (completely controllable). \\ 
\noindent  a) Chứng tỏ rằng đối với hệ tuyến tính thì tính chất điều khiển được về $0$ tương đương với điều khiển được toàn phần. \ Từ đó dẫn đến một tính chất quan trọng là điều khiển được đến một điểm $x_1$ cho trước sẽ tương đương với điều khiển được toàn phần. \\
b) Đối với hệ phi tuyến mệnh đề trong câu a) còn đúng nữa không? Vì sao? \\
c) Chứng minh rằng đối với hệ thống 
\begin{equation}
	\dot{x} = A(t) x + k(t,u),
\end{equation}
trong đó $k$ là một hàm liên tục, thì tính chất điều khiển được về $0$ cũng tương đương với điều khiển được toàn phần. 
\end{bt}

\begin{bt} \textbf{Bài tập về hệ dương (positive systems)} \\
Cho hệ điều khiển LTI
\begin{align}
	\dot{x} &= Ax + Bu, \quad \forall t \geq 0, \\
	y &= Cx + Du
\end{align}
Hệ trên được gọi là \textbf{dương trong (internally positive)} nếu với mọi điều kiện ban đầu $x(t_0) = x_0 \geq 0$, $u(t) \geq 0$ với mọi $t\geq 0$ thì cả trạng thái $x(t)$ và đầu ra $y(t)$ đều không âm. Ở đây ta hiểu một vector không âm ($\geq 0$) nghĩa là mọi tọa độ của nó đều không âm. 

\noindent a) Hãy tìm các điều kiện cần và đủ của 4 ma trận $A$, $B$, $C$, $D$ sao cho hệ đã cho là dương trong/dương ngoài.

\noindent b) Hãy tìm điều kiện cần và đủ của 4 ma trận $A$, $B$, $C$, $D$ sao cho hệ vừa là dương trong, vừa là ổn định (theo nghĩa hệ tự do là ổn định).
\end{bt}   

\begin{bt}
Cho hai hệ thống LTI được \textbf{mắc nối tiếp (cascade)} hoặc \textbf{mắc song song (parallel)}. \\
a) Hỏi nếu hai hệ thống đó là ổn định thì hệ thống tổng có ổn định không? \\
b) Hỏi nếu hai hệ thống đó là dương trong/dương ngoài thì hệ thống tổng có là dương trong/dương ngoài không? \\
c) Hỏi nếu hai hệ thống đó là không ổn định thì hệ thống tổng có thể ổn định hay không?
\end{bt}


\end{document}


\begin{bt} \textbf{(tính ổn định của hệ thống LTI):} Cho $A \in \R^{n,n}$. Chứng minh các khẳng định sau: \\
	a) PTVP $x'(t) = Ax(t)$ là ổn định tiệm cận khi và chỉ khi $\sigma(A) := \{ \lambda \in \C \ | \ \det(\lambda I-A) = 0\} \subset \C_{-}$ và $\sigma(A) \cap i\R = \emptyset$. \\
	b) PTVP $x'(t) = Ax(t)$ là ổn định khi và chỉ khi $\sigma(A) := \{ \lambda \in \C \ | \ \det(\lambda I-A) = 0\} \subset \C_{-}$ và các giá trị riêng thuần ảo là đơn (bội 1).
\end{bt}   

\begin{bt}
	a) Cho trước ma trận $A \in \R^{m,n}$. Chứng minh rằng $\|A\|_2 = \max\{\lambda \ | \ \det(\lambda I - A^TA) = 0 \}$. \\
	b) Cho ma trận $A \in \R^{n,n}$ khả nghịch, $B \in \R^{n,m}$, $C \in \R^{p,n}$, $D \in \R^{p,m}$. Chứng minh rằng $\m{A & B \\ C & D}$ khả nghịch khi và chỉ khi ma trận $M := D - CA^{-1}B$ là khả nghịch. \\
	c) Xét TH đặc biệt $p = m$ và $D \in \R^{m,m}$ là xác định âm. Chứng minh rằng ma trận $\m{A & B \\ B^T & D}$ là (nửa) xác định âm khi và chỉ khi $M$ là (nửa) xác định âm.
\end{bt}

\begin{bt} \textbf{Bài tập thực hành về tính ổn định - vẽ hình trong Matlab.}
Cho hệ \eqref{3} với các ma trận hệ số là 
%
\[
A = \m{1 & 1 \\ 4 & -2}, \ B = \m{1 & -1 \\ 1 & -1}, \ C = \m{1 & 0}, \ D = 0. 
\]
%
Hãy tìm hiểu các hàm \verb|eig| và \verb|ode45| trong MATLAB để xác định tính ổn định của hệ và vẽ hình 
đầu ra $y(t)$, trạng thái $x(t)$ với các dữ kiện sau:
%
\[
u(t) \mbox{ là hàm Heaviside (google nhé)}, \ x_0 = \m{1 \\ 1}, \ [t_0,t_f] = [0,10] \ .
\]
% 
\end{bt}
