\documentclass[11pt]{article}
\usepackage{amsmath}
%\usepackage{extsizes}
\usepackage{amsmath,amssymb}
%\usepackage{omegavn,ocmrvn}
%\usepackage[utf8x]{inputenc}
\usepackage[utf8]{vietnam}

\usepackage{listings}
\lstset{language=Python}          % Set your language (you can change the language for each code-block optionally)


\usepackage{longtable}
\usepackage{answers}
\usepackage{graphicx}
\usepackage{array}
\usepackage{pifont}
\usepackage{picinpar}
\usepackage{enumerate}
\usepackage[top=3.0cm, bottom=3.5cm, left=3.5cm, right=2.5cm] {geometry}
\usepackage{hyperref}


\newtheorem{bt}{Câu}
\newcommand{\RR}{\mathbb R}
\Newassociation{sol}{Solution}{ans}
\newtheorem{ex}{Câu}
\renewcommand{\solutionstyle}[1]{\textbf{ #1}.}


\begin{document}
% \noindent

\begin{tabular*}
	{\linewidth}{c>{\centering\hspace{0pt}} p{.7\textwidth}}
	Trường ĐHKHTN, ĐHQGHN & {\bf Học Kỳ 2 (2021-2022)}
	\tabularnewline
	K64 TTƯD - Thầy Hà Phi & {\bf Bài Tập Giải Tích Số \\ \today}
	% Exercises on pages 239, 240 Cheney/Kincaid are really nice
	\tabularnewline
	\rule{1in}{1pt}  \small  & \rule{2in}{1pt} %(Due date:)
	\tabularnewline
	%  \tabularnewline
	%  &(Đề thi có 1 trang)
\end{tabular*}




\begin{center}
	{\bf Bài Tập Lý Thuyết Điều Khiển Hệ Thống - No. 2}
\end{center}

\begin{bt}
a) Tính chất \textbf{điều khiển được về 0 tương đương với điều khiển được toàn phần} có đúng cho hệ điều khiển phi tuyến không? Vì sao? \\
b*) Chứng minh Định lý về tính điều khiển được cho hệ LTV thông qua kiểm tra điều kiện 
\textbf{ma trận Kalman đủ hạng dòng}, i.e., \\
%
\begin{equation}
 \rank \ K(t) = \m{M_0(t) & M_1(t) & \dots & M_{n-1}(t) } = n \ \mbox{ với $t>t_0$ nào đó}.
\end{equation}
%
\end{bt}

\begin{bt}
Mệnh đề sau có luôn luôn đúng không?
%
\[
\rank \m{B & A B & \dots & A^{n-1}B} = \rank \m{AB & A^2 B & \dots & A^{n} B} \ .
\]
% 
Nếu không, nó sẽ đúng với điều kiện nào?
\end{bt}

\begin{bt}
Chuyển các hệ điều khiển LTI cấp $n$ sau về hệ điều khiển cấp 1 với biến điều khiển $u$, biến trạng thái $x$ và xét tính điều khiển được của hệ cấp 1 thu được. \\
a) 
%
\begin{equation}
	x^{(n)} + \a_{n-1} x^{(n-1)} + \dots + \a_1 \dot{x} + \a_0 x = u(t) 
\end{equation}
%
b*) 
\begin{equation}
	x^{(n)} + \a_{n-1} x^{(n-1)} + \dots + \a_1 \dot{x} + \a_0 x = 
	u^{(n)} + \b_{n-1} u^{(n-1)} + \dots + \b_1 \dot{u} + \b_0 u \ . 
\end{equation} 
\end{bt}


\begin{bt}
Đối với hệ LTI, hãy chứng tỏ rằng $(A, B)$ là điều khiển được khi và chỉ khi $(-A, B)$ là điều khiển được. Điều này có đúng với các hệ thống LTV không?
\end{bt}   

\begin{bt}
Hãy xét tính điều khiển được của các hệ điều khiển sau
%
\begin{equation}
	\dot{x} = \m{0 & 1 & 0 \\ 0 & 0 & 1 \\ 1 & -3 & -3} x + \m{0 \\ 0 \\ 1} u, \quad
	y = \m{1 & 2 & 1} x. 
\end{equation}
%
%
\begin{equation}\label{1}
\dot{x} = \m{0 & 1 \\ 0 & t} x + \m{0 \\ 1} u, \quad y = \m{0 & 1} x \ .
\end{equation}
%
\begin{equation}\label{2}
	\dot{x} = \m{0 & 0 \\ 0 & -1} x + \m{1 \\ e^{-t}} u, \quad y = \m{0 & e^{-t}} x \ .
\end{equation}
%

\end{bt}
	
	
\end{document}



