\documentclass[11pt]{article}
\usepackage{amsmath}
%\usepackage{extsizes}
\usepackage{amsmath,amssymb}
%\usepackage{omegavn,ocmrvn}
%\usepackage[utf8x]{inputenc}
\usepackage[utf8]{vietnam}

\usepackage{listings}
\lstset{language=Python}          % Set your language (you can change the language for each code-block optionally)


\usepackage{longtable}
\usepackage{answers}
\usepackage{graphicx}
\usepackage{array}
\usepackage{pifont}
\usepackage{picinpar}
\usepackage{enumerate}
\usepackage[top=3.0cm, bottom=3.5cm, left=3.5cm, right=2.5cm] {geometry}

\usepackage{extarrows}
\usepackage{hyperref}


\newtheorem{bt}{Câu}
\newcommand{\RR}{\mathbb R}
\Newassociation{sol}{Solution}{ans}
\newtheorem{ex}{Câu}
\renewcommand{\solutionstyle}[1]{\textbf{ #1}.}


\begin{document}
% \noindent

\begin{tabular*}
	{\linewidth}{c>{\centering\hspace{0pt}} p{.7\textwidth}}
	Trường ĐHKHTN, ĐHQGHN & {\bf Học Kỳ 2 (2021-2022)}
	\tabularnewline
	K64 TTƯD - Thầy Hà Phi & {\bf Bài Tập Giải Tích Số \\ \today}
	% Exercises on pages 239, 240 Cheney/Kincaid are really nice
	\tabularnewline
	\rule{1in}{1pt}  \small  & \rule{2in}{1pt} %(Due date:)
	\tabularnewline
	%  \tabularnewline
	%  &(Đề thi có 1 trang)
\end{tabular*}




\begin{center}
	{\bf Bài Thi Giữa Kỳ - Tính Toán Khoa Học. \\
	 Đối với câu nào muốn sử dụng lập trình để hỗ trợ thì viết vào bài là "Nhờ lập trình ta có ...". Cuối buổi nộp cả bài lập trình, đặt tên file là Họ\_tên\_SV\_SBD.}
\end{center}

\begin{center}
	Đề 1 - Thời gian 80 phút.
\end{center}

\begin{bt}\label{Câu 1}
Cho hệ thống điều khiển
%
\begin{align}\label{eq0}
	\dot{x}(t) &= \m{1 & 2 \\ 3 & 4} x(t) + \m{1 & 2 \\ 3 & 2} u(t), \\
	y(t) &= \m{2 & 1} x(t) + \m{3 & 1} u(t). 
\end{align}
%
a) Hãy tìm hàm truyền của hệ và các không điểm, cực, lợi của hàm truyền. Tính gần đúng đến 4 chữ số thập phân. \\
b) Hãy sử dụng điều kiện phổ để kiểm tra xem trong trường hợp $u=0$ thì hệ có ổn định không. \\
c) Hãy tìm số nguyên $\a$ nhỏ nhất có thể sao cho với phản hồi có dạng $u(t) = K x(t)$, với $K = \a \cdot \m{ -0.5 & 0.5 \\ 0.75 & -0.25}$ thì hệ là ổn định. \\ 
d) Vẽ biểu đồ mô phỏng của hệ thống điều khiển trên.
\end{bt}

\begin{bt}
Cho hệ thống điều khiển với các tham số $\a$, $\b$ như sau
%
\begin{align}\label{eq1}
	\dot{x}(t) &= \m{0 & 1 & 0 \\ 1 & 0 & 0 \\ -2 & 1 & 2} x(t) + \m{1 \\ 2 \\ \a} u(t), \\
	y(t) &= \m{1 & \b & 1} x(t). 
\end{align}
%
a) Tìm ma trận điều khiển Kalman của hệ. \\
b) Tìm điều kiện của $\a$, $\b$ để hệ điều khiển \eqref{eq1} là điều khiển được. \\
\end{bt}

\begin{bt}\label{Câu 1b}
	Cho hệ thống điều khiển
	%
	\begin{align}
		\dot{x}(t) &= \m{1 & 2 \\ 3 & 4} x(t) + \m{1 & 2 \\ 3 & 2} u(t), \\
		y(t) &= \m{2 & 1} x(t) + \m{3 & 1} u(t). 
	\end{align}
	%
	a) Hãy tìm hàm truyền của hệ và các không điểm, cực, lợi của hàm truyền. Tính gần đúng đến 4 chữ số thập phân. \\
	b) Hãy sử dụng điều kiện phổ để kiểm tra xem trong trường hợp $u=0$ thì hệ có ổn định không. \\
	c) Hãy tìm số nguyên $\a$ nhỏ nhất có thể sao cho với phản hồi có dạng $u(t) = K x(t)$, với $K = \a \cdot \m{ -0.5 & 0.5 \\ 0.75 & -0.25}$ thì hệ là ổn định. \\ 
	d) Vẽ biểu đồ mô phỏng của hệ thống điều khiển trên.
\end{bt}

\begin{center}
	--------------------------- Hết ---------------------------
\end{center}

\pagebreak 

\begin{tabular*}
	{\linewidth}{c>{\centering\hspace{0pt}} p{.7\textwidth}}
	Trường ĐHKHTN, ĐHQGHN & {\bf Học Kỳ 2 (2021-2022)}
	\tabularnewline
	K64 TTƯD - Thầy Hà Phi & {\bf Bài Tập Giải Tích Số \\ \today}
	% Exercises on pages 239, 240 Cheney/Kincaid are really nice
	\tabularnewline
	\rule{1in}{1pt}  \small  & \rule{2in}{1pt} %(Due date:)
	\tabularnewline
	%  \tabularnewline
	%  &(Đề thi có 1 trang)
\end{tabular*}




\begin{center}
	{\bf Bài Thi Giữa Kỳ - Tính Toán Khoa Học. \\
		Đối với câu nào muốn sử dụng lập trình để hỗ trợ thì viết vào bài là "Nhờ lập trình ta có ...". Cuối buổi nộp cả bài lập trình, đặt tên file là Họ\_tên\_SV\_SBD.}
\end{center}

\begin{center}
	Đề 2 - Thời gian 80 phút.
\end{center}

\begin{bt}
Cho hệ điều khiển
%
\begin{align}\label{eq3}
	\dot{x} &= \m{0 & 1 & 0 \\ 0 & 0 & 1 \\ -\a_1 & -\a_2 & -\a_{3} } x + \m{1 \\ 1 \\ 1} u, \\
	y &= \m{1 & 2 & 3} x, 
\end{align}
%
trong đó $\a := \m{\a_1 & \a_2 & \a_3 }$ là vector hệ số thực. \\
a) Hãy tính đa thức đặc trưng $P(s) := \det(sI_3-A)$. \\
b) Cho $\a_1 \in [1,2]$, $\a_2 \in [2,3]$, $\a_3 \in [3,4]$. Hãy sử dụng định lí Kharitonov để kiểm tra xem hệ có ổn định hay không? \\
c) Với $\a = \m{1 & 2 & 3}$, hãy kiểm tra xem phương trình Lyapunov $A X + X A^T = -B B^T$ có nghiệm $X$ đối xứng, xác định dương hay không? \\  
d) Từ đó kết luận xem hệ có điều khiển được hay không?
\end{bt}   

\begin{bt}
	Cho hệ thống điều khiển với các tham số $\a$, $\b$ như sau
	\begin{align}\label{eq2}
		\dot{x}(t) &= \m{0 & 1 & 0 \\ 1 & 0 & 0 \\ -2 & 1 & 2} x(t) + \m{1 \\ 2 \\ \a} u(t), \\
		y(t) &= \m{1 & \b & 1} x(t). 
	\end{align}
	%
	a) Hãy tìm ma trận quan sát Kalman của hệ. \\
	b) Hãy tìm điều kiện của $\a$, $\b$ để hệ điều khiển \eqref{eq2} là quan sát được. \\ 
\end{bt}

\begin{bt}
	Cho hệ điều khiển
	%
	\begin{align}\label{eq4}
		\dot{x} &= \m{-\a_1 & 1 & 0 \\ -\a_2 & 0 & 1 \\ -\a_{3} & 0 & 0 } x + \m{1 \\ 2 \\ 3} u, \\
		y &= \m{1 & 0 & 0} x, 
	\end{align}
	%
	trong đó $\a := \m{\a_1 & \a_2 & \a_3 }$ là vector hệ số thực. \\
	a) Hãy tính đa thức đặc trưng $P(s) := \det(sI_3-A)$. \\
	b) Cho $\a_1 \in [1,2]$, $\a_2 \in [2,3]$, $\a_3 \in [3,4]$. Hãy sử dụng định lí Kharitonov để kiểm tra xem hệ có ổn định hay không? \\
	c) Với $\a = \m{1 & 2 & 3}$, hãy kiểm tra xem phương trình Lyapunov $XA +  A^T X = - C^T C$ có nghiệm $X$ đối xứng, xác định dương hay không?  \\
	d) Từ đó kết luận xem hệ có quan sát được hay không?
\end{bt}   

\begin{center}
	--------------------------- Hết ---------------------------
\end{center}

\end{document}



