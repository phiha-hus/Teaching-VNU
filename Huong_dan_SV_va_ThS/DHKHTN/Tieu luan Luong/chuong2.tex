\section{Các tính chất của hàm Lambert}
	Chúng ta sẽ bắt đầu bằng việc xác định miền hội tụ của nhánh chính $W_{0}(x)$ được biểu diễn dưới dạng chuỗi lũy thừa theo định lý Larange như sau \\
	Giả sử z được định nghĩa là hàm của x theo phương trình có dạng $z = f(x)$\\
trong đó $f$ là giải tích tại $x_0$ và $f'(x_{0}) \neq 0$. Từ đó ta có thể đảo ngược hay giải phương trình với $x = g(z)$ bằng chuỗi lũy thừa. 
\begin{align}
g(z) &= x_0 + \displaystyle \sum_{n=1}^{\infty}g_n \frac{(z-f(x_0))^{n}}{n!}) \notag \\
\mbox{với }g_n &= \displaystyle \lim_{x \to x_0} \left[\frac{d^{n-1}}{dx^{n-1}}\left(\frac{x-x_0}{f(x)-f(x_0)}\right)^{n}\right]
\end{align}
Áp dụng định lý Larange, ta có khai triển chuỗi của nhánh $W_0(z)$ là $W_0(z) = \displaystyle \sum_{n=1}^{\infty}\frac{(-n)^{n-1}}{n!}z^{n}$\\
Thật vậy, đặt $f(x) = xe^{x}$ ,chọn $x_0 = 0 \mbox{ do đó } f(x_0) = 0$. Khi đó $f'(x) = e^{x} + xe^{x} \mbox{ do đó } f'(x_0) = 1$.
\begin{align*}
W_{0}(z) &= \displaystyle \sum_{n=1}^{\infty}g_n\frac{z^{n}}{n!} \\
\mbox{với }g_n &= \displaystyle \lim_{x \rightarrow 0}\frac{d^{n-1}}{dx^{n-1}}\frac{x^{n}}{f^{n}x}
=  \displaystyle \lim_{x \rightarrow 0}\frac{d^{n-1}}{dx^{n-1}}(\frac{x}{xe^{x}})^{n} 
= \displaystyle \lim_{x \rightarrow 0}\frac{d^{n-1}}{dx^{n-1}}e^{-nx} = \displaystyle \lim_{x \rightarrow 0}(-n)^{n-1}e^{-nx} \\
&= (-n)^{n-1}
\end{align*}

Vì vậy
\begin{align*}
W_0(z) = \displaystyle \sum_{n=1}^{\infty}\frac{(-n)^{n-1}}{n!}z^{n}
\end{align*}
Theo tiêu chuẩn D' Alembert thì chuỗi lũy thừa $\displaystyle \sum_{n=0}^{\infty}(a_nz^{n})$ có bán kính hội tụ là  \\
r = $\displaystyle \lim_{x \to \infty}\frac{|a_n|}{|a_{n+1}|}$ \\
Áp dụng tiêu chuẩn D'Alembert vào chuỗi $W_0(z) = \displaystyle \sum_{n=1}^{\infty}\frac{(-n)^{n-1}}{n!}z^{n}$ \\
Đặt $a_n =\frac{n^{n-1}}{n!} \mbox{ ta có } \frac{a_n}{a_{n+1}} = \frac{n^{n-1}}{n!} \Bigg/ \frac{(n+1)^{n}}{(n+1)!} \\
= \frac{1}{(\frac{n+1}{n})^{n-1}} = \frac{1}{(1+\frac{1}{n})^{n-1}} \xrightarrow{n \rightarrow \infty} \frac{1}{e}\\
\mbox{Do đó }r = \displaystyle \lim_{n \to \infty}\frac{|a_n|}{|a_{n+1}|} = \frac{1}{e}$.
Vậy bán kính hội tụ của chuỗi $W_0(z)$ là $\frac{1}{e}$.
Nói về sự hội tụ này , Euler đã không chỉ ra rằng chuỗi (1.2) hội tụ với 
\begin{align*}
	M_E(\alpha,\beta)|v| < \frac{1}{e}
\end{align*}
trong đó $M_E(a,b) = \sqrt{\frac{a^2+b^2}{2}}$ \\
Xét hàm Lambert $W(x)$ thỏa mãn $x = W(x)e^{W(x)}$, ta đi tìm $W'(x)$:
Tại mọi điểm mà $W$ là khả vi, ta sẽ có 
\begin{align}
\frac{d}{dx}W(x)e^{W(x)} &= \frac{dx}{dx} &= 1 \nonumber \\  
\mbox{Do đó }W'(x)e^{W(x)} + W(x)W'(x)e^{W(x)} = W'(x)e^{W(x)}(1+W(x)) \tag{*}\\
\mbox{Vì vậy } W'(x) = \frac{1}{e^{W(x)}(1+W(x))} = \frac{W(x)}{x(1+W(x))} ,\mbox{ với mọi } x \neq 0, x \neq \frac{-1}{e} \nonumber
\end{align}
Ngoài ra, nếu ta xét nhánh chính $W_{0}$ và $x=0$ thì từ $(*)$ ta có $1 = W'(0)e^{W(0)}(1+W(0)) = W'(0)e^{0}(1+0)$ \\
Do đó $W'_{0}(0) = 1$.
Nói thêm về đạo hàm, chúng ta sẽ dùng phương pháp quy nạp để tính đạo hàm cấp $n$ của $W$ với $n \ge 1$ là 
$$ \frac{d^{n}W(x)}{dx^n} = \frac{e^{-nW(x)}p_n(W(x))}{(1+W(x))^{2n-1}},$$ với $n \ge 1 $ \\
trong đó đa thức $p_n(w)$ thỏa mãn ràng buộc :\\
$$ p_{n+1}(w) = -(nw + 3n - 1)p_n(w) + (1+w)p'_n(w), $$ \\
Về công thức chi tiết và các tính toán của $p_n$ và $p'_n$, xem tài liệu $[7,8]$ \\
Ta có công thức sau đối với nhánh chính khi $z = We^{W} > 0 $
$$ \log W(z) = \log z - W(z) $$ xem $[9]$ để biết thêm công thức tổng quát.
Bây giờ, chúng ta sẽ trả lời câu hỏi về việc tính tích phân các biểu thức chứa hàm Lambert W. Trong [10], K. B. Ranger sử dụng bài tập để minh họa việc tính tích phân của phương trình Navier-Stokes trong dạng tham số 
\begin{align}
 x = pe^{p},\\
\frac{dy}{dx} = p. \\
\Rightarrow \frac{dx}{dy} = \frac{d(pe^{p})}{dy} = \frac{dp}{dy}pe^{p} + \frac{dp}{dy}e^{p} = e^{p}\frac{dp}{dy}(1+p) = \frac{1}{p} \nonumber \\
\Rightarrow \frac{dy}{dp} = p(p+1)e^{p} 
\end{align}
Từ phương trình (1.9)ta có thể dễ dàng lấy tích phân để thu được : \\
$$ y = (p^{2} - p + 1)e^{p}  +  C $$\\
Vì $y$ là nguyên hàm của $W(x)$, Ranger đã phát hiện ra một phương pháp đơn giản để tính tích phân của $W(x)$ như sau 
$$ \displaystyle \int(W(x)dx) = (W^2(x) - W(x) +1)e^{W(x)} + C  \\
= x(W(x) - 1 + \frac{1}{W(x)}) + C $$
Khi ta thử kỹ thuật này trên các hàm khác có chứa W, ta thấy rằng đó là sự thay đổi đặc biệt của biến. Đặt $w = W(x)$, do đó $x = we^w$ và $dx = (w+1)e^{w}dw$
\begin{align*}
\displaystyle \int{xW(x)dx} &= \displaystyle \int{we^{w}.w.(1+w)e^{w}dw} \\
&= \frac{1}{8}(2w-1)(2w^{2}+1)e^{2w} + C  \\
&= \frac{1}{2}(W(x)-\frac{1}{2})(W^{2}(x)+\frac{1}{2})e^{2W(x)} + C
\end{align*}
Điều này đúng cho mọi nhánh của W, theo định nghĩa $\frac{d}{dw}we^{w} \neq 0$ tại điểm trong bất kỳ thuộc nhánh bất kỳ. \\
Bài toán tích phân của các biểu thức chứa $W$ là trường hợp đặc biệt của phép lấy tích phân các biểu thức của hàm ngược. Bằng việc sử dụng kỹ thuật đã nêu trên thì chúng ta có công thức sau, xem thêm tại $[10]$
\begin{align*}
\displaystyle \int{f^{-1}(x)dx} = yf(y) -\displaystyle \int{f(y)dy}
\end{align*}
Thật vậy, đặt $x = f(y)$ \\
$\mbox{Do đó } \displaystyle \int{f^{-1}(x)dx} = \displaystyle \int{ydf(y)} = yf(y) - \displaystyle \int{f(y)dy} + C $ \\
Cuối cùng, lưu ý rằng kỹ thuật này cho phép thuật toán Risch được áp dụng để xác định tích phân có chứa $W$ có phải là sơ cấp hay không. Xem thêm thuật toán Risch tại $[11]$
\section{Một số ứng dụng của hàm Lambert}
\subsection{Bài toán về mô hình đốt cháy}
 Bài toán 
$$ \frac{dy}{dt} = y^{2}(1-y),\quad y(0) = \varepsilon > 0 $$ được sử dụng trong $[12,17]$ để nghiên cứu phương pháp nhiễu. Ta chỉ công thức nghiệm hiển theo hàm $W$, và do đó mọi kết quả nhiễu trong $[12]$ có thể được kiểm chứng lại bằng cách so sánh với nghiệm chính xác. Bài toán mô hình là phương trình phân ly biến số và tích phân sẽ cho kết quả là dạng ẩn của $y(t)$ như sau
\begin{align*} \frac{1}{y} + \log (\frac{1}{y} -1 ) &= \frac{1}{\varepsilon} + \log (\frac{1}{\varepsilon}  - 1) - t  \\
\mbox{Do đó } e^{\frac{1}{y}}(\frac{1}{y} - 1) &= e^{\frac{1}{\varepsilon}-t}(\frac{1}{\varepsilon}-1)  \\
\mbox{Vì vậy } e^{\frac{1}{y}-1}(\frac{1}{y}-1) &= e^{\frac{1}{\varepsilon}-1-t}(\frac{1}{\varepsilon}-1) 
\end{align*}
Đặt $\frac{1}{\varepsilon} - 1 = u \mbox{ ta có } \frac{1}{y} - 1 = W(ue^{u-y}) \mbox{ do đó } y = \frac{1}{1+W(ue^{u-t})}$. Qua việc định tính phương trình vi phân,ta có thể chỉ ra rằng $0 < \varepsilon \le y < 1$, điều này chỉ ra rằng nhánh chính của hàm Lambert W cần được sử dụng. Bắt nguồn từ nghiệm hiển, tất cả các dạng chuỗi kết quả của hàm trong $[12]$ có thể dễ dàng được kiểm tra.
\subsection{Phương trình vi phân hệ số hằng}
Ví dụ cho dạng đơn giản của phương trình vi phân có trễ :
$$ \dot{y}(t) = ay(t-1), $$ với $y(t) = f(t)$ là 1 hàm đã biết , trên đoạn $0 \le t \le 1$. Ý nghĩa của bài toán bắt nguồn từ việc nghiên cứu sự ổn định của phương trình vi phân phi tuyến có trễ\\
Một cách tiếp cận bài toán là phỏng đoán y = $e^{st}$ là nghiệm với vài giá trị của s. Điều đó cho ta
$$ se^{st} = ae^{st}e^{-s} \\ \Rightarrow se^{s} = a $$ hay $$ s =  W_k(a) , $$  với nhánh bất kỳ $W_k$. 
Nếu $e^{W_k(a)t}$ là nghiệm của $ \dot{y} = ay(t-1) $ thì theo nguyên lý chồng chất nghiệm
$$ y = \displaystyle \sum_{k=-\infty}^{\infty}{c_ke^{W_k(a)t}}$$ cũng là nghiệm với mọi cách chọn $c_k$ "chấp nhận được".  Người ta thấy rằng nghiệm sẽ tăng theo cấp số mũ nếu một $W_k(a)$ nào đó có phần thực dương, và điều đó dẫn đến những định lý quan trọng trong lý thuyết của phương trình vi phân có trễ.\\
Cách tiếp cận này có thể được tổng quát hóa để nghiên cứu các phương trình vi phân có trễ hệ số hằng có dạng vô hướng
$$ \dot{y}(t) = ay(t-1) + by(t), $$ 
hoặc $ \dot{y} = Ay(t-1) + By(t)$ với $A, B$ là các ma trận giao hoán. Xem thêm $[18,19,20]$ về vấn đề này.
 
