\chapter{Hàm Lambert và phổ của phương trình vi phân có trễ}
\section{Giới thiệu}
Trong chương này, chúng ta sẽ xem xét các phương trình vi phân có trễ dạng sau :
\begin{align}
\sum = \begin{cases}
\dot{x}(t) &= Ax(t) + Bx(t-\tau), t > 0\\
x(t) &= \varphi(t) , t \in [-\varphi,0] \\
\end{cases} 
\end{align}
với $A, B  \in \mathbb{C}^{n \times n}$, trễ $\tau$ > 0. Phương trình đặc trưng tương ứng là
$$ 0 = \det (-sI + A + Be^{-s\tau}) \\$$
Ở đây, ta sẽ kí hiệu tập hợp tất cả các nghiệm  của phương trình đặc trưng(tập các giá trị riêng hay phổ) của $\sum$ bởi $\sigma_{\sum}$. \\
Không giống như phương trình vi phân không trễ, phương trình vi phân có trễ sẽ có vô hạn nhưng đếm được số giá trị riêng. Đó là một điểm khó khăn của phương trình vi phân có trễ từ quan điểm tính toán. Trong tài liệu, có một số kết quả về tính chất định tính của phổ. Ví dụ, chúng ta biết từ $[21]$ rằng giá trị riêng sẽ phân bố dọc theo các đường cong trong mặt phẳng phức (được gọi là xích các nghiệm) và với bất kỳ đường thẳng nào trong mặt phẳng phức thì chỉ có một số hữu hạn các giá trị riêng nằm bên phải của đường thẳng này.\\
Trong chương này, chúng ta sẽ nghiên cứu biểu diễn hiển của các giá trị riêng của lớp đặc biệt các phương trình vi phân có trễ dựa trên dạng ma trận của hàm Lambert $W$. Phổ của các hệ một trễ vô hướng có thể được tính toán sử dụng hàm Lambert, điều này đã được biết đến từ lâu, ví dụ như trong $[22]$. Trong $[23]$ và các nghiên cứu tiếp theo $[24]$ và $[25]$ thì những mở rộng cho trường hợp nhiều chiều sử dụng dạng ma trận của hàm Lambert $W$. Tuy nhiên các kết quả trong $[23]$ không đúng trong trường hợp tổng quát. Mục tiêu của chương này là đưa ra các điều kiện đủ cho các ma trận của hệ để công thức trong $[23]$ đúng. Cá biệt thì chúng ta sẽ chỉ ra rằng công thức là đúng nếu $A$ và $B$ là tam giác hóa được đồng thời. Những quan sát tương tự đã được nêu ra trong $[26]$ và hoàn toàn độc lập với nghiên cứu này, trong đó các kết quả gần tương tự nhận được không sử dụng công thức hiển cho hàm Lambert $W$. Ở đây chúng ta sẽ thiết lập các kết quả này cho các biểu diễn trong $[23]$. Hơn nữa, chúng ta sẽ trình bày các phản ví dụ, điều này chứng minh rằng công thức có thể sai. Điều này rất quan trọng vì kết quả của $[23]$ rất được quan điểm và được trích dẫn. Do đó đáng để làm rõ phạm vi ứng dụng của công thức.
\section{Dạng ma trận của hàm Lambert}
Nhắc lại rằng với z $\in \mathbb{C}$, hàm Lambert W được định nghĩa (đa trị) là hàm ngược của hàm $z \mapsto ze^{z}$ :
$$ W_k(z) \in \left\{w \in \mathbb{C} : z = we^{w}\right\}, $$ \\
trong đó $W_{k}$ là nhánh chính thứ $k, k \in \mathbb{Z}$ \\
Ngoài điểm $ z = -e^{-1}$ mà nhánh chính $W_0$ không khả vi, tại đó tất cả các nhánh đều là giải tích địa phương. Do đó, chúng ta định nghĩa hàm Lambert $W$ một cách tiêu chuẩn ví dụ như trong $[27]$ và $[28]$. Đầu tiên chúng ta định nghĩa hàm Lambert $W$ cho ma trận theo dạng chuẩn tắc Jordan
$$ J = diag (J_{n_{1}}(\lambda_{1}),J_{n_{2}}(\lambda_{2}),...,J_{n_{s}}(\lambda_{s})) $$
với $J_n(\lambda)$ là ma trận Jordan cỡ n $\times$ n tương ứng với giá trị riêng  $\lambda$ bội n.
Vậy :
$$ W_k(J) = diag (W_{k_{1}}(J_{n_{1}}(\lambda_{1})),W_{k_{2}}(J_{n_{2}}(\lambda_{2})),...,W_{k_{s}}(J_{n_{s}}(\lambda_{s}))).$$ \\
Chú ý rằng chúng ta được phép lấy các nhánh khác nhau cho mỗi khối Jordan. Nếu J có $s$ khối Jordan  và tập chỉ số cho các nhánh của hàm Lambert W là $\mathbb{Z}$ thì tập chỉ số của các nhánh $W_k(j)$ là $ \mathbb{Z}^{s} $  . \\
Với khối Jordan có số chiều là 1 thì ta có thể sử dụng hàm Lambert $W$ vô hướng . Với khối số chiều lớn hơn, ta sẽ định nghĩa hàm Lambert $W$ (với nhánh cố định) của một khối Jordan bởi định nghĩa tiêu chuẩn của hàm ma trận dạng
$$ P(J_k(\lambda)) = \left(
\begin{array}{cccc}
p(\lambda) & p'(\lambda) & \cdots & \frac{1}{(k-1)!}p^{(k-1)}(\lambda) \\
 & p(\lambda) & \ddots & \vdots \\
\vdots &  &\ddots & p'(\lambda) \\
 &  & \cdots & p(\lambda) \\
\end{array}
\right)
 \\ $$
$$ \Rightarrow W_k(J_n(\lambda)) =  \left(
\begin{array}{cccc}
W_k(\lambda) & W_k'(\lambda) & \cdots & \frac{1}{(n-1)!}W_{k}^{(n-1)}(\lambda) \\
 & W_k(\lambda) & \ddots & \vdots \\
\vdots &  &\ddots & W'_k(\lambda) \\
 &  & \cdots & W_k(\lambda) \\
\end{array}
\right) \\ $$
Nếu k = 0 thì ta phải giả sử thêm là $\lambda \neq -e^{-1}$(vì $W'_{0}(-e^{-1})$ không xác định). \\
Ta hoàn thành việc định nghĩa hàm Lambert W cho ma trận, bằng việc chuyển đổi ma trận về dạng chuẩn Jordan $A = SJS^{-1}$. Từ đó ta có thể định nghĩa
$$ W_k(A) = S W_k(J) S^{-1} $$
Với nhánh chính k = 0, từ bây giờ chúng ta sẽ giả sử rằng $-e^{-1}$ không phải là giá trị riêng tương ứng với khối Jordan với số chiều lớn hơn 1, tức là
\begin{align} rank (A + e^{-1}I) = rank (A + e^{-1}I)^{2} \end{align}

\begin{note} 
Giả thiết (3.2) làm giảm đi sự đẹp đẽ của hàm Lambert W. Điểm này được chỉ ra bởi Robert Corless.
\end{note}

\begin{vd}
Chúng ta sẽ minh họa định nghĩa hàm Lambert $W$ cho khối Jordan 2 $\times$ 2 với $\lambda \neq -e^{-1} : \\ $ 
\end{vd}

Cho
\begin{align*}
 J = \begin{bmatrix} 
\lambda & 1 \\
0 & \lambda
 \end{bmatrix} 
\end{align*}
thì 
\begin{align*}
 W_k(J) = \begin{bmatrix} 
W_k(\lambda) & W'_k(\lambda) \\
0 & W_k(\lambda) \end{bmatrix} 
\end{align*}
Chúng ta kiểm tra rằng $J = W_k(J)e^{W_k(J)}$ như sau. Vì $\lambda = W_k(\lambda)e^{W_k(\lambda)}$ ta có $1 = W'_k(\lambda)e^{W_k(\lambda )} + W'_k(\lambda)\lambda$
\begin{align*}
\mbox{Do đó } W_k(J)e^{W_k(J)} = \begin{bmatrix} 
W_k(\lambda) & W'_k(\lambda) \\
0 & W_k(\lambda) 
\end{bmatrix}
e^{W_k(\lambda)} \begin{bmatrix} 
1 & W'_k(\lambda) \\
0 & 1 
\end{bmatrix} \\
= \begin{bmatrix} 
\lambda & \lambda W'_k(\lambda) + e^{W_k(\lambda)} W'_k(\lambda)\\
0 & \lambda 
\end{bmatrix} 
 = J \\
\end{align*}
Mặt khác, nếu $\lambda = -e^{-1}$ thì $W_0(\lambda) = -1$ và giả thiết $W = \begin{bmatrix} 
-1 & w \\
0 & -1
\end{bmatrix}$ thì ta có $We^{W} = -e^{-1}I \mbox{ với mọi }  w \in \mathbb{C}$.Do đó, $\sigma(W) =  \left\{-1\right\}$ dẫn đến $We^{W} = -e^{-1}I$. Chúng ta kết luận rằng $W_0(J)$ không xác định trong trường hợp này vì $W_0(-e^{-1})$ không phải là duy nhất.

\section{Kết quả chính}
Sau đây chúng ta sẽ đi nghiên cứu công thức phổ cho các hệ phương trình vi phân có trễ trong một số trường hợp đặc biệt

 \begin{bd}
 Nếu $M = [m_{ij}]_{i,j = 1,...,n} \in \mathbb{R}^{n \times n}$ có dạng tam giác dưới hoặc tam giác trên thì dạng chuẩn tắc Jordan của M là $J_{M}$ sẽ chứa chính xác $m_{11}, ..., m_{nn}$ trên đường chéo chính (theo một hoán vị nào đó) \\
 \end{bd}

 Từ Bổ đề 1 : 
 \begin{align*}
 \mbox{Theo định nghĩa} \hspace{0.2cm} W_{k}(M) = SW_{k}(J_{M})S^{-1} \\
 \mbox{Do đó} \sigma(W_{k}(M)) =  \sigma(W_{k}(J_{M})) \\
 = \bigcup_{i = 1,...,n} W_{k}(m_{ii})\\
 \end{align*}
 Từ đó ta có hệ quả sau
 \begin{hq}
 Nếu $M \in \mathbb{R}^{n \times n}$ là 1 ma trận tam giác dưới (hoặc tam giác trên) thì $\hspace{0.2cm} \sigma(W_{k}(M)) = \displaystyle \bigcup_{i = 1,...,n} W_{k}(m_{ii}),\\ $
 Từ Bổ đề 1, vì tổng và tích của 2 ma trận là tam giác dưới (hoặc tam giác trên) cũng là tam giác dưới (hoặc tam giác trên). Ta có hệ quả sau  \\
 Nếu A, B cùng tam giác dưới hoặc tam giác trên thì 
 \begin{align}
 \sigma_{\sum} = \displaystyle \bigcup_{k}\sigma(\frac{1}{\tau}W_{k}(B\tau e^{-A\tau})+A) = \displaystyle \bigcup_{j=\overline{1,n}}\frac{1}{\tau}W_{k}(b_{jj}\tau e^{\tau a_{jj}}) + a_{jj}
 \end{align}
 \end{hq}
 
 \begin{note} Theo $[29]$
  Cặp A, B là tam giác hóa được khi và chỉ khi [A,B] = AB - BA là lũy linh (tức là tồn tại $n \in \mathbb{N}$ để $[A,B]^{n} = 0$)  \\
 \end{note}	
 \begin{hq}
 Nếu A = 0 thì $\sigma_{\sum} = \displaystyle \bigcup_{k}\sigma(\frac{1}{\tau}W_{k}(\tau B)) = \frac{1}{\tau}\displaystyle \bigcup_{k}W_{k}(\tau \sigma(B))$
 \end{hq}
 \begin{dn}
 Cặp ma trận (A,B) (với $A \in \mathbb{R}^{n \times n}, B \in \mathbb{R}^{n \times p}$) được gọi là \emph{điều khiển được} nếu thỏa mãn 1 trong các điều kiện tương đương sau.
 \begin{enumerate}
\item[1] Ma trận $K = \m{B & AB & ... & A^{n-1}B}$ có hạng = n.
\item[2] Tồn tại $\lambda \in \mathbb{R}$ sao cho rank 
$\m{\lambda I_{n}-A & B} = n$.
 \end{enumerate}
 \end{dn}
 \begin{vd}
 $A = \begin{bmatrix}
 		1 & 0 \\
 		0 & 0
 	  \end{bmatrix}$ ,
 $B_{1} = \begin{bmatrix}
 			1 \\
 			0
 		  \end{bmatrix}$ ,
 $B_{2}= \begin{bmatrix}
 			1 \\
 			1
         \end{bmatrix}$ \\
 \end{vd}
 $(A,B_{1})$ không điều khiển được vì $K = \begin{bmatrix}
 											1 & 1\\
 											0 & 0
 										   \end{bmatrix}$
 có hạng bằng 1 < 2 \\
 $(A,B_{2})$ không điều khiển được vì $K = \begin{bmatrix}
 											1 & 0\\
 											1 & 1
 										   \end{bmatrix}$
 có hạng bằng 2 \\
 hoặc $\m{\lambda I-A   & B} = 
\m{ \lambda - 1 & 0 & \vline &1 \\
         	 0  & \lambda & \vline &1
 						  }$
 có hạng bằng 2 với mọi $\lambda$ \\	
 \underline{Chú ý} : Tính điều khiển được của cặp ma trận (A,B) được xuất phát từ bài toán điều khiển $\dot{x}(t) = Ax(t) + Bu(t)$ (xem thêm $[30]$). \\
 \begin{bd}
 (Phân tích Kalman).
 \end{bd} 
Với mọi $A \in \mathbb{R}^{n \times n}, B \in \mathbb{R}^{n \times n}, \mbox{ tồn tại } S \in \mathbb{R}^{n \times n}$ khả nghịch sao cho : \\
\begin{align}
 S^{-1}AS = \begin{bmatrix}
 				A_{11} & A_{12} \\
 				0 & A_{22}
 			 \end{bmatrix} ,
 S^{-1}BS = \begin{bmatrix}
 				B_{11} & B_{12} \\
 				0 & 0
 			 \end{bmatrix}
\end{align}
 trong đó cặp $(A_{11},B{11})$ là điều khiển được. \\
 Phổ $\sigma(A_{22})$ được gọi là tập các giá trị riêng không điều khiển được của A, kí hiệu là $\sigma_{u}(A)$ (xem $[30]$).
 \begin{dl}
 Nếu cặp (A,B) không điều khiển được thì :
 $$ \sigma_{u}(A) \subset \sigma_{\sum} \cap \displaystyle \bigcup_{k}\sigma(\frac{1}{\tau}W_{k}(B\tau e^{-A\tau})+A) $$
 \end{dl}
\begin{proof}
Theo phân tích Kalman đã trình bày ở trên, ta có $\sigma(A_{22}) = \sigma_{u}(A)$.Khác với phân tích Kalman tiêu chuẩn, $B$ cũng được nhân với $S$, tuy nhiên điều này không làm thay đổi cấu trúc của nó. Do đó chúng ta có sự chuyển đổi đồng thời và không mất tính tổng quát, ta giả sử rằng $A$ và $B$ đã được chuyển đổi như dạng (2.3). Như vậy $\sigma(A_{22}) \subset \sigma_{\sum}$. Ở đây $B\tau e^{-A\tau}$ có dạng $\begin{bmatrix}
 X & Y \\
 0 & 0
 \end{bmatrix}$ và W = $\begin{bmatrix}
 W_{k}(X) & Ye^{-W_{k}(X)} \\
 0 & 0
 \end{bmatrix}$ thỏa mãn $We^{W} = B\tau e^{-A\tau}$. Vì vậy $\sigma(A_{22}) \subset \sigma(\frac{1}{\tau}W_{k}(B\tau e^{-A\tau})+A)$ với một số nhánh của hàm Lambert $W$.
\end{proof} 
 
 \begin{hq}
 Nếu cặp (B,A) không điều khiển được thì :
 $$ \frac{1}{\tau}W_{k}(\tau\sigma_{u}(B)) \subset \sigma_{\sum} \cap \displaystyle \bigcup_{k}\sigma(\frac{1}{\tau}W_{k}(B\tau e^{-A\tau})+A) $$
 \end{hq}
 \section{Phản ví dụ}
 Để minh họa rằng công thức (2.3) không áp dụng được cho hệ vi phân có trễ bất kỳ, chúng ta sẽ lấy cặp ma trận không tam giác hóa được đồng thời và không giao hoán như sau.
 Đầu tiên, ta chọn cặp ma trận tam giác  :  \\
 $A = \begin{pmatrix}
 0 & 0 \\
 \alpha & 0
 \end{pmatrix}$ , $B = \begin{pmatrix}
 0 & 1 \\
 0 & 0
 \end{pmatrix}$ \\
 với $\alpha > 0, \alpha \in \mathbb{R}$ \\
 Ta có phương trình đặc trưng là
 \begin{align}
 det(-sI + A + Be^{-s\tau}) = s^{2} - \alpha e^{-s\tau} = 0
 \end{align}
 Vì vậy giá trị riêng s được tính như sau :
 \begin{align}
 \alpha = s^{2}e^{s\tau} \Leftrightarrow \pm\frac{1}{2}\tau \sqrt{\alpha} = \frac{1}{2}s\tau e^{\frac{1}{2}s\tau}.
 \end{align}
 Trong trường hợp đặc biệt, $s_{0} = \frac{2}{\tau}W_{0}(\pm\frac{1}{2}\tau\sqrt{a})$ là giá trị riêng mà tại đó $W_{0}$ là nhánh chính của hàm Lambert W. \\
 Nếu ta cho $\tau = 1, \alpha = \pi^{2}$ và dùng $W_{0}(-\frac{1}{2}\pi) = \frac{1}{2}\pi i$ thì ta được $s_{0} = \pi i$\\
 Giả sử (3.6) vẫn đúng thì ta phải có :
 \begin{align}
 \sigma_{\sum} = \displaystyle \bigcup_{k}\sigma(Be^{-A}+A) ,
 \end{align}
 với $\tau = 1$ và $s_{0} = \pi i$ \\
 Trong phần dưới, ta sẽ chứng mình 2 tập không có tập nào chứa tập nào. Đầu tiên, ta tìm $s$ sao cho $s \in \sigma_{\sum}$ nhưng $s \notin \displaystyle \bigcup_{k}\sigma(Be^{-A}+A)$ do đó $\sigma_{\sum} \not\subset \displaystyle \bigcup_{k}\sigma(Be^{-A}+A)$. Tiếp theo, chúng ta tìm $s$ sao cho $s \in \displaystyle \bigcup_{k}\sigma(Be^{-A}+A)$ nhưng $s \notin \sigma_{\sum}$ và do đó $\sigma_{\sum} \not\supset \displaystyle \bigcup_{k}\sigma(Be^{-A}+A)$ \\
 Chú ý rằng $Be^{-A} = \begin{pmatrix}
 -\alpha & 1 \\
 0 & 0
 \end{pmatrix}$, và 
 $W_{k}(Be^{-A}) = \begin{pmatrix}
 W_{k}(-\alpha) & -\frac{1}{\alpha}W_{k}(-\alpha) \\
 0 & 0
 \end{pmatrix}$ \\
 Theo (3.9), giá trị sẽ được tính là :
 \begin{align}
 0 = s^{2} - sW_{k}(-\alpha) + W_{k}(-\alpha),
 \end{align}
 và ta tính được
 \begin{align}
 s = \frac{W_{k}(-\alpha) \pm \sqrt{W_{k}(-\alpha)^{2} - 4W_{k}(-\alpha)}}{2}
 \end{align}
 Trong trường hợp đặc biệt, với $\alpha = \pi^{2}$ và $k \in \mathbb{Z}$, giá trị riêng $s_{0} = \pi i$ sẽ thỏa mãn (3.10). Vì vậy, \\
 $0 = (i\pi)^{2} - (i\pi)W_{k}(-\pi^{2}) + W_{k}(-\pi^{2})
 = W_{k}(-\pi^{2})(1 - i\pi) - \pi^{2}$. \\
 Vậy $W_{k}(-\pi^{2}) = \frac{\pi^{2}}{1-i\pi}$. Điều đó không đúng với vài nhánh $k$ khi 
 $$-\pi^{2} = \frac{\pi^{2}}{1-i\pi}e^{\frac{\pi^{2}}{1-i\pi}} \Leftrightarrow i\pi - 1 = e^{\frac{\pi^{2}}{1-i\pi}}$$
 Lấy trị tuyệt đối của 2 vế, ta có $\sqrt{\pi^{2}+1} = e^{\frac{\pi^{2}}{1-i\pi}}$ mâu thuẫn với $\pi > e$. Do đó $\sigma_{\sum} \not\subset \displaystyle \bigcup_{k}\sigma(W_{k}(Be^{-A})+A)$.\\
 
 Ngược lại, chúng ta có thể đưa ra một ví dụ mà tại đó $\sigma_{\sum} \not\supset \displaystyle \bigcup_{k}\sigma(W_{k}(Be^{-A})+A)$. Cho $\alpha = \frac{1}{2}\pi$. Với nhánh chính của hàm Lambert W thì (2.9) trở thành
 $$s = \frac{i\pi \pm \sqrt{-\pi^{2}-8\pi i}}{4}$$
 Điều cần phải chứng minh là không phải lúc nào s cũng thỏa mãn phương trình đặc trưng $s^{2} = \frac{\pi}{2}e^{-s}$ từ (3.7) \\
 Đặt $a +bi = \pm \sqrt{\pi^{2}-8\pi i}$ với $a > 0$, chúng ta có $ab = -4\pi$, do đó $b < 0$, và $a^{2} - b^{2} = -\pi^{2}$, dẫn đến $b < -\pi$. Hơn nữa, $b > -2\pi$, mặt khác $a^{2} = b^{2} - \pi^{2} \ge 3\pi^{2}$ và $a^{2}b^{2} \le 12\pi^{4} > 16\pi^{2}$. Do đó, Re $s$ > 0 và 0 > $Im s$ > $\frac{-\pi}{4}$ dẫn đến $Im s^{2}$ < 0, $Im e^{-s}$ > 0 $\Rightarrow s^{2} \neq \frac{\pi}{2}e^{-s}$ \\
 Người ta có thể hỏi rằng, liệu phản ví dụ có phải bắt nguồn từ việc $(A,B)$ điều khiển được hay không? Điều này trong thực tế là hệ quả trực tiếp của phân tích Kalman nói rằng mọi cặp ma trận cỡ $2 \times 2$ mà không tam giác hóa được thì là điều khiển được. Mặc dù vậy, chúng ta có thể đặt bài toán của chúng ta vào hệ không điều khiển được với số chiều lớn hơn $\tSi$ như sau
 $$ \tilde{A} = \begin{pmatrix}
 A & 0\\
 0 & 1
 \end{pmatrix} , \tilde{B} = \begin{pmatrix}
 B & 0\\
 0 & 0
 \end{pmatrix} $$
 vì thế 
 $$ det(-sI + \tilde{A} + \tilde{B}e^{-s}) = (1-s)det(-sI + A + Be^{-s}),$$ tức là $\sigma_{\tilde{\sum}} = \sigma_{\sum} \cup \left\{1\right\}$ và với mọi nhánh $W_{k}$ : \\
 $det(-sI + W_{k}(\tilde{B}e^{-\tilde{A}})+\tilde{A})
 = (1-s)det(-sI + W_{k}(Be^{-A})+A)$ \\
 Từ đó 
 $\sigma(W_{k}(\tilde{B}e^{-\Bar{A}})) = \sigma(W_{k}(Be^{-A})+A) \cup \left\{1\right\}.$ \\
 Vậy câu hỏi phía trên đã được giải quyết, giá trị riêng không điều khiển được $\lambda = 1$ được chứa trong $\sigma_{\tilde{\sum}} \cap \sigma(W_{k}(\tilde{B}e^{-\tilde{A}})+A)$ \\
 \chapter{Tính toán phổ cho phương trình vi phân có trễ}
 Sử dụng hàm Lambert W, chúng ta có thể biểu diễn được phổ của hệ tam giác như sau.
 Xét hệ tổng quát 
 \begin{align} 
 \dot{x}(t) = Ax(t) + Bx(t-\tau)
 \end{align}
 với $A, B \in \mathbb{C}^{n \times n }$\\
 Ý tưởng của chúng ta ở đây sẽ là : Tìm nghiệm dạng $x(t) = e^{St}c$ với $\left\{\begin{matrix}
S \in \mathbb{R}^{n \times n}
\\ 
c \in \mathbb{R}^{n \times 1}
\end{matrix}\right.$ \\
 Thay $x(t) = e^{St}c$ vào phương trình (3.1)
 \begin{align*}
 \mbox{ta được } Se^{St}c = Ae^{St}c + Be^{S(t - \tau)}c \\
 \mbox{Vậy } (S - A - Be^{-S\tau})e^{St}c = 0 
 \end{align*}
 Do đó ta đi tìm $S$ để \begin{align}
 S - A - Be^{-S\tau} = 0 \tag{*} 
 \end{align}
 hay tương đương với $(S - A)e^{S\tau} = B$
 
 \begin{op} 
 Xét trường hợp S, A giao hoán \\
 \end{op}
 Nếu S, A giao hoán thì nhân cả 2 vế của (3.2) với $e^{-A\tau}$ ta có 
 $$ (S - A)e^{S\tau}e^{-A\tau} = Be^{-A\tau} $$
 Vì S, A giao hoán nên $e^{S\tau}e^{-A\tau} = e^{(S-A)\tau}$
 \begin{align*}
  S-A)e^{(S-A)\tau} &= Be^{-A\tau} \\
  \mbox{Do đó } (S-A)\tau &= W(Be^{-A\tau}\tau) \\
  \mbox{Vì vậy } S &= A + \frac{1}{\tau}W(Be^{-A\tau}\tau) \\ 
  \end{align*}  
 
 \begin{op}  
 Xét trường hợp S, A không giao hoán\\
 \end{op}
 Ý tưởng của bài báo (Jah) là tìm ma trận $Q$ có dạng $Q = e^{-S\tau}e^{(S-A)\tau}$. Khi đó vì $(S-A)e^{S\tau}Q = BQ$, ta có phương trình $(S-A)\tau e^{(S-A)\tau} = \tau BQ$\\
 
 Đặt $M = \tau BQ \in \mathbb{R}^{n,n}$ ($Q$ chưa biết) \\
 Khi đó $(S-A)\tau = W(M) \mbox{ ta có } S = A +\frac{1}{\tau}W(M)$. \\
 Thay lại $S$ vào phương trình $(*)$ ta có 
 \begin{align}
\frac{1}{\tau}W(M) - Be^{-S\tau} &= 0 \nonumber \\
\mbox{Vì vậy } \frac{1}{\tau}W(M) - Be^{\tau A-W(M)} &= 0 \nonumber \\
\mbox{Do đó } W(M)e^{\tau A+W(M)} &= \tau B
 \end{align}
 \begin{al}
 Do đó, việc tính toán tập phổ $\sigma_{\sum}$ được thực hiện như sau(xem $[31]$)  \\
 \end{al}
 Xét k = 0, $\pm 1$, $\pm 2$, ... \\
 1. Giải hệ phương trình phi tuyến (3.2) đối với nhánh $k$
 $$ W_{k}(M)e^{\tau A + W_{k}(M)} = \tau B $$
 2.Tìm $S_{k} = A + \frac{1}{\tau}W_{k}(M)$. \\
 3.Tính các giá trị riêng của $S_k$.\\
 Khi đó phổ của phương trình vi phân có trễ :
 $$\dot{x}(t) = Ax(t) + Bx(t - \tau)$$ 
 chính là $\displaystyle \bigcup_{k \in \mathbb{Z}}\lambda(S_{k})$. \\
 Để hệ ổn định mũ, tức là $ ||X(t)|| \leq Ce^{\alpha t}||\sup_{t \in \left[\tau,0\right]}|| $, với $C$ là hằng số(Xuất phát từ tính chất giá trị riêng của phần thực lớn nhất của phổ của (3.1)).\\
 $$\sigma := \left\{\lambda \in \mathbb{C}|det(\lambda I_{n}-A-e^{\lambda\tau}B) = 0\right\}$$
 Nghiệm của phương trình (3.1) có dạng 
 $$ x(t) = \displaystyle \sum_{k \in \mathbb{Z}}e^{S}k^{t}C_{k}^{I}, C_{k}^{I} \in \mathbb{R}^{n,1} $$ sao cho chuỗi x(t) có nghĩa. \\
 Việc xác định các $C_{k}^{I}$ được thực hiện dựa trên điều kiện ban đầu \\
 $x(t) = \phi(t)$ cho trước với mọi $t \in \left[\tau,0\right]$ \\
 \begin{vd}
 Xét phương trình : 
 \begin{align*}
 \dot{x}(t) = \begin{bmatrix}
 -1 & -3 \\
 2 & -5
 \end{bmatrix} x(t) + \begin{bmatrix}
 1.66 & -0.697 \\
 0.93 & -0.33
 \end{bmatrix}x(t-1) \\
 \end{align*}
 \begin{align*}
 \phi(t) = x(t) = \begin{bmatrix}
 1 \\
 0
 \end{bmatrix} \forall t \leq 0 \\
 \end{align*}
 \end{vd} 
 \begin{verbatim}
 A = input('Nhap ma tran A : ');
 B = input('Nhap ma tran B :');
 tau = input('Nhap tre tau : ');
 A = jordan(A)
 B = jordan(B)
 M = tau*B*exp(-tau*A)
 S = (1/tau)*lambertw(M) + A
 lambda = eig(S)
 \end{verbatim}
 Kết quả sẽ như sau 
 \begin{verbatim}
 Nhap ma tran A : [-1 -3;2 -5]
A =
    -1    -3
     2    -5
Nhap ma tran B : [1.66 -0.697;0.93 -0.33]
B =
    1.6600   -0.6970
    0.9300   -0.3300
Nhap tre tau : 1
tau =
     1
A =
  -3.0000 - 1.4142i   0.0000 + 0.0000i
   0.0000 + 0.0000i  -3.0000 + 1.4142i
B =
    0.0804         0
         0    1.2496
M =
   0.2517 + 1.5941i   0.0804 + 0.0000i
   1.2496 + 0.0000i   3.9142 -24.7928i
S =
  -2.3864 - 0.7922i   0.0746 + 0.0000i
   0.6514 + 0.0000i  -0.6979 + 0.4110i
lambda =
  -2.4055 - 0.7789i
  -0.6787 + 0.3976i
 \end{verbatim}

 
 
 	 


 
 
 
 
 
 
 
 
 
 
 
