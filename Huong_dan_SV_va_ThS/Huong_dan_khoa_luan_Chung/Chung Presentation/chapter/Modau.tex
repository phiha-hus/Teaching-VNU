\addcontentsline{toc}{chapter}{Mở đầu}
\begin{center}
\textbf{\large {MỞ ĐẦU}}
\end{center}
\noindent 
Một trong những đòi hỏi đặt ra trong lý thuyết điều khiển chính là việc thiết kế hệ thống điều khiển, giúp cho máy móc có một hiệu suất làm việc hợp lý dưới nhiều điều kiện đầu vào và các yếu tố gây nhiễu khác nhau. Lý thuyết điều khiển \hinf được sử dụng để có thể giảm các lỗi mô hình hóa và các nhiễu không xác định trong một hệ thống, đồng thời cung cấp khả năng tối ưu hóa một cách định lượng được đối với một bài toán quy mô lớn có nhiều biến tham gia. Trong thực tế hầu hết các nghiệm của bài toán điều khiển \hinf thực ra chỉ là các bộ điều khiển dưới mức tối ưu, nghĩa là hàm chuyển của bộ điều khiển đạt đến một giới hạn định trước với chuẩn \hinf. Trong cách thiết lập bài toán này, kết quả là bền vững đối với một giới hạn cho trước. Tuy nhiên, nếu ta tìm thấy chuẩn \hinf, ta sẽ có câu trả lời cho sự tồn tại của bộ điều khiển trong một phạm vi nhiễu động nhất định đối với hệ thống.

\medskip
George Zames, \cite{23} đưa ra lý thuyết điều khiển \hinf bằng cách xây dựng việc giảm độ nhạy như một vấn đề trong tối ưu với toán tử chuẩn, cụ thể là chuẩn \hinf  \cite{15}. Tại đó \hinf là không gian của mọi hàm giải tích và bị chặn có giá trị ma trận, bị giới hạn trong nửa bên phải mặt phẳng phức. Cách thiết lập bài toán này hoàn toàn dựa trên miền tần số. Zames gợi ý rằng việc sử dụng chuẩn \hinf làm thước đo hiệu suất sẽ đáp ứng tốt hơn nhu cầu ứng dụng so với thiết kế điều khiển dạng toàn phương tuyến tính Gauss \cite{25}. Từ đó, ta có thiết kế của điều khiển tối ưu \hinf  nhằm mục đích tìm ra một bộ điều khiển có thể ổn định được hệ thống trong khi giảm thiểu tác động của nhiễu gây ra. Chuẩn \hinf được sử dụng để đánh giá số độ nhạy, độ bền vững và hiệu suất của bộ điều khiển của hệ thống phản hồi vòng kín.

\medskip
Phương pháp miền tần số là một trong hai phương pháp chính để nghiên cứu \hinf của hệ thống điều khiển, chính vì vậy trong luận văn này chúng ta sẽ đi tìm hiểu hai thuật toán kinh điển đầu tiên, được trình bày trong các bài báo \cite{3}, \cite{4} để tính toán chuẩn này. Các thuật toán này vẫn còn được tham khảo và trích dẫn cho đến nay (2021).