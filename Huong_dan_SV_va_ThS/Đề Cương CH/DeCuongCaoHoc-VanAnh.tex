\documentclass[12pt,oneside,portrait,a4paper]{book}
\usepackage[utf8]{vietnam}
\usepackage[mathscr]{eucal}
\usepackage{amsfonts}
\usepackage{amsmath,amsxtra,latexsym,amsthm, amssymb, amscd,amsfonts}
\usepackage{enumerate}
\usepackage{fancyhdr}
\usepackage{graphics}
\usepackage{xr}
\usepackage{utopia}

\usepackage[portrait, top=2.5cm, bottom=2.5cm, left=3.5cm, right=2.0cm] {geometry}

\theoremstyle{definition}
\newtheorem{dn}{Định nghĩa}[section]
\newtheorem{nx}[dn]{Nhận xét}
\newtheorem{vd}[dn]{Ví dụ}
\newtheorem{cy}[dn]{Chú ý}

\theoremstyle{plain}
\newtheorem{dl}[dn]{Định lí}
\newtheorem{md}[dn]{Mệnh đề}
\newtheorem{bd}[dn]{Bổ đề}
\newtheorem{hq}[dn]{Hệ quả}

\newcommand{\bdn}{\begin{dn}}
\newcommand{\edn}{\end{dn}}
\newcommand{\bdl}{\begin{dl}}
\newcommand{\edl}{\end{dl}}
\newcommand{\bmd}{\begin{md}}
\newcommand{\emd}{\end{md}}

\newcommand{\bE}{\mathbb{E}}

\newcommand{\bhq}{\begin{hq}}
\newcommand{\ehq}{\end{hq}}
\newcommand{\bbd}{\begin{bd}}
\newcommand{\ebd}{\end{bd}}
\newcommand{\bvd}{\begin{vd}}
\newcommand{\evd}{\end{vd}}
\newcommand{\bnx}{\begin{nx}}
\newcommand{\enx}{\end{nx}}
\newcommand{\bcy}{\begin{cy}}
\newcommand{\ecy}{\end{cy}}
\newcommand{\bpr}{\begin{proof}\item[]}
\newcommand{\epr}{\end{proof}}

\newcommand{\Ass}{\textup{Ass}}
\newcommand{\Ann}{\textup{Ann}}
\newcommand{\Supp}{\textup{Supp}}
\newcommand{\Ima}{\textup{Im}}
\newcommand{\Ker}{\textup{Ker}}

\renewcommand{\labelenumi}{(\roman{enumi})}

%%%%%%%%%%%%%%%%%%%%%%%
\begin{document}

\thispagestyle{empty}
\begin{titlepage}
\centerline{\fontsize{14pt}{18pt}\bf TRƯỜNG ĐẠI HỌC SƯ PHẠM HÀ NỘI}
\centerline{\Large\bf KHOA TOÁN - TIN}
\centerline{--------------------------o0o--------------------------}
\vspace*{4cm}
\centerline{\fontsize{14pt}{18pt}\bf ĐỀ CƯƠNG LUẬN VĂN THẠC SĨ}
\vspace*{2cm}

\begin{center}
\centerline{\fontsize{14pt}{18pt}\bf Tên đề tài}
\end{center}
\centerline{\fontsize{14pt}{18pt}\bf PHÂN TÍCH TÍNH CHẤT ỔN ĐỊNH CỦA CÁC HỆ ĐỘNG LỰC CÓ TRỄ}
\centerline{\fontsize{14pt}{18pt} \bf SỬ DỤNG PHƯƠNG PHÁP HÀM LAMBERT VÀ ỨNG DỤNG}
\vspace*{0,2cm}


\vspace*{4cm}
\begin{center}
\begin{tabular}{l l}
\hspace*{-0,5cm}{\fontsize{14pt}{18pt}\bf \textit{Chuyên ngành}}&{\fontsize{14pt}{18pt}\bf : Toán Ứng Dụng}\\
\hspace*{-0,5cm}{\fontsize{14pt}{18pt}\bf \textit{Mã số}}&{\fontsize{14pt}{18pt}\bf : 8.46.01.12 }\\
\hspace*{-0,5cm}{\fontsize{14pt}{18pt}\bf \textit{Học viên}}&{\fontsize{14pt}{18pt}\bf : Nguyễn Thị Vân Anh}\\
\hspace*{-0,5cm}{\fontsize{14pt}{18pt}\bf \textit{Giảng viên hướng dẫn}}&{\fontsize{14pt}{18pt}\bf : TS. Hà Phi}\\
\end{tabular}
\end{center}
\vfill
\centerline{\fontsize{14pt}{18pt}\bf HÀ NỘI - 2019}
\end{titlepage}
%%%%%%%%%%%%%%%%%%%%%%%  
%\pagestyle{fancy}
%\renewcommand{\chaptermark}[1]{\markboth{\chaptername\ \thechapter. \sc #1}{}}
%\lhead[\fancyplain{}{}]{\fancyplain{}{\leftmark}}
%\rhead{}\sloppy
%\cfoot[\fancyplain{}{}]{\fancyplain{}{\thepage}}

\pagenumbering{roman}
\fontsize{12pt}{18pt}\selectfont


\newpage

\pagenumbering{arabic}       %%%%%Đánh số trang! Cấm bỏ!
\renewcommand{\baselinestretch}{1.5}

%%%%%%%%%%%%%%%%%%%%%%%
%\begin{flushleft}
\noindent 
\textbf{ I. LÝ DO CHỌN ĐỀ TÀI}\\
\hspace*{0,5cm}

Các hệ có trễ biểu diễn các hệ động lực có chứa độ trễ thời gian trong hệ thống, hoặc trễ được sử dụng như một công cụ để và kiểm soát các tính chất mong muốn của hệ thống, ví dụ như tính ổn định, tính được hay quan sát được, v.v. Độ trễ về mặt thời gian như vậy là rất phổ biến trong các hệ động lực hay hệ thống trong khoa học và kỹ thuật, và có thể dẫn đến một số vấn đề không mong muốn như sự không ổn định và thiếu chính xác, và do đó, hạn chế và làm giảm hiệu suất có thể đạt được của các hệ điều khiển. Thêm vào đó, bởi vì các phương trình vi phân có trễ là các hệ động lực vô hạn chiều, việc phân tích các hệ có trễ bằng các phương pháp cổ điển được phát triển cho các hệ hữu hạn chiều là không khả thi.

Trong hai thập niên gần đây, các hàm Lambert W được nghiên cứu và sử dụng như một phương pháp tiếp cận hiệu quả cho các hệ phương trình vi phân đơn trễ (tức là chỉ có một trễ) với hệ số hằng số. 
Cách tiếp cận sử dụng hàm Lambert W dẫn đến công thức nghiệm hiển cho các phương trình vi phân có trễ và cho phép nghiên cứu sâu hơn các đặc tính của tập phổ cũng như tính chất ổn định của nghiệm. Từ phương diện định lượng, những công trình nghiên cứu tiên phong gần đây của Yi, Nelson và Ulsoy (2008-2012) dẫn đến sự ra đời của Toolbox LambertWDDE được lập trình trong ngôn ngữ tính toán khoa học MATLAB. Trên phương diện định tính, tính chất ổn định của nghiệm của hệ tuyến tính có trễ (sử dụng phương pháp hàm Lambert W) rất được quan tâm nghiên cứu trong 10 năm trở lại đây.

Với mong muốn được tìm hiểu kĩ hơn về tính chất ổn định của hệ có trễ và ứng dụng trong thực tế, tôi quyết định chọn đề tài \textbf{``Phân tích tính chất ổn định của các hệ động lực có trễ sử dụng phương pháp hàm Lambert và ứng dụng''} cho luận văn thạc sĩ của mình. 

\vspace{6pt}
\noindent 
\textbf{ II. MỤC TIÊU NGHIÊN CỨU}\\
\hspace*{0,5cm} Nghiên cứu phương pháp xấp xỉ nghiệm cho hệ phương trình vi phân tuyến tính có trễ. 
Nghiên cứu tính chất ổn định nghiệm của hệ phương trình vi phân tuyến tính có trễ sử dụng phương pháp hàm Lambert. 
Áp dụng các kết quả nghiên cứu trong một số mô hình thực tế.   

\vspace{6pt}
\noindent 
\textbf{ III. ĐỐI TƯỢNG NGHIÊN CỨU}
\begin{itemize}
\item Phương pháp xấp xỉ nghiệm của hệ phương trình vi phân tuyến tính có trễ.
\item Tính chất ổn định của nghiệm của hệ phương trình vi phân tuyến tính có trễ.
\item Hàm Lambert W
\end{itemize}

\pagebreak

\vspace{6pt}
\noindent 
\textbf{ IV. PHƯƠNG PHÁP NGHIÊN CỨU}\\
\hspace*{0,5cm} Nghiên cứu lý luận. Nghiên cứu thực hành.

\vspace{6pt}
\noindent 
\textbf{ V. CẤU TRÚC LUẬN VĂN}\\
\hspace*{1cm} Nội dung của luận văn bao gồm hai chương:


\hspace*{0,5cm}\textbf{Chương I: Xấp xỉ nghiệm của hệ có trễ sử dụng hàm Lambert W}\\
\hspace*{1cm} 1.1 Sơ lược về hàm Lambert W \\
\hspace*{1cm} 1.2 Công thức nghiệm của các hệ có trễ \\
\hspace*{1cm} 1.3 Tính toán nghiệm gần đúng của các hệ có trễ 

\vskip 0.5cm 
\hspace*{0,5cm}\textbf{Chương II: Phân tích tính ổn định của các hệ có trễ sử dụng phương pháp hàm Lambert và ứng dụng}\\
\hspace*{1cm} 2.1 Tính chất ổn định của các hệ phương trình vi phân tuyến tính có trễ \\
\hspace*{1cm} 2.2 Một số ví dụ thực tế \\

\vspace{6pt}
\noindent 
\textbf{ VI. KẾ HOẠCH THỰC HIỆN}\\
\hspace*{1cm} - Từ tháng 9 đến tháng 11 năm 2020: Gặp giảng viên hướng dẫn, nhận đề tài, lập và bảo vệ đề cương.\\
\hspace*{1cm} - Từ tháng 12 năm 2020 đến tháng 5 năm 2021: Nghiên cứu kết hợp trao đổi với giảng viên hướng dẫn để viết luận văn.\\
\hspace*{1cm} - Tháng 6 năm 2021: Hoàn chỉnh luận văn và bảo vệ trước Hội đồng chấm luận văn Thạc sĩ.\\
%\end{flushleft}

\begin{thebibliography}{99}
\bibitem{IvR15} 
I. Ivanoviene and J.	Rimas,	\emph{Complement	to	method	of	analysis	of	time	delay	systems	via	the	Lambert	W	function},	Automatica,	Vol.	54,	2015,	pp.	25-28.

\bibitem{CepM15} R. Cepeda-Gomez	and	W.	Michiels, \emph{Some	special	cases	in	the	stability analysis	of	multi-dimensional	time	delay	systems	using	the	matrix	Lambert	W	function},	Automatica, Vol.	53,	2015,	pp.	339-345.

\bibitem{Yi12} 
S. Yi, S. Duan, P. Nelson, A. Ulsoy, \emph{The Lambert W function approach to time delay systems and the lambertw-dde toolbox}. In Proceedings of the 10th IFAC Workshop on Time Delay Systems, volume 10 (2012), pp. 114-119

\bibitem{UlG20}
G. Ulsoy and R. Gitik, \emph{On the Convergence of the Matrix Lambert W Approach to Solution of Systems of Delay Differential Equations}, J. Dyn. Sys., Meas., Control., volume 142 (2020)

\end{thebibliography}
\vspace{10pt}

\phantom{***}

\begin{flushright}
\textit{Hà Nội, tháng 10 năm 2020}
\end{flushright}
\par
\par
{\small
	\hspace {-0,1cm} Học viên \hspace {0.5cm} Giảng viên hướng dẫn \hspace {0.7cm} Trưởng bộ môn \hspace {1.2cm} Phó trưởng khoa \\
	\par
	\vspace{1cm}
	\noindent  \hspace{4.2cm} TS \hspace{3cm} PGS.TS \hspace{2.7cm} PGS.TS\\
	\noindent Nguyễn T. Vân Anh  \hspace*{0.9cm} Hà Phi \hspace{2,4cm}Ngô Hoàng Long \hspace{1,1cm} Nguyễn Công Minh
	
}

\end{document}
 
 

%%23:1:5 8/11/2013Last Modification of contents