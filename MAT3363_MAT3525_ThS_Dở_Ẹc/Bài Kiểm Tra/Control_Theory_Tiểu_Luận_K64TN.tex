\documentclass[11pt]{article}
\usepackage{amsmath}
%\usepackage{extsizes}
\usepackage{amsmath,amssymb}
%\usepackage{omegavn,ocmrvn}
%\usepackage[utf8x]{inputenc}
\usepackage[utf8]{vietnam}

\usepackage{listings}
\lstset{language=Python}          % Set your language (you can change the language for each code-block optionally)


\usepackage{longtable}
\usepackage{answers}
\usepackage{graphicx}
\usepackage{array}
\usepackage{pifont}
\usepackage{picinpar}
\usepackage{enumerate}
\usepackage[top=3.0cm, bottom=3.5cm, left=3.5cm, right=2.5cm] {geometry}

\usepackage{extarrows}
\usepackage{hyperref}


\newtheorem{bt}{Câu}
\newcommand{\RR}{\mathbb R}
\Newassociation{sol}{Solution}{ans}
\newtheorem{ex}{Câu}
\renewcommand{\solutionstyle}[1]{\textbf{ #1}.}


\begin{document}
% \noindent

\begin{tabular*}
	{\linewidth}{c>{\centering\hspace{0pt}} p{.7\textwidth}}
	Trường ĐHKHTN, ĐHQGHN & {\bf Học Kỳ 2 (2021-2022)}
	\tabularnewline
	K64 TTƯD - Thầy Hà Phi & {\bf Bài Tập Giải Tích Số \\ \today}
	% Exercises on pages 239, 240 Cheney/Kincaid are really nice
	\tabularnewline
	\rule{1in}{1pt}  \small  & \rule{2in}{1pt} %(Due date:)
	\tabularnewline
	%  \tabularnewline
	%  &(Đề thi có 1 trang)
\end{tabular*}




\begin{center}
	{\bf Bài Tiểu Luận - Lý Thuyết Điều Khiển Hệ Thống - K64TN}
\end{center}

\textbf{Các em cùng nhóm chỉ cần nộp 1 bản cứng Tiểu Luận + 1 bản mềm vào Google Classroom, ghi tên đầy đủ các bạn trong nhóm theo mẫu thầy gửi kèm.} \\

\hrule

\begin{center}
	{\bf Đề 1: Tính điều khiển được (3 bạn làm)}
\end{center}

\begin{bt}(\textbf{7 điểm}) \\
Hãy đọc, tìm hiểu và dịch các phần liên quan đến tính chất điều khiển được của hệ suy biến (singular systems) trong các trang \textbf{121 - 142} (Section 4.1 \& 4.2) trong Part 1, cuốn sách của Duan. 
\end{bt}

Một hệ thống điều khiển dạng 
%
\begin{align}\label{0}
	\dot{x}(t) &= A x(t) + B u(t), \quad \forall t\geq 0, \\
	y(t) &=  C x(t), \notag
\end{align}
%
được gọi là \textbf{$L$-điều khiển được một phần ($L$-partially controllable)}, trong đó $L$ là một ma trận có cỡ phù hợp, nếu thành phần $Lx$ là điều khiển được.
Thông thường nếu $A \in \RR^{n,n}$ thì $L \in \RR^{\ell,n}$ và có đủ hạng dòng, trong đó $\ell < n$. Một ví dụ tiêu biểu của việc điều khiển được một phần là không phải mọi hệ cơ học bậc hai
%
\begin{equation}\label{1}
	M \ddot{x}(t) + D \dot{x}(t) + K x(t) = B u(t), \mbox{ for all } t\geq 0,
\end{equation}
%
đều có thể điều khiển được cả vị trí ($x$) và vận tốc ($\dot{x}$). Khi đó nếu chỉ quan tâm đến điều khiển vị trí và hệ thống \eqref{1} được chuyển về dạng bậc nhất với biến
trạng thái mới $z(t) = \m{x(t) \\ \dot{x}(t)}$, thì ta có thể lấy $L = \m{I_n & 0}$.


\begin{bt}(\textbf{3 điểm}) \\
Xét một hệ thống điều khiển suy biến (hệ mô tả/descriptor systems) có dạng 
%
\begin{align}\label{2}
	E \dot{x}(t) &= A x(t) + B u(t), \quad \forall t\geq 0,  \\
	y(t) &=  C x(t), \notag
\end{align}
%
trong đó $E$, $A \in \RR^{n,n}$, $B \in \RR^{n,p}$. Cho trước ma trận $L \in \RR^{\ell,n}$ và có đủ hạng dòng, trong đó $\ell < n$. \\
Hệ điều khiển (\eqref{0} hay \eqref{2}) được gọi là $L$-điều khiển được một phần, nếu đối với bất kỳ điều kiện ban đầu nào $x(0) = x_0$ và bất kỳ trạng thái cuối cùng nào $x_1$ luôn tồn tại $t_1 > 0$ và một hàm đầu vào $u(t)$ sao cho nghiệm của hệ \eqref{2} thỏa mãn $L x(t_1;t_0,x_0,u) = L x_1$.\\  
a) Tham khảo bài báo của Bashirov. Hãy xây dựng các điều kiện để kiểm tra cho tính $L$-điều khiển được một phần của hệ \eqref{0}. Yêu cầu phải có ít nhất 2 trong 3 điều kiên: (i) Điều kiện Gramian; (ii) Điều kiện tương tự Hautus; (iii) Điều kiện tương tự ma trận điều khiển Kalman. \\
b) Xây dựng ít nhất một điều kiện để kiểm tra cho tính $L$-điều khiển được một phần của hệ mô tả \eqref{2}.
\end{bt}

\hrule

\begin{center}
	{\bf Đề 2: Tính quan sát được (2 bạn làm)}
\end{center}

\begin{bt}(\textbf{7 điểm}) \\
Hãy đọc, tìm hiểu và dịch các phần liên quan đến tính chất quan sát được của hệ suy biến (singular systems) trong các trang \textbf{142 - 156} (Section 4.3 \& 4.4) trong Part 1, cuốn sách của Duan. 
\end{bt}

\begin{bt}(\textbf{3 điểm}) \\
	Xét một hệ thống điều khiển suy biến (hệ mô tả/descriptor systems) có dạng 
	%
	\begin{align}\label{02}
		E \dot{x}(t) &= A x(t) + B u(t), \quad \forall t\geq 0,  \\
		y(t) &=  C x(t), \notag
	\end{align}
	%
	trong đó $E$, $A \in \RR^{n,n}$, $B \in \RR^{n,p}$. Hãy đọc Mục 6.7, Chương 6 cuốn sách của Chen để hiểu về việc bảo toàn tính chất điều khiển được qua quá trình lấy mẫu/sampling. \\
	a) Dựa vào tính chất đối ngẫu, hãy phát biểu và chứng minh Định lý tương tự Định lý 6.9 cho tính quan sát được khi lấy mẫu hệ điều khiển
	%
	\begin{align}\label{00}
		\dot{x}(t) &= A x(t) + B u(t), \quad \forall t\geq 0, \\
		y(t) &=  C x(t), \notag
	\end{align}
	%
	\noindent b) Hãy mở rộng các kết quả nói trên cho hệ mô tả \eqref{02}.
\end{bt}


\end{document}



