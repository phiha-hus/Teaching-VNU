\documentclass[11pt]{article}
\usepackage{amsmath}
%\usepackage{extsizes}
\usepackage{amsmath,amssymb}
%\usepackage{omegavn,ocmrvn}
%\usepackage[utf8x]{inputenc}
\usepackage[utf8]{vietnam}

\usepackage{listings}
\lstset{language=Python}          % Set your language (you can change the language for each code-block optionally)


\usepackage{longtable}
\usepackage{answers}
\usepackage{graphicx}
\usepackage{array}
\usepackage{pifont}
\usepackage{picinpar}
\usepackage{enumerate}
\usepackage[top=3.0cm, bottom=3.5cm, left=3.5cm, right=2.5cm] {geometry}

\usepackage{extarrows}
\usepackage{hyperref}


\newtheorem{bt}{Câu}
\newcommand{\RR}{\mathbb R}
\Newassociation{sol}{Solution}{ans}
\newtheorem{ex}{Câu}
\renewcommand{\solutionstyle}[1]{\textbf{ #1}.}


\begin{document}
% \noindent

\begin{tabular*}
	{\linewidth}{c>{\centering\hspace{0pt}} p{.5\textwidth}}
	ĐẠI HỌC QUỐC GIA HÀ NỘI	
	 & {ĐỀ THI KẾT THÚC HỌC PHẦN}  
	\tabularnewline
	TRƯỜNG ĐẠI HỌC KHOA HỌC TỰ NHIÊN & {HỌC KỲ II, NĂM HỌC 2021-2022}
	% Exercises on pages 239, 240 Cheney/Kincaid are really nice
	\tabularnewline
	\rule{3in}{1pt}  \small  & \rule{2in}{1pt} %(Due date:)
	\tabularnewline
	%  \tabularnewline
	%  &(Đề thi có 1 trang)
\end{tabular*}

\def\hro{\mathbb}
\def\vphi{\varphi}
\def\tet{\theta}
\def\a{\alpha}
\def\b{\beta}
\def\rar{\rightarrow}
\def\R{\hro{R}}
\def\C{\hro{C}}
\def\Si{\Sigma}
\def\si{\sigma}
\def\ep{\varepsilon}
\def\rank{\mathrm{rank}}
\newcommand{\m}[1]{
	\begin{bmatrix}
		#1
	\end{bmatrix}
}


\begin{center}
	Tên học phần: {\bf Tính Toán Khoa Học} \\ 
	Mã học phần: \textbf{MAT3525}	\quad Số tín chỉ: \textbf{03} \quad	Đề số: \textbf{02} \\ 
	% Dành cho sinh viên lớp học phần: MAT3525 \\
	Thời gian làm bài: \textbf{90 phút} (không kể thời gian phát đề) \quad Đề bao gồm: \textbf{01 trang}
\end{center}


\begin{bt}
	Cho hệ thống điều khiển với các tham số $\a$, $\b$ như sau
	%
	\begin{align}\label{eq1}
		\dot{x}(t) &= \m{0 & 1 & 0 \\ 1 & 0 & 0 \\ -2 & 1 & 2} x(t) + \m{1 \\ 2 \\ \a} u(t), \\
		y(t) &= \m{1 & \b & 1} x(t). 
	\end{align}
	%
	a) Hãy tìm hàm truyền của hệ và các không điểm, cực, lợi của hàm truyền. Tính gần đúng đến 4 chữ số thập phân. \\	
	b) Tìm ma trận điều khiển Kalman và quan sát Kalman của hệ. \\
	c) Tìm điều kiện của $\a$, $\b$ để hệ điều khiển \eqref{eq1} là điều khiển được. \\
	d) Vẽ biểu đồ mô phỏng của hệ thống điều khiển trên.
\end{bt}


\begin{bt}\label{Câu 1}
	Một hệ thống điều khiển trong cơ khí có dạng như trong Hình \ref{fig:mechanicalsystem} dưới đây. 
	
	\begin{figure}[!h]
		\centering
		\includegraphics[width=0.7\linewidth]{Mechanical_System}
		\caption[Hệ thống điều khiển cơ học]{Hệ thống điều khiển cơ học gồm cả lò xo, pittông và vật nặng.}
		\label{fig:mechanicalsystem}
	\end{figure}
	
	a) Từ biểu đồ mô phỏng bên phải hãy viết công thức không gian trạng thái của hệ thống điều khiển đó. \\
	b) Ý nghĩa của đầu vào $u$, đầu ra $y$ và vector trạng thái $x$ là gì? \\
	c) Hãy tìm hàm truyền của hệ và các không điểm, cực của hàm truyền. Tính gần đúng đến 3 chữ số thập phân. \\
	d) Hãy tìm điều kiện cần và đủ của $k$ (\textbf{độ cứng lò xo}), $b$ (\textbf{hệ số đẩy}), $m$ (\textbf{khối lượng vật nặng}) sao cho hệ hở là ổn định.
\end{bt}


\begin{center}
	--------------------------- Hết ---------------------------
\end{center}

\end{document}



